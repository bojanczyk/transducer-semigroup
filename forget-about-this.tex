\tito{Regular functions over arbitrary semigroups!\\
  Instead of just \enquote{transducer semigroups} we're going to prepend the adjective \enquote{functorial}, to distinguish them from the new \enquote{register transducer semigroups}. The notion of regular semigroup-to-semigroup function that we consider is closed under composition (\Cref{rem:composition}) and when the codomain is finite, it coincides with recognizable functions.}

\section{(Attempt at an alternative) Introduction}

The starting point of this paper is the fundamental class of regular
string-to-string functions. It has many equivalent descriptions: {\color{red}\bf
  [\ldots]}

Some generalizations of this class are well understood. For instance, MSO transductions, streaming string transducers\footnote{The suitable generalization of the copyless streaming string transducer is the \emph{single use restricted macro tree transducer}~\cite{MacroMSO} which predates SSTs by two decades while being very similar. A more recent variant, the \emph{register tree transducer}, appears in~\cite[Section~4]{FOTree}.\tito{sur vs copyless + amina's thing is actually copyless but uses fo-relabeling}} and the formalisms based on the linear $\lambda$-calculus\footnote{The results of~\cite{LambdaTransducer} applies \enquote{out of the box} to trees. As for the \enquote{implicit automata in typed $\lambda$-calculi} approach~\cite{IATLC,titoPhD}, its extension to trees requires the use of the \emph{additive connectives} of linear logic.} admit extensions to tree-to-tree functions that are still equivalent, leading to a robust notion of regular tree functions for which functional combinators can also be designed~\cite{FOTree}.

The present paper is concerned with a different and new generalization of the
string-to-string case: we propose a notion of regular functions \emph{between
  arbitrary semigroups}, which coincides with the regular string
functions when the semigroups under consideration are free monoids. Some of the
aforementioned models, such as two-way automata or streaming string transducers,
make sense when the output belongs to an arbitrary semigroup; however, all of
them depend on the string structure of the input. This is why we introduce two
new devices that can operate over any semigroup, which we prove to be equivalent
in expressive power to argue for the robustness of our definition. Their
specialization to free monoids adds two new characterizations of the regular
string-to-string functions to the above list.
\begin{description}
\item[Register transducer semigroups] are to streaming string transducers what
  semigroups or monoids are to deterministic finite automata. An SST basically consists of a DFA plus some string-valued registers, while a register transducer semigroup consists of a finite control semigroup plus some registers with values in the output semigroup. A significant difference made by replacing the DFA by a semigroup is that we do not need any \enquote{copyless} or \enquote{bounded-copy} condition anymore to capture precisely the regular functions.
    \item[Functorial transducer semigroups] use minimal syntax, and refer only to basic concepts from algebra and category theory. The characterization is concise enough to be fully stated here: a regular function $A\to B$ is one that can be decomposed as
    \[ A \xrightarrow{\;\text{some semigroup homomorphism}\;} \functor B \xrightarrow{\;\outfun_B\;} B \]
    where $\functor$ is some finiteness-preserving functor from the category of semigroups to itself and the output function $\outfun_{B}$ is a natural transformation from $\mathsf{U}\functor$ to $\mathsf{U}$ -- here $\mathsf{U}$ is the forgetful functor from semigroups to sets (so $\mathsf{U}B = B$ on objects).
\end{description}
From the category-theoretic definition, it is easy to show that regular semigroup-to-semigroup functions are \emph{closed under composition}.
\tito{refer to difficulties with 2DFT / SST composition here? (as in \Cref{rem:composition})}

\section{Preliminaries: streaming string transducers}

\newcommand{\rmst}{\mathrm{st}}
\newcommand{\rmreg}{\mathrm{reg}}
\newcommand{\rmout}{\mathrm{out}}
\newcommand{\deltast}{\delta_\rmst}
\newcommand{\deltareg}{\delta_\rmreg}
\newcommand{\deltaout}{\delta_\rmout}
\newcommand{\ttout}{\mathtt{out}}
\newcommand{\ttreverse}{\mathtt{rev}}
\newcommand{\tterase}{\mathtt{erase}}
\newcommand{\SubsTrans}{\mathrm{SubsTrans}}
\newcommand{\substcomp}{\odot}
\newcommand{\regassign}{\mathbin{:=}}

Before formally defining streaming string transducers (SSTs), let us start with an example:
\begin{example}\label{ex:cecilia}
  The transducer below is a deterministic finite automaton with two states over the alphabet $\Sigma = \Gamma\cup\{\#\}$ (with $\#\notin\Gamma$), enriched with two \emph{registers} $X$ and $Y$.  
\begin{center}
  \begin{tikzpicture}[->,node distance=1.5cm]
    %[->,node distance=3.5cm,transform canvas={scale=1}]
    %every text node part/.style={font=\footnotesize}]
    \node[state] (A) {0} ;
    \node (D) [right of=A] {};
    \node (E) [right of=D] {};
    \node[state] (B) [right of=E] {1};
    \node (I) [below of=A] {};
    \node (OA) [above of=A] {};
    \node (OB) [above of=B] {};
  
  \path
    (I) edge node[left, style={font=\footnotesize}]
   {$\begin{array}{l@{~}c@{~}l} X &\regassign& \varepsilon \\ Y &\regassign& \varepsilon \end{array}$} (A) ;
  \path
    (A) edge node[left, style={font=\footnotesize}]
   {$YX$} (OA) ;
  \path
    (B) edge node[right, style={font=\footnotesize}]
   {$\varepsilon$} (OB) ;
  \path
    (A) edge [bend left] node[above, style={font=\footnotesize}]
   {$\#\left| \begin{array}{l@{~}c@{~}l} X &\regassign& \varepsilon \\ Y &\regassign& YX \end{array} \right.$} (B) ;
  \path (B) edge [loop right] node[right,style={font=\footnotesize}]
  {$a \in \Gamma \left| \begin{array}{l@{~}c@{~}l} X &\regassign& aX \\ Y &\regassign& Y \end{array} \right.$}
  (A) ;
  \path (B) edge [bend left] node[below,style={font=\footnotesize}]
   {$\#\left| \begin{array}{l@{~}c@{~}l} X &\regassign& \varepsilon \\ Y &\regassign& XY \end{array} \right.$}
  (A) ;
  \path (A) edge [loop left] node[left,style={font=\footnotesize}]
  {$a \in \Gamma \left| \begin{array}{l@{~}c@{~}l} X &\regassign& Xa \\ Y &\regassign& Y \end{array} \right.$}
  (A) ;
  \end{tikzpicture}
\end{center}
It computes the string-to-string function:
  \[
  \begin{array}{rcl!{\hspace{7em}} l}
  (\Gamma \sqcup \{ {\#} \})^* &\to& \Gamma^*
  & {(w_i \in \Gamma^*,\; (\Gamma \sqcup \{\#\})^* \cong \Gamma^*(\#\Gamma^*)^*)}\\
  w_0\#\ldots\#w_{2k} &\mapsto& \multicolumn{2}{l}{
    \ttreverse(w_{2k-1})
    \ttreverse(w_{2k-3})
    \ldots
    \ttreverse(w_1)
    w_0
    w_2
    \ldots
    w_{2k}} \\
  w_0\#\ldots\#w_{2k+1} &\mapsto& \varepsilon
  \end{array}
  \]
The SST starts out in the initial state $0$, with the initial values of $X$ and $Y$ both set to $\varepsilon$. At each transition, the new values of the registers are computed from the old ones by a parallel assignment. So for example, after reading $aab\#bbc$ (for $a,b,c\in\Gamma$), we are in state $1$ with the register contents $X = cbb$ and $Y = aab$. Finally, if the final state is $0$, the transducer outputs $(\text{final value of}\ Y)\cdot(\text{final value of}\ X)$; if the final state is $1$, it outputs the empty word.
\end{example}

Formally, a \emph{substitution} for the register set $R$ over the alphabet $\Gamma$ is a map $\sigma\colon R \to (\Gamma \cup R)^*$ (assuming that $R \cap \Gamma = \varnothing$). Typically, we have $\sigma(X) = Xa$ and $\sigma(Y) = Y$ for the leftmost register update in the state diagram of \Cref{ex:cecilia}. Substitutions \emph{compose}: define
\[ \sigma \substcomp \tau = (\text{the extension of}\ \sigma\ \text{to a monoid endomorphism of}\ (\Gamma \cup R)^*\ \text{fixing}\ \Gamma) \circ \tau \]
for $\sigma,\tau \colon R \to (\Gamma \cup R)^*$. There is some \enquote{contravariance} going on in this definition: even though we use $f \circ g$ to mean \enquote{apply $g$ then $f$} as usual, the register update specified by $\sigma\substcomp\tau$ corresponds to the update given by $\sigma$ followed by the one given by $\tau$. In other words, if we write $\sigma^\dagger \colon (\Gamma^*)^R \to (\Gamma^*)^R$ for the map between register values specified by $\sigma$ -- so in the above example, $\sigma^\dagger(\rho)(X) = \rho(X)\cdot a$ and $\sigma^\dagger(\rho)(Y) = \rho(Y)$ -- then $(\sigma\substcomp\tau)^\dagger = \tau^\dagger \circ \sigma^\dagger$.

To take into account the finite-state control, we consider the notion of \emph{substransition}: a map $\alpha \colon Q \to Q \times (R \to (\Gamma \cup R))^*$ where $Q$ is a set of states. Substransitions also compose (this is an instance of the wreath product construction):
\[ \text{for}\ q\in Q,\quad (\alpha\substcomp\beta)(q) = (q'',\sigma\substcomp\tau)\ \text{where}\ \alpha(q) = (q',\sigma)\ \text{and}\ \beta(q') = (q'',\tau) \]
\begin{definition}
  We write $\SubsTrans(Q,R,\Gamma)$ for the monoid of all substransitions over the set of states $Q$, the set of registers $R$ and the alphabet $\Gamma$, equipped with $\substcomp$.
\end{definition}
We first introduce \emph{copyful} streaming string transducers, which compute a strict superclass of the regular string functions~\cite{CopyfulSST}. Then we introduce the \emph{bounded-copy} restriction, which brings the computational power down to precisely the regular functions. Syntactically, this is a more permissive discipline than the \enquote{copyless} condition mentioned in the introduction; the equivalence between copyless and bounded-copy SSTs was established in\footnote{\tito{Finite coyping~\cite{MacroMSO} = bounded-copy~\cite{AlurFT12} (TODO: check)}
}~\cite[Section~IV.A]{AlurFT12}.

\begin{definition}
  \label{def:sst}
  A (deterministic copyful) \emph{streaming string transducer (SST)} with input
  alphabet $\Sigma$ and output alphabet $\Gamma$ is a tuple $\mathcal{T} = (Q, q_0, R,
  \delta, \vec{u}_I, F)$ where
\begin{itemize}
\item $Q$ is a finite set of \emph{states} and $q_0 \in Q$ is the \emph{initial
    state};
\item $R$ is a finite set of \emph{register names}, that we require to be
  disjoint from $\Gamma$;
\item $\delta \colon \Sigma \to \SubsTrans(Q,R,\Gamma)$ is the \emph{transition function};
\item $\vec{u}_I \in (\Sigma^*)^R$ describes the \emph{initial register values};
\item $F \colon Q \to (\Gamma \cup R)^*$ is the \emph{final output function}.
\end{itemize}
For an input word $w \in \Sigma^*$, the output of $\mathcal{T}$ is determined as follows. Let $(q,\sigma) = \delta^*(w)(q_0)$ where $\delta^*$ is the unique extension of $\delta$ to a monoid homomorphism $\Sigma^* \to \SubsTrans(Q,R,\Gamma)$. Apply $\sigma^\dagger \colon (\Gamma^*)^R \to (\Gamma^*)^R$ to $\vec{u}_I$ to get a tuple $\vec{v} = (v_X)_{X\in R}$ of final register values. Finally, replace each occurrence in $F(q)$ of a register name $X\in R$ by the corresponding value $v_X \in \Gamma^*$ to obtain an output string in $\Gamma^*$.
\end{definition}
\begin{definition}[see e.g.~{\cite{AlurFT12,AperiodicSST}}]
  A streaming string transducer whose transition function is $\delta \colon \Sigma \to \SubsTrans(Q,R,\Gamma)$ is \emph{bounded-copy} when there is some bound $k\in\mathbb{N}$ such that for every $q\in Q$, $w \in \Gamma^*$ and $X,Y\in R$, the string $\sigma(X)$ where $\delta^*(w)(q) = (q',\sigma)$ contains at most $k$ occurrences of~$Y$.
  In this paper, we take computability by bounded-copy SSTs as the definition of \emph{regular string-to-string functions}.
\end{definition}
The SST of \Cref{ex:cecilia} is bounded-copy (see \Cref{ex:cecilia-monoid} below). The following example of streaming string transducer is \emph{not} bounded-copy: it has a single state $q$ and a single register $X$; it performs $X \regassign XX$ for each input letter; the initial value is $X = a$ and the final output function is $F(q) = X$. This SST computes the function is $w \mapsto a\dots(2^{|w|}\ \text{times})\dots a$ which is not a regular function since its growth is not linearly bounded. Observe that the substitution $\sigma$ corresponding to an input factor $w$ is $\sigma \colon X \mapsto X\dots(2^{|w|}\ \text{times})\dots X$ -- this violates the bounded-copy condition.

The notion of bounded-copy SST can be rephrased using the monoid homomorphism $\tterase_\Gamma \colon \SubsTrans(Q,R,\Gamma) \to \SubsTrans(Q,R,\varnothing)$ which \enquote{erases all letters from $\Gamma$}, built by lifting the homomorphism $w \in \Gamma^* \mapsto \varepsilon \in \varnothing^*$ in the unique sensible way. 
\begin{proposition}\label{prop:bounded-copy-finiteness}
  An SST with transition function $\delta \colon \Sigma \to \SubsTrans(Q,R,\Gamma)$ is bounded-copy if and only if $\tterase_\Gamma(\delta^*(\Sigma^*))$ is a \emph{finite} submonoid of $\SubsTrans(Q,R,\varnothing)$. 
\end{proposition}
\begin{remark}
  This monoid $\tterase_\Gamma(\delta^*(\Sigma^*))$ is called the \emph{substitution transition monoid} of the given SST in~\cite[Section~3]{AperiodicSST}.
\end{remark}
\begin{example}\label{ex:cecilia-monoid}
  For the streaming string transducer of \Cref{ex:cecilia} (recall that its input alphabet is $\Gamma\cup\{\#\}$), we have (abbreviating $\ttreverse(w)$ by $\widetilde{w}$):
  \begin{align*}
    \text{for}\ w\in\Gamma^*,\; \delta^*(w) &= \begin{cases}
      0 \mapsto (0, (X \mapsto Xw \mid Y \mapsto Y))\\
      1 \mapsto (1, (X \mapsto \widetilde{w}X \mid Y \mapsto Y))
    \end{cases}\\
    \delta^*(w_0 \# \dots \# w_{2k+1}) &= \begin{cases}
      0 \mapsto (1, (X \mapsto \widetilde{w_{2k+1}} \mid Y \mapsto \widetilde{w_{2k-1}} \dots \widetilde{w_1} YX w_0 \dots w_{2k}))\\
      1 \mapsto (0, (X \mapsto w_{2k+1} \mid Y \mapsto \widetilde{w_{2k}} \dots \widetilde{w_0}XY w_1 \dots w_{2k-1}))
    \end{cases}\\
    \delta^*(w_0 \# \dots \# w_{2k+2}) &= \begin{cases}
      0 \mapsto (0, (X \mapsto w_{2k+2} \mid Y \mapsto \widetilde{w_{2k+1}} \dots \widetilde{w_1}YX w_0 \dots w_{2k}))\\
      1 \mapsto (1, (X \mapsto \widetilde{w_{2k+2}} \mid Y \mapsto \widetilde{w_{2k}} \dots \widetilde{w_0} XY w_1 \dots w_{2k+1}))
    \end{cases}
  \end{align*}
  By erasing the parts in $\Gamma^*$ of the right-hand sides, we see that the substitution transition monoid has three elements: the identity (which is the image of all words without $\#$) and
  \[ \begin{cases}
    0 \mapsto (1, (X \mapsto \varepsilon \mid Y \mapsto YX))\\
    1 \mapsto (0, (X \mapsto \varepsilon \mid Y \mapsto XY))
  \end{cases} \qquad\quad \begin{cases}
    0 \mapsto (0, (X \mapsto \varepsilon \mid Y \mapsto YX))\\
    1 \mapsto (1, (X \mapsto \varepsilon \mid Y \mapsto XY))
  \end{cases} \]
  Hence, since $3 < \infty$, the SST of \Cref{ex:cecilia} is bounded-copy.
\end{example}


\section{Register transducer semigroups}

We now introduce our first definition of regular semigroup-to-semigroup functions, heavily inspired by streaming string transducers. In the definition below, the maps $\mu_{s_1,s_2}$ should be understood as analogous to the substitutions in SSTs. At the same time, the whole definition itself can be seen at first as an abstraction of the monoid denoted by $\delta^*(\Sigma^*)$ in the previous section -- though we will see that even for string-to-string functions, we have good reasons to work with semigroups rather than monoids.

\begin{definition}
    A \emph{finitary semigroup with registers} $\mathcal{S}$ consists of a finite semigroup $S$, a family $(R_s)_{s\in S}$ of finite nonempty sets, and a family of functions
    \[ \mu_{s_1, s_2}\colon R_{s_1 s_2} \to (R_{s_1} + R_{s_2})^+ \quad\text{for}\ s_1, s_2 \in S\]
    such that for all $s_1, s_2, s_3 \in S$, the following associativity diagram
    commutes:
    \[\begin{tikzcd}
        [column sep=2.5cm]
        &
        (R_{s_1 s_2} + R_{s_3})^+
        \ar[r,"(\mu_{s_1,s_2}+\mathrm{id})^+"']
        &
        ((R_{s_1}+R_{s_2})+R_{s_3})^+
        \ar[dd,"\text{assoc.\ of}\ +"]
        \\
        R_{s_1 s_2 s_3}
        \ar[ur,"\mu_{s_1s_2,s_3}"]
        \ar[dr,"\mu_{s_1,s_2s_3}"']
        \\
        &
        (R_{s_1} + R_{s_2 s_3})^+
        \ar[r,"(\mathrm{id}+\mu_{s_2,s_3})^+"]
        &
        (R_{s_1}+(R_{s_2}+R_{s_3}))^+
      \end{tikzcd}\]
    For any semigroup $A$, we define $\mathcal{S}[A] = \displaystyle\bigcup_{s\in S} \{s\} \times A^{R_s}$
    endowed with the binary operation
    \[ (s_1,(a_{1,X})_{X\in R_{s_1}}) \cdot (s_2,(a_{2,Y})_{Y\in R_{s_2}}) = (s_1 s_2,\, (\overbrace{\mathrm{flatten}}^{\mathclap{A^+ \to A\ \text{using the semigroup operation in}\ A}} \circ \underbrace{\mathrm{substitute}}_{\mathclap{\text{replace each}\ X \in R_{s_1}\ \text{(resp.\ $Y \in R_{s_2}$) by}\ a_{1,X}\ \text{(resp.\ $a_{2,Y}$)}}} \circ \mu_{s_1 s_2}(Z))_{Z\in R_{s_1 s_2}}) \]
\end{definition}
\begin{proposition}
  For any finitary semigroup with registers $\mathcal{S}$ and any semigroup $A$, the binary operation on $\mathcal{S}[A]$ is associative: $\mathcal{S}[A]$ is a semigroup.
\end{proposition}
\begin{example}\label{ex:fsr-simple}
  Let $S = \{0,1\}$ with usual multiplication and $R_0 = R_1 = \{X,Y\}$. Let us write $\{X,Y\}+\{X,Y\}=\{X_{\mathrm{l}}, X_{\mathrm{r}}, Y_{\mathrm{l}}, Y_{\mathrm{r}}\}$ with $X_{\mathrm{l}}$ (resp.~$X_{\mathrm{r}}$) referring to the left (resp.~right) copy of $X$. The following definition of $\mu$ satisfies the associativity condition:
  \[ \forall i,j \in \{0,1\}, \qquad \mu_{i,j}(X) = X_{\mathrm{l}} X_{\mathrm{r}} \qquad \mu_{i,0}(Y) = X_{\mathrm{l}} Y_{\mathrm{r}} \qquad \mu_{i,1}(Y) = Y_{\mathrm{l}} Y_{\mathrm{r}} \]
  So $\mathcal{S} = (S,R,\mu)$ is a finitary semigroup with registers. We have for example in $\mathcal{S}[(\mathbb{N},+)]$
  \[ (1,(X\mapsto42\mid Y\mapsto218)) \cdot (0,(X\mapsto1\mid Y\mapsto100)) = (0,(X\mapsto43\mid Y\mapsto142)) \]
\end{example}

\begin{definition}\label{def:reg-trans-semigroup}
    A \emph{register transducer semigroup} consists of a finitary semigroup with registers $\mathcal{S} = (S,R,\mu)$ together with a family of strings $\omega_s \in (R_s)^+$ for $s\in S$.

    A function $f\colon A \to B$ between semigroups is said to be \emph{recognized} by $(\mathcal{S}, \omega)$ if it admits a decomposition $f = \outfun_B \circ h$ where $h \colon A \to \mathcal{S}[B]$ is some semigroup homomorphism and
    \[ \outfun_B(s,(b_X)_{X \in R_s}) = \mathrm{flatten}(\text{replace each}\ X\in R_s\ \text{in}\ \omega_s\ \text{by}\ b_X) \]
    $f$ is called \emph{regular} when there exists a register transducer semigroup that recognizes it.
\end{definition}
\begin{example}\label{ex:rts-simple}
  Let $\mathcal{S}$ be the finitary semigroup with registers from \Cref{ex:fsr-simple}. Reusing the notations from that example, let us choose $\omega_0 = \omega_1 = Y$; this makes $(\mathcal{S},\omega)$ is a register transducer semigroup. It can recognize the function $f\colon \{a,b,c\}^* \to \{a,b\}^*$ such that
  \[ \forall u \in \{a,b,c\}^*,\;\forall v \in \{a,b\}^*,\quad f(ucv) = a^{|u|}bv \quad\text{and}\quad f(v) = v \]
  (inspired by~\cite[Theorem~5.6]{MacroMSO}). Indeed, to do so, we pick the semigroup homomorphism
  \[ h\colon w \in \{a,b,c\}^* \mapsto ((1\ \text{if}\ w\in\{a,b\}^*,\;\text{else}\ 0),\; (X\mapsto a^{|w|} \mid Y \mapsto f(w)) \in \mathcal{S}[\{a,b\}^*]\]
\end{example}

\begin{theorem}
  A function between free monoids is regular in the sense of \Cref{def:reg-trans-semigroup} (i.e.\ recognized by some register transducer semigroup) if and only if it is a regular string function in the conventional sense (i.e.\ computed by some bounded-copy SST).
\end{theorem}
\begin{proof}
  We first treat \enquote{only if}, then \enquote{if}.

  \proofsubparagraph{From a register transducer semigroup to a bounded-copy streaming string transducer.}
  Let $(\mathcal{S}=(S,R,\mu),\,\omega)$ be a register transducer semigroup and $h\colon \Sigma^* \to \mathcal{S}[\Gamma^*]$ be a semigroup homomorphism. We want a bounded-copy SST that computes $\outfun_{\Gamma^*}\circ h$. Our solution will be to take an \enquote{obvious choice} of SST, which clearly computes this function -- the idea is that the configuration (state + register contents) of the SST, after reading an input prefix $w\in\Sigma^*$, represents $h(w)$ -- and then check that it is bounded-copy.
  \begin{itemize}
    \item The set of states is $Q = S$ and the initial state is the first component of $h(\varepsilon)$. (Note that while $\mathcal{S}[\Gamma^*]$ is not necessarily a monoid, $h(\Sigma^*)$ always is, with $h(\varepsilon)$ as its unit element.)
    \item The registers are $R = \displaystyle\bigcup_{s\in S} R_s$ (without loss of generality, we take the $R_s$ pairwise disjoint).
    \item For $c\in\Sigma$ and $q\in Q=S$, we define $\delta(c)(q) = (qs,\sigma)$ where $(s, (v_X)_{X\in R_s}) = h(c)$ and
    \[ \sigma\colon Y \in R ~~\mapsto~~ \begin{cases}
      \text{replace each}\ X \in R_s\ \text{by}\ v_X\ \text{in}\ \mu_{q,s}(Y) & \text{when}\ Y \in R_{qs}\\
      \varepsilon & \text{otherwise}
    \end{cases} \]
    \item The initial register values are $X:=u_{0,X}$ for any $X\in R_{q_0}$ where $(q_0,(u_{0,X})_{X\in R_{q_0}}) = h(\varepsilon)$, and $X := \varepsilon$ for other registers.
    \item The final output function is $q \mapsto \omega_q$.
  \end{itemize}
  One can verify that $\tterase_\Gamma(\delta^*(w))(q) = (qs, [\text{something fully determined by}\ \mu_{q,s}])$ where $s$ is the 1st component of $h(w)$ for any $w\in\Sigma^*$ and $q\in Q$. Since $s$ lives in the finite semigroup~$S$, the monoid $\tterase_\Gamma(\delta^*(\Sigma^*))$ is finite. By \Cref{prop:bounded-copy-finiteness}, our SST is bounded-copy.

  \proofsubparagraph{From a bounded-copy streaming string transducer to a register transducer semigroup.}

  We start by illustrating the main idea on the SST of \Cref{ex:cecilia}. This SST has several convenient properties that simplify the construction (in particular, the semigroup that we get is a monoid): all its registers are initialized with $\varepsilon$, and the final output function merely combines registers without adding letters from the output alphabet. We will see later how to handle the minor complications that may arise without these properties.
  \begin{claim}
    The function $\Sigma^* \to \Gamma^*$ (with $\Sigma = \Gamma\cup\{\#\}$) of \Cref{ex:cecilia} is recognized by a register transducer semigroup $(\mathcal{M} = (M,R,\mu),\; \omega)$, where $M = \tterase_\Gamma(\delta^*(\Sigma^*))$ is the finite substitution transition monoid described in \Cref{ex:cecilia-monoid}, and such that the infinite monoid $\delta^*(\Sigma^*)$ can be identified with a submonoid of $\mathcal{M}[\Gamma^*]$.
  \end{claim}
  \begin{claimproof}[Proof sketch]
    We shall work with the following abbreviations for $u,v,\ldots \in \Sigma^*$:
    \begin{align*}
      \alpha[u,v] &= \begin{cases}
        0 \mapsto (0, (X \mapsto Xu \mid Y \mapsto Y))\\
        1 \mapsto (1, (X \mapsto vX \mid Y \mapsto Y))
      \end{cases}\\
      \text{for}\ i \in \{0,1\},\quad \beta_i[u_1,\dots,v_3] &= \begin{cases}
        0 \mapsto (i,\quad\;\; (X \mapsto u_1 \mid Y \mapsto u_2 YX u_3))\\
        1 \mapsto (1-i, (X \mapsto v_1 \mid Y \mapsto v_2 XY v_3))
      \end{cases}      
    \end{align*}
    We also write $\alpha = \alpha[\varepsilon,\varepsilon]$ and $\beta_i = \beta_i[\varepsilon,\dots,\varepsilon]$.
    Note that $M = \{\alpha,\beta_0,\beta_1\}$ (equipped with composition of substransitions), while every element of $\delta^*(\Sigma^*)$ is of the form $\alpha[\dots]$ or $\beta_i[\dots]$. The set $\mathfrak{M}$ of all $\alpha[\dots]$ and $\beta_i[\dots]$ is a submonoid of $\SubsTrans(\{0,1\},\{X,Y\},\Gamma)$, since $\alpha$ is the unit element and we have composition equations such as
    \[ \alpha[u,v] \cdot \beta_1[u_1,u_2,u_3,v_1,v_2,v_3] = \beta_1[u_1,u_2,(u \cdot u_3),v_1,(v_2 \cdot v),v_3] \]
    The idea is then to define $\mathcal{M} = (M,R,\mu)$ in such a way that $\mathfrak{M} \cong \mathcal{M}[\Gamma^*]$. For the registers, we take $R_\alpha = \{U,V\}$ and $R_{\beta_i} = \{U,V\} \times \{1,2,3\}$. This allows us to define a bijection
    \begin{align*}
      \mathcal{M}[\Gamma^*] &\to \mathfrak{M}\\
      (\alpha, (w_{U}, w_{V})) &\mapsto \alpha[w_{U},w_{V}]\\
      (\beta_i, (w_{U,1},\dots,w_{V,3})) &\mapsto \beta_i[w_{U,1},\dots,w_{V,3}]
    \end{align*}
    Observe also that for $\lambda,\gamma\in\{\alpha,\beta_0,\beta_1\}$, we have $\lambda[\dots] \cdot \gamma[\dots] = (\lambda\cdot\gamma)[\dots]$, which brings us halfway to having the above bijection be a monoid isomorphism. To get there, there remains to choose a definition of $\mu$ reflecting the composition equations; for example the above one concerning $\alpha$ and $\beta_1$ leads us to define $\mu_{\alpha,\beta_1}$ as
    \[ (U,1)\mapsto(U,1) \mid \dots \mid (U,3) \mapsto U \cdot (U,3) \mid \dots \mid (V,2) \mapsto (V,2) \cdot V \mid (V,3) \mapsto (V,3) \]
    We set the output strings of our register transducer semigroup according to the initial state and final output function of \Cref{ex:cecilia}, as follows:
    \begin{itemize}
      \item $\omega_\alpha = U$ because after applying $\alpha[u,v]$ to the initial configuration of the SST, we are in state 0, $X = u$ \& $Y = \varepsilon$, and the final output function tells us to output $YX = u$;
      \item $\omega_{\beta_0} = (U,2)\cdot(U,3)\cdot(U,1)$ because after applying $\beta_0[u_1,\dots,v_3]$ to the initial configuration, we are in state 0 and $X$ contains $u_1$ while $Y$ contains $u_2 u_3$;
      \item $\omega_{\beta_1} = \varepsilon$ because applying $\beta_1[\dots]$ to the initial configuration brings us in state 1, in which the final output function tells us to output $\varepsilon$.
    \end{itemize}
    Finally, to recognize the function of \Cref{ex:cecilia}, we use the semigroup homomorphism $(\text{inverse of above isomorphism}\ \mathcal{M}[\Gamma^*] \to \mathfrak{M}) \circ \delta^* : \Sigma^* \to \mathcal{M}[\Gamma^*]$ -- indeed $\delta^*(\Sigma^*) \subset \mathfrak{M}$.
  \end{claimproof}

  \tito{todo when $f(\varepsilon) \neq \varepsilon$}
\end{proof}



\section{Functorial transducer semigroups and warm-up theorems}

\tito{Exactly like \Cref{sec:warm-up}}

\section{Finiteness-preserving functors define regular functions}


\subsection{From registers to functors}

\tito{todo}

\begin{remark}
  This data induces the finite polynomial functor from semigroups to sets BLAH  
  which can be turned into a functor from semigroups to semigroups thanks to $\mu$.


    Equivalently, consider the following symmetric monoidal category:
    \begin{description}
        \item[Objects:] finite polynomial functors (in one variable).
        \item[Morphisms:] natural transformations.
        \item[Monoidal product:] pointwise cartesian product of sets.
    \end{description}
    A finitary semigroup with registers is the same thing as a \enquote{semigroup object} (defined by removing the unit requirement from monoid objects) in this monoidal category of finite polynomial functors.

    \tito{Blah is equivalent to specifying a natural transformation from the associated polynomial functor to the forgetful functor from semigroups to sets, so it induces a functorial transducer semigroup in the expected way.}
\end{remark}


\begin{remark}
This should also work if we have a single finite nonempty set of registers $R$ for all semigroup elements instead of a family $(R_s)_{s\in S}$ though the equivalence proof is slightly less immediate for semigroups than for monoids (you need to use the output function to produce \enquote{default} elements with which to fill unused registers). Note that the \enquote{substitution transition monoid} operation \enquote{naturally} produces something with a varying register set.

Another possible variation on the definition is to have a choice of output register for each element of the control semigroup: $r_s \in R_s$ for $s\in S$ instead of $\omega_s \in (R_s)^+$. This does not decrease the expressivity of the model.
\end{remark}


\subsection{From functors to registers}

\begin{theorem}
  For every finiteness-preserving functorial transducer semigroup $(\functor, \outfun)$, there is another one $(\functor',\outfun')$
  \begin{itemize}
    \item which is induced by some register transducer semigroup,
    \item and which admits a natural transformation $\eta \colon \functor \Rightarrow \functor'$ such that $\outfun = \outfun' \circ \eta$.
  \end{itemize}
  Hence every function recognized by $(\functor,\outfun)$ is also recognized by $(\functor',\outfun')$ (by the second item) and thus is regular (by the first item).
\end{theorem}
\begin{proof}
    Let $(\functor,\outfun)$ be a functorial transducer semigroup, and suppose $\functor$ preserves finiteness, so $\functor1$ is finite. We first build a finitary semigroup with registers that doesn't entirely fulfill our objectives, but still contains the core mechanism: $S = \functor1$, and for $s\in S$,
    \[ R_s = \sum_{l,r \in F(1)} \{\text{occurrences of}\ \mathtt{m}\ \text{in}\ (\text{factorized output of}\ (l,s,r)) \in 1\oplus1\oplus1 \cong \{\mathtt{l,m,r}\}^+\} \]
    For $s_1,s_2\in S$ we must now define a map $R_{s_1 s_2} \to (R_{s_1} + R_{s_2})^+$ describing the register recombination during a product. We derive it from the correspondence between
    \begin{itemize}
        \item the occurrences of $\mathtt{m}$ in the factorized output of $l \cdot (s_1 s_2)\cdot r$
        \item and the maximal factors in $\{\mathtt{m_1,m_2}\}^+$ inside
        \[ (\text{factorized output of}\ (l,s_1,s_2,r)) \in 1\oplus1\oplus1\oplus1 \cong \{\mathtt{l,m_1,m_2,r}\}^+ \]
    \end{itemize}
    Indeed, the morphism $1\oplus\mathop{!}\oplus1$ where $! \colon 1\oplus1 \to 1$ is the terminal morphism sends the maximal factors in $\{\mathtt{m_1,m_2}\}^+$ to $\mathtt{m}$ and is the identity on $\mathtt{l,r}$. To see that applying this to the factorized output of $(l,s_1,s_2,r)$ indeed yields the factorized output of $(l,s_1s_2,r)$, apply \Cref{lem:merge-middle} to $A=B=C=1$.

    We must also check that this operation is \enquote{associative}. For this, consider the factorized output of $(l,s_1,s_2,s_3,r)$ \ldots{}
    \tito{TODO above}

    Finally we define $\widehat\eta \colon \functor \Rightarrow \functorg$ where $\functorg$ is the functor induced by our semigroup with registers. For any $x \in \functor A$, apply $\functor! \colon \functor A \to \functor 1$ to get some $s \in S$. Then look at the factorized output of $(l,x,r)$ in $1\oplus A \oplus 1$ for each $(l,r)\in (\functor1)^2$ to get register values in $A^{R_s}$.

    The issue is that the output function for $\functor$ may not factor through $\widehat\eta$. To remediate that, we consider instead another semigroup with registers, inducing the functor $\functor'$, with the same underlying semigroup $S=\functor1$ and bigger register sets
    \[ R'_s = \{\bullet\} + R^{\mathrm{left}}_s + R^{\mathrm{right}}_s + R_s \quad\text{for}\ s\in S\]
    where $R^{\mathrm{left}}_s$ is the sum over $r \in \functor1$ of the sets of occurrences of $\mathtt{l}$ in the factorized output of $(s,r)$ which is in $1\oplus1\cong\{\mathtt{l,r}\}^+$, etc. Thus
    \[ \functor'A \cong \sum_{s\in S} A \times A^{R^{\mathrm{left}}_s} \times \dots \cong A \times \sum_{s\in S} A^{R^{\mathrm{left}}_s} \times \dots \]
    and we have then: $\outfun_A = (\text{left projection on}\ A) \circ \eta$ for the definition of $\eta$ extending $\widehat\eta$ in the obvious way.
\end{proof}

\begin{lemma}\label{lem:merge-middle}
    Let $A,B,C$ be semigroups. The following diagram commutes:
    \[\begin{tikzcd}
        [column sep=3cm]
        \functor A \times \functor B\times \functor B \times \functor C
        \ar[r,"\text{factorized output}"]
        \ar[d,"\functor A \times (\text{semigroup operation}) \times \functor C"']
        &
        A \oplus B \oplus B \oplus C
        \ar[d,"A \oplus (\mathrm{id}_B\,\text{or}\,\mathrm{id}_B) \oplus C"]
        \\
        \functor A \times \functor B \times \functor C
        \ar[r,"\text{factorized output}"]
        &
        A \oplus B \oplus C
    \end{tikzcd}\]
\end{lemma}
\tito{similar to \Cref{claim:merge-factorized-output}}
\begin{proof}
    By rotating the above diagram, we see that it corresponds to the outer rectangle in the following diagram (where we have expanded the definition of the factorized output functions, and used the associativity of 4-fold multiplication):
    \[\begin{tikzcd}
        [column sep=4cm]
        \functor A \times \functor B\times \functor B \times \functor C
        \ar[d,"\functor(\text{co-projection})^4"']
        \ar[r,"\functor A \times (\text{semigroup op}) \times \functor C"]
        &
        \functor A \times \functor B \times \functor C
        \ar[d,"\functor(\text{co-projection})^3"]
        \\
        \functor(A \oplus B \oplus B \oplus C)^4
        \ar[d,"\text{multiply middle two together}"']
        &
        \functor(A \oplus B \oplus C)^3
        \ar[dd,""]
        \\
        \functor(A \oplus B \oplus B \oplus C)^3
        \ar[d,"\text{semigroup operation}"']
        \ar[ru,"F(A \oplus (\mathrm{id}_B\,\text{or}\,\mathrm{id}_B) \oplus C)^3"']
        \\
        \functor(A \oplus B \oplus B \oplus C)
        \ar[d, "\outfun_{A \oplus B \oplus B \oplus C}"']
        \ar[r,"F(A \oplus (\mathrm{id}_B\,\text{or}\,\mathrm{id}_B) \oplus C)"]
        &
        \functor(A \oplus B \oplus C)
        \ar[d, ""]
        \\ 
        A \oplus B \oplus B \oplus C
        \ar[r,"A \oplus (\mathrm{id}_B\,\text{or}\,\mathrm{id}_B) \oplus C"]
        &
        A \oplus B \oplus C
    \end{tikzcd}\]
    The lower rectangle commutes by naturality of the output function. The middle trapeze commutes because of the homomorphism property of $F(A \oplus (\mathrm{id}_B\,\text{or}\,\mathrm{id}_B) \oplus C)$. Concerning the upper trapeze, it can be decomposed as the functor $(\cdots \times \cdots \times \cdots)$ applied to three diagrams that we can analyze independently. The first of those is
    \[\begin{tikzcd}
        [column sep=4cm]
        \functor A
        \ar[d,"\functor(\text{co-projection})"']
        \ar[r,"\mathrm{id}_A"]
        &
        \functor A
        \ar[d,"\functor(\text{co-projection})"]
        \\
        \functor(A \oplus B \oplus B \oplus C)
        \ar[d,"\mathrm{id}"']
        &
        \functor(A \oplus B \oplus C)
        \\
        \functor(A \oplus B \oplus B \oplus C)
            \ar[ru,"F(A \oplus (\mathrm{id}_B\,\text{or}\,\mathrm{id}_B) \oplus C)"']
    \end{tikzcd}\]
    By functoriality of $\functor$, the commutativity of this diagram reduces to
    \[ (\mathrm{id}_A \oplus \dots) \circ (\text{co-projection of}\ A\ \text{into}\ A\oplus(B\oplus B\oplus C)) = \text{co-projection of}\ A\ \text{into}\ A\oplus(B\oplus C) \]
    which is an elementary property of the coproduct $\oplus$ in any category (indeed the bifunctor structure of $\oplus$ is built using a coparing, which \enquote{cancels out} with the coprojection).
    Among the 3 diagrams that were combined by a 3-fold cartesian product, another one is identical to the above, with $C$ replacing $A$. There remains this one, in which we have added some parts in the middle:
    \[\begin{tikzcd}
        [column sep=2.5cm]
        \functor B\times \functor B
        \ar[d,"\functor(\text{co-proj})^2"']
        \ar[dr,"\functor(\text{co-proj})^2"]
        \ar[rr,"\text{semigroup operation}"]
        &
        &
        \functor B
        \ar[d,"\functor(\text{co-proj})"]
        \\
        \functor(A \oplus B \oplus B \oplus C)^2
        \ar[d,"\text{semigroup op}"']
        \ar[r,"h\times h"']
        &
        \functor(A \oplus B \oplus C)^2
        \ar[r,"\text{semigroup op}"]
        &
        \functor(A \oplus B \oplus C)
        \\
        \functor(A \oplus B \oplus B \oplus C)
            \ar[rru,"h = F(A \oplus (\mathrm{id}_B\,\text{or}\,\mathrm{id}_B) \oplus C)"']
    \end{tikzcd}\]
    The upper right trapeze commutes because $\functor$ applied to the co-projection of $B$ into $A\oplus B\oplus C$ is a homomorphism. The lower triangle commutes because $h$ is a homomorphism. Finally, the small upper left triangle can be shown to commute using the functoriality of $\functor$ followed by properties of the coproduct.
\end{proof}

\section{Conclusion}

\tito{Extension to forest algebra, etc.}

