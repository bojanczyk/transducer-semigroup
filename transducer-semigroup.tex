\section{Transducer semigroups and the main theorem}
% We use some basic notions from category theory, such as functors or natural transformations. 
We do not assume that the reader has a background in category theory, beyond the two most basic notions: category (objects together with morphisms between them, with composition of morphisms and identity morphisms) and functor (a map from objects $A$ of the source category to objects $\functor A$ of the target category, and from morphisms $f : A \to B$ to $\functor f : \functor A \to \functor B$, preserving composition and identities). In this paper, we will be working mainly with:
\begin{description}
    \item[Sets.] Objects are sets,  morphisms are functions between them.
    \item[Semigroups.] Objects are semigroups,  morphisms are semigroup homomorphisms.
\end{description}

One example of a functor is the \emph{forgetful functor} from the category of semigroups to the category of sets, which maps a semigroup to its underlying set, and a semigroup homomorphism to the corresponding function on sets. 

 \begin{example}\label{ex:functors}
    In this paper, we will mainly be studying functors from the category of semigroups to itself; such functors can be seen as semigroup constructions.  Several such constructions are listed in the following example. 
    \begin{description}
        \item[Tuples.] The functor maps a semigroup $A$ to its square $A^2$, with the semigroup operation being defined coordinate-wise. The functor extends to morphisms in the expected way. This functor also makes sense for higher powers, including infinite powers, such as $A^\omega$.
        \item[Reverse.] The functor maps a semigroup $A$ to the semigroup where the underlying set is the same, but the multiplication is reversed, i.e.~the product of $a$ and $b$ in the new semigroup is the product $ba$ in the old semigroup. Morphisms are not changed by the functor: they retain the homomorphism property despite the change in the multiplication operation.
        \item[Lists.] The functor maps a semigroup $A$ to the free semigroup $A^+$ which consists of non-empty strings over the alphabet $A$  equipped with string concatenation. On morphisms, the functor is defined coordinate-wise. 
        \item[Powerset.] The (covariant) powerset functor maps $A$ to the powerset semigroup $\powerset A$, whose underlying set is the family of all subsets of $A$, endowed with the operation
        \begin{align*}
        (A_1,A_2) \quad \mapsto \quad \left\{a_1 a_2 \mid a_1 \in A_1\ \text{and}\ a_2 \in A_2\right\}
        \end{align*}
        There are several variants of this construction: we could, for example,  require the subsets to be nonempty, or finite, or both.
        % \item Fix some semigroup $C$, and consider the functor which maps a semigroup $A$ to the sub-semigroup of the semigroup $A^C$, as in the first item, which consists only of those tuples $A^C$ that describe semigroup homomorphisms $C \to A$. For morphisms, the functor is defined coordinate-wise, as in the first item.
    \end{description}
 \end{example}

%  \begin{myexample}
%     Here is a non-example of a functor from the category of semigroups to itself. Suppose that, on objects,  we want to map each semigroup $A$ to the set of all functions $A \to A$, with the semigroup operation being function composition. The problem with this construction is that it is not clear how to extend it to morphisms, i.e.~how to map a semigroup homomorphism $f : A \to B$ to some semigroup homomorphism
%     \[
%     \begin{tikzcd}
%     (A \to A)
%     \ar[r,"\functor f"]
%     &
%     (B \to B).
%     \end{tikzcd}
%     \]
%     There are artificial ways to do this. For example, we could choose for each semigroup $B$ some distinguished element $b_0 \in B$, and map a semigroup homomorphism $f : A \to B$ to the semigroup homomorphism which maps all functions $A \to A$ to the constant function $b \mapsto b_0$. 
%  \end{myexample}
 
\noindent
 We now present the central definition of this paper. 

\newcommand{\emptytester}{2}
\begin{definition}
    A \emph{transducer semigroup} is defined to be a functor $\functor$ 
    from the category of semigroups to itself, together with an output mechanism defined as follows. For each semigroup~$A$, there is an output function
    %\begin{align*}
    %\myunderbrace{\outfun_A : \functor A \to A,}{a function between two sets, that \\ is not necessarily a semigroup homomorphism}
    %\end{align*}
    \[ \outfun_A : \functor A \to A \]
    % the left alignment of the lipics style messes up the underbrace positioning
    (between sets, not necessarily a semigroup homomorphism) and this family of functions is natural in the sense that the following diagram commutes for every homomorphism $h$: 
    \[
    \begin{tikzcd}
    \functor A 
    \ar[r,"\functor h"]
    \ar[d,"\outfun_A"']
    &
    \functor B
    \ar[d,"\outfun_B"]
    \\
    A
    \ar[r,"h"']
    &
    B
    \end{tikzcd}
    \]
    A function between two semigroups $f : A \to B$, not necessarily a homomorphism, is  \emph{recognized} by a transducer semigroup if it can be decomposed for some homomorphism $h$ as
    \[
        \begin{tikzcd}
        A 
        \ar[r,"h"]
        &
        \functor B
        \ar[r,"\outfun_B"]
        &
        B
        \end{tikzcd}
        \]
\end{definition}
In the language of category theory, the naturality condition from the above definition says that the output mechanism is a natural transformation of type 
\[\begin{tikzcd}
    [column sep=1cm]
    {\text{Semigroups}} && {\text{Sets}}
    \arrow[""{name=0, anchor=center}, "\text{apply $\functor$ and return underlying set}", curve={height=-18pt}, from=1-1, to=1-3]
    \arrow[""{name=1, anchor=center, inner sep=0}, "\text{return underlying set (forgetful functor)}"', curve={height=18pt}, from=1-1, to=1-3]
    \arrow[ shorten <=5pt, shorten >=5pt, Rightarrow, from=0, to=1]
\end{tikzcd}\]

\begin{example}
    Consider the transducer semigroup in which the functor is the identity, and the output mechanism is also the identity. The functions of type $A \to B$ that are recognized by this transducer semigroup are exactly the semigroup homomorphisms from $A$ to $B$.
\end{example}

\begin{example}\label{ex:duplicator}
    Consider the transducer semigroup in which the functor is the identity, and the output mechanism is the function $a \in a \mapsto aa \in A$.
    %\begin{align*}
    %A & \to A\\
    %a & \mapsto aa.
    %\end{align*}

    The functions of type $A \to B$ that are recognized by this transducer semigroup are exactly those of the form $a \mapsto h(a)h(a)$ where $h$ is some homomorphism. In particular, if $h$ is the identity on $\Sigma^+$, then we get the duplicating function on strings over the alphabet $\Sigma$.
\end{example}



\begin{example}
    Consider the reversing functor from \Cref{ex:functors}. Define the output function to be the identity. Using this transducer semigroup, we can recognize the reversing function $f : \Sigma^+ \to \Sigma^+$. More generally, the functions $f : A \to B$ recognized by this transducer semigroup are \enquote{anti-homomorphisms}, i.e.\ they are those that verify $f(ab) = f(b)f(a)$.
\end{example}

\begin{example}\label{ex:squaring}
    Consider the list functor $A \mapsto A^+$ described in Example~\ref{ex:functors}. The output semigroup is the free semigroup with generators $A$, and the functor is defined coordinate-wise on morphisms. Consider the following output mechanism 
    \begin{align*}
    A^+ & \to A\\
    [a_1,\ldots,a_n] & \mapsto \myunderbrace{(a_1 \cdots a_n) \cdots (a_1 \cdots a_n)}{$n$ times}.
    \end{align*}
    This transducer semigroup recognizes the squaring function $\Sigma^+ \to \Sigma^+$ that is illustrated in the following example 
    \begin{align*}
    123 \mapsto 123123123.
    \end{align*}
\end{example}

\begin{example}\label{ex:squaring-generalized}
    Here is a generalization of the previous example. The functor continues to be $A^+$. The output mechanism $A^+ \to A$ is given by a sequence of strings 
    \begin{align*}
    w_1 \in \set{1}^+, \quad w_2 \in \set{1,2}^+, \quad w_3 \in \set{1,2,3}^+, \quad \ldots.
    \end{align*}
    When applied to lists of  length $n$, the output mechanism is
    \[
        \begin{tikzcd}
            [column  sep=3cm]
        A^n
        \ar[r,"f \mapsto \text{$f$ applied to $w_n$}"]
        &
        A^+ 
        \ar[r,"\text{semigroup operation}"]
        & 
        A.
        \end{tikzcd}
        \]
\end{example}

% \begin{myexample} This example is more challenging than the previous ones, and it is meant to describe copyful \sst.
%     In this example, we use a slightly different setup: we assume that the output mechanism is a partial function, but still natural. For some finite set $R$ of \emph{register names} with a designated \emph{output register}. Define a functor where  
%     \begin{align*}
%     \functor A = R \to (R^+ \oplus A),
%     \end{align*}
%     where $\oplus$ is the co-product of semigroups, and the  semigroup structure of $\functor A$ is defined as follows. The product of two elements $f,g, \in \functor A$ is the composition of the functions described below
%     \[
%     \begin{tikzcd}
%     R
%     \ar[r,"f"]
%     &
%     R^+ \oplus A
%     \ar[r, "g^+ \oplus \id"]
%     &
%     (R^+ \oplus A)^+ \oplus A
%     \ar[r,"\text{mult} \oplus \id"]
%     &
%     R^+ \oplus A \oplus A
%     \ar[r,"\text{merge}"]
%     &
%     R^+ \oplus A.
%     \end{tikzcd}
%     \]
%     Unlike the functors in the previous examples, this functor does outputs an infinite semigroup even if the input semigroup $A$ is finite. 

%     The output mechanism is a partial function, which is obtained by selecting some fixed register. Once we have fixed the initial register, we get an output map as follows: given an element of $\functor A$, we apply it to the initial register, getting an element of $R^+ \oplus A$. If the element is in $A$, then we send it to the output; otherwise the output is undefined.

%     The functions recognized by this transducer semigroup are exactly those that can be recognized by copyful \sst with one state. 
% \end{myexample}



\subsection{Two simple characterizations}
We begin with two simple theorems that describe two classes of string-to-string functions in terms of the transducer semigroups that can be used to recognize them: all functions (Theorem~\ref{thm:all-functions}) and functions that reflect recognizability (Theorem~\ref{thm:reco-reflecting-functions}). These two theorems have simple proofs. In the next Section~\ref{sec:reg-char} we present a third, more  interesting, theorem about  regular functions.

\paragraph*{All functions.} The first theorem shows that every function between two semigroups is recognized by some transducer semigroup.

\begin{theorem}\label{thm:all-functions} 
     Every function    between two semigroups, not necessarily a semigroup homomorphism, is recognized  by some transducer semigroup.
\end{theorem}
\begin{proof}
    Consider some semigroup $A$. Based on this semigroup we will define a transducer semigroup which will recognize all functions from $A$ to some other semigroup. Define  a functor as follows:
\begin{align*}
\functor B = A \times \myunderbrace{(A \to B)}{the set of all  functions, viewed as a semigroup \\ 
with the trivial semigroup operation $xy = x$}.
\end{align*}
The functor is defined on morphisms
as follows: the first coordinate, corresponding to $A$, is not changed, and the second coordinate, corresponding to the set of functions, is transformed   coordinate-wise, when viewed as a tuple indexed by $A$. (This is similar to the tuple construction in Example~\ref{ex:functors}, except that the semigroup structure of $A \to B$ is not defined coordinate-wise.)  This is easily seen to be a functor. 
The output mechanism, which is easily seen to be natural, is function application i.e.
\begin{align*}
    (a,f) \mapsto f(a).
\end{align*}
The transducer semigroup defined above recognizes the function $f : A \to B$. The appropriate homomorphism  is 
\begin{align*}
a \in A  \mapsto  (a,f).
\end{align*} 
\end{proof}

\paragraph*{Recognizability reflecting functions.} Recall that a language 
\[
\begin{tikzcd}
A 
\ar[r,"L"]
&
\set{\text{yes, no}}
\end{tikzcd}
\]
is called \emph{recognizable} if it factors through a homomorphism from $A$ to some \emph{finite} semigroup.  More generally, recognizable maps from the semigroup $A$ to some set, not necessarily with two elements, are those that factor through some homomorphism into a finite semigroup. A function is called \emph{recognizability reflecting} if  
inverse images of recognizable languages are also recognizable. The following example shows that there are a lot of recognizability reflecting functions.

\begin{example}
    Consider the semigroup $\Nat$ of natural numbers with addition. In this semigroup, the recognizable languages are ultimately periodic subsets.  We will show that there are uncountably many recognizability preserving functions of type $\Nat \to \Nat$.
    To see this, consider any  non-decreasing function 
    \begin{align*}
    g : \Nat \to \Nat,
    \end{align*}
    and define $f$ to be the composition of $g$ with the factorial operation
    \begin{align*}
    f : n \mapsto g(n)!
    \end{align*}
    The property of the factorial operation is that every ultimately periodic subset of $\Nat$ contains all or no factorials, up to finitely many exceptions. Therefore, the inverse image of any ultimately periodic set under $f$ will be either finite or co-finite, and therefore also ultimately periodic.
\end{example}


We now present a second characterization, which concerns functions between semigroups that are recognizability reflecting.

\begin{theorem}\label{thm:reco-reflecting-functions}
     The following conditions are equivalent for a function $f : A \to B$, which is not necessarily a semigroup homomorphism:
    \begin{enumerate}
        \item \label{it:reco-refl} $f$ is recognizability reflecting, which means for every recognizable language 
        \[
\begin{tikzcd}
B 
\ar[r,"L"]
&
\set{\text{yes, no}}
\end{tikzcd}
\]
        the language $f;L$ is also recognizable.
        \item \label{it:trans-semig-reco}$f$ is recognized by a transducer semigroup  such that the output mechanism 
        \begin{align*}
        \outfun_B : \functor B \to B
        \end{align*}
        is recognizable for every finite semigroup $B$.
    \end{enumerate}
\end{theorem}
\begin{proof}
    For the implication \ref{it:reco-refl} $\Rightarrow$ \ref{it:trans-semig-reco} we  use a similar construction as in the proof of Theorem~\ref{thm:all-functions}.
    Consider some function $f : A \to B$, which is recognizability reflecting. Define  a functor as follows:
    \begin{align*}
    \functor C = A \times \myunderbrace{\mathrm{Hom}(B,C)}{the set of all  semigroup  homomorphisms, viewed \\ as a semigroup 
    with the trivial semigroup operation $xy = x$}.
    \end{align*}
    On morphisms, the functor is defined as in the proof of Theorem~\ref{thm:all-functions}, and the output mechanism  is function application with $f$ inserted as an interface i.e.
    \begin{align*}
        \outfun : (a,g) \mapsto g(f(a)).
    \end{align*}
    By the assumption that $f$ is recognizability reflecting, the output mechanism $\outfun_C$ is recognizable for finite $C$.
    \tito{not sure this holds when $B$ isn't finitely generated?}
    The transducer semigroup defined above recognizes the function $f$ via the homomorphism 
    \begin{align*}
    a \in A  \mapsto  (a,\id).
    \end{align*} 
    
Consider now the converse implication \ref{it:reco-refl} $\Leftarrow$ \ref{it:trans-semig-reco}. Take a function $f : A \to B$ between two semigroups that satisfies~\ref{it:trans-semig-reco}, i.e.~it is a composition 
\[\begin{tikzcd}
    [column sep=2cm]
	A & {\functor B} & B 
	\arrow["h", from=1-1, to=1-2]
	\arrow["\outfun_{B}", from=1-2, to=1-3]
\end{tikzcd}\]
where $h$ is some homomorphism. We want to show that $f$ is recognizability reflecting. To prove this, consider some recognizable language over the output semigroup, i.e.~a composition of some homomorphism from $B$ into a finite semigroup, followed by an arbitrary boolean-valued function
\[
\begin{tikzcd}
    [column sep=2cm]
B 
\ar[r,"g"']
&
C
\ar[r,"\text{accepting set}"']
&
\set{\text{yes,no}}
\end{tikzcd}
\]
We want to show that is inverse image of the language under $f$ is also recognizable. Consider the following diagram. 
\[\begin{tikzcd}
    [column sep=2cm]
	A & {\functor B} & B \\
	& {\functor C} & C & \set{\text{yes, no}}
	\arrow["h", from=1-1, to=1-2]
	\arrow["\outfun_B", from=1-2, to=1-3]
	\arrow["{\functor g}"', from=1-2, to=2-2]
	\arrow["\outfun_A"', from=2-2, to=2-3]
	\arrow["g", from=1-3, to=2-3]
	\arrow["\text{accepting set}"', from=2-3, to=2-4]
\end{tikzcd}\]
The upper path from $A$ to $\set{\text{yes, no}}$ describes the inverse image under $f$. 
 The middle rectangle commutes by naturality of the output mechanism, and therefore the upper path describes the same function as the lower path. The lower path is a recognizable function, since the first three arrows on it are semigroup homomorphisms.
\end{proof}

% The straightforward construction in the above proof could be extended to characterization functions which reflect other properties of languages, such as being context-free or decidable. 
