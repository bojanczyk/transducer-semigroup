\section{Proof of \Cref{claim:merge-factorized-output}}

The property that we want on factorized outputs reduces immediately to the following lemma on vectorial outputs, by taking $C = 1$ and precomposing everything with $(h,\, h,\, \functor1 \circ h)$.

\begin{lemma}\label{lem:merge-middle}
    Let $A$ and $C$ be semigroups. The following diagram commutes:
    \[\begin{tikzcd}
        [column sep=4cm]
        \functor A\times \functor A \times \functor C
        \ar[d,"\text{vectorial output}"']
        \ar[r,"(\text{semigroup operation}) \times \functor C"]
        &
        \functor A \times \functor C
        \ar[d,"\text{vectorial output}"]
        \\
        A \oplus A \oplus C
        \ar[r,"\text{merge}"]
        &
        A \oplus C
    \end{tikzcd}\]
\end{lemma}
\begin{proof}
    By expanding the definition of the vectorial output functions, and by using the associativity of three-fold multiplication, we see that the diagram in the lemma statement corresponds to the outer rectangle below:
    \[\begin{tikzcd}
        [column sep=4cm]
        \functor A\times \functor A \times \functor C
        \ar[d,"\functor(\text{co-projection})^3"']
        \ar[r,"(\text{semigroup op}) \times \functor C"]
        &
        \functor A \times \functor C
        \ar[d,"\functor(\text{co-projection})^2"]
        \\
        \functor(A \oplus A \oplus C)^3
        \ar[d,"\text{multiply first two together}"']
        &
        \functor(A \oplus C)^2
        \ar[dd,"\text{semigroup op}"]
        \\
        \functor(A \oplus A \oplus C)^2
        \ar[d,"\text{semigroup operation}"']
        \ar[ru,"F(\text{merge})^2"']
        \\
        \functor(A \oplus A \oplus C)
        \ar[d, "\outfun_{A \oplus A \oplus C}"']
        \ar[r,"F(\text{merge})"]
        &
        \functor(A \oplus C)
        \ar[d, "\outfun_{A\oplus C}"]
        \\ 
        A \oplus A \oplus C
        \ar[r,"\text{merge}"]
        &
        A \oplus C
    \end{tikzcd}\]
    The lower rectangle commutes by naturality of the output function. The middle trapeze commutes because merging is a semigroup homomorphism. Concerning the upper trapeze, it can be decomposed as the bifunctor $\times$ applied to two diagrams that we can analyze independently. We start by analyzing the second of those, which is simpler:
    \[\begin{tikzcd}
        [column sep=4cm]
        \functor C
        \ar[d,"\functor(\text{co-projection})"']
        \ar[r,"\mathrm{id}_C"]
        &
        \functor C
        \ar[d,"\functor(\text{co-projection})"]
        \\
        \functor(A \oplus A \oplus C)
        \ar[d,"\mathrm{id}"']
        &
        \functor(A \oplus C)
        \\
        \functor(A \oplus A \oplus C)
            \ar[ru,"F(\text{merge})"']
    \end{tikzcd}\]
    By functoriality of $\functor$, the commutativity of this diagram reduces to
    \[ \text{merge} \circ (\text{co-projection of}\ C\ \text{into}\ B\oplus B\oplus C) = \text{co-projection of}\ C\ \text{into}\ B\oplus C \]
    which is an elementary property of the coproduct $\oplus$ in any category (indeed, the generic categorical definition of merging uses a coparing, which \enquote{cancels out} with the coprojection).
    There remains to check that the first of the two diagrams combined by $\times$ is also commutative; towards this purpose, we have added some parts in the middle:
    \[\begin{tikzcd}
        [column sep=2.5cm]
        \functor A\times \functor A
        \ar[d,"\functor(\text{co-proj})^2"']
        \ar[dr,"\functor(\text{co-proj})^2"]
        \ar[rr,"\text{semigroup operation}"]
        &
        &
        \functor A
        \ar[d,"\functor(\text{co-proj})"]
        \\
        \functor(A \oplus A \oplus C)^2
        \ar[d,"\text{semigroup op}"']
        \ar[r,"h\times h"']
        &
        \functor(A \oplus C)^2
        \ar[r,"\text{semigroup op}"]
        &
        \functor(A \oplus C)
        \\
        \functor(A \oplus A \oplus C)
            \ar[rru,"h = F(\text{merge})"']
    \end{tikzcd}\]
    The upper right trapeze commutes because $\functor$ applied to the co-projection of $a$ into $A\oplus C$ is a homomorphism. The lower triangle commutes because $h$ is a homomorphism. Finally, the small upper left triangle can be shown to commute using the functoriality of $\functor$ followed by properties of the coproduct.
\end{proof}
