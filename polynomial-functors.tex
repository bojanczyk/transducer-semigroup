
\subsection{From a transducer semigroup to a regular function}
\label{sec:hard}
We now turn to the difficult implication $(\ref{it:trans-semig-regular}) \Rightarrow (\ref{it:regular})$ in Theorem~\ref{thm:regular-functions}. 
% The proof is presented in a way which, if sometimes slightly verbose, makes it easier to see how it can be adapted to other algebraic structures instead of semigroups (such as forest algebras).

\subparagraph*{Functorial streaming string transducers.}
\label{sec:abstract-sst} 
The assumption of the implication uses an abstract model (transducer semigroups), while the conclusion uses a concrete operational model (streaming string transducers). To bridge the gap, we use an intermediate model, similar to streaming string transducers, but a bit more abstract. The abstraction arises by using polynomial functors instead of registers, as described below. 

Define a \emph{polynomial functor} to be a semigroup-to-set functor of the form
\begin{align*}
A \quad \mapsto \quad \coprod_{q \in Q} A^{\text{dimension of } q},
\end{align*}
where $Q$ is some possibly infinite set, called the \emph{components}, with each  component having an associated \emph{dimension} in $\set{0,1,\ldots}$. The symbol $\coprod$ stands for disjoint union of sets. This functor does not take into account the semigroup structure of the input semigroup, since the output is seen only as a set.
On morphisms, the functor works in the expected way, i.e.~coordinate-wise.  

A \emph{finite polynomial functor} is one with finitely many components. For example, $A \mapsto A^2 + A^2 + A$ is a finite polynomial functor. 
A finite polynomial functor can be seen as a mild generalization of the construction which maps a semigroup $A$ to the set $A^R$ of register valuations for some fixed set $R$ of register names.  In the generalization, we allow a variable number of registers, depending on some finite information (the component). 

Having defined a more abstract notion of ``register valuations'', namely finite polynomial functors, we now define a more abstract notion of ``register updates''.  The first condition for such updates is that they do not look inside the register contents; this condition is captured by naturality as described in the following definition. 



\begin{definition}[Natural functions]\label{def:natural-functions}
    Let $\functor$  and $\functorg$ be polynomial functors, let $A$ be a semigroup. A function\footnote{This function is not necessarily a semigroup homomorphism. In fact, it would not even make sense call it a homomorphism, since the functors $\functor$ and $\functorg$ produce sets and not semigroups.}  $f : \functor A \to \functorg A$ is called \emph{natural} if it can be extended to natural transformation of type $\functor \Rightarrow \functorg$. This means that there is a family of functions, with one function
    $
    f_A : \functor A \to \functorg A
    $
    for each semigroup $A$, such  that $f=f_A$, and the the  diagram
    \[
    \begin{tikzcd}
    \functor A 
    \ar[r,"\functor h"]
    \ar[d,"f_A"']
    & 
    \functor B 
    \ar[d,"f_B"]
    \\
    \functorg A 
    \ar[r,"h"]
    &
    \functorg B
    \end{tikzcd}
    \]
    commutes for every semigroup homomorphism $h$.
\end{definition}

% Intuitively speaking, naturality says that the function is not allowed to look inside the semigroup elements that are stored in a polynomial functor $\functor A$, but it is allowed to manipulate them using the semigroup operation.

\begin{example}
    Consider the polynomial functors 
    \begin{align*}
    \functor A = A^* = 1 + A^1 + A^2 + \cdots  \qquad \functorg A = A + 1,     \end{align*} 
    where $1$ represents the singleton set $A^0$.
An example of a natural transformation between these two functors is the function which maps a nonempty list in $A^*$ to the product of its elements, and which maps the empty list to the unique element of $1$. A non-example is the function that maps a list $[a_1,\ldots,a_n] \in A^*$ to the leftmost element $a_i$ that is an idempotent in the semigroup, and returns $1$ if such an element does not exist. The reason why the non-example is not natural is that a semigroup homomorphism can map a non-idempotent to an idempotent.
\end{example}
% The naturality condition is an abstraction of the condition (a) that was mentioned before, namely that register contents are not inspected. For example, as we will see below, a natural transformations from the functor $\functor A = A^2$ to itself is $(a,b) \mapsto (abbaa,bbba)$, while 
% \begin{align*}
% (a,b) \mapsto \begin{cases}
%     (1,1) & \text{if the semigroup $A$ has an identity element $1$, and $a=1$}\\
%     (a,b) & \text{otherwise.}
% \end{cases}
% \end{align*}
% is not a natural transformation.


Apart from naturality, we will want our register updates to be copyless. 

\begin{definition}[Copyless natural function] \label{def:copyless} A natural function $f : \functor A \to \functorg A$ is  called \emph{copyless} if it arises from some natural transformation with the following property:  when instantiated to the semigroup\footnote{The choice of the semigroup $(\Nat,+)$ in the \Cref{def:copyless} is not particularly important. For example, the same notion of copylessness would arise if instead of $(\Nat,+)$, we used the semigroup $\set{0,1}$ with addition up to threshold $1$ (i.e.~the only way to get zero is to add two zeros). In the appendix, we present a more syntactic characterization of copyless natural transformations, which will be used later on when proving equivalence with streaming string transducers. 
    } $(\Nat,+)$, the corresponding function of type $\functor \Nat \to \functorg \Nat$ 
    does not increase the norm. Here, the norm of an element in a polynomial functor $\functor \Nat$ or $\functorg \Nat$ is defined to be the sum of numbers that appear in it.
\end{definition}





Having defined functions that are natural and copyless, we now describe the more abstract model of streaming string transducers that is be used in our proof. The main difference is that instead of register valuations and updates given by some finite set of register names, we have two abstract finite polynomial functors, together with an explicitly given application function. Another minor difference is that we allow the model to define partial functions; this will be useful in the proof.


\begin{definition}\label{def:functorial-sst}
    The syntax of a functorial streaming string transducer is given by:
    \begin{itemize}
        \item A finite \emph{input alphabet} $\Sigma$ and an \emph{output semigroup $A$}.
    \item Two finite polynomial functors $\functorr$ and $\functoru$, called the \emph{register} and \emph{update} functors, together with a function of type $\functorr A \times \functoru A \to \functorr A$, called \emph{appliction}, which is natural and copyless.
    \item A distinguished \emph{initial register valuation}  in $\functorr A$.
    \item A \emph{final function} of type $\functorr A \to A + 1$, which is natural and copyless.
    \item An \emph{update oracle}, which is a rational function of type $\Sigma^* \to (\functoru A)^*$.
    \end{itemize}
\end{definition}

The semantics of the transducer is a partial function of type 
$\Sigma^* \to A$ defined as follows. As in Definition~\ref{def:usual-sst}, for every input string we use  the initial register valuation, the application function and the update oracle to define a sequence of register valuations in $\functor A$. We then apply the final function to the last register valuation, yielding a result in $A+1$.  If this result is in the $1$ part, then the output of the transducer is undefined, otherwise the output of the transducer is the semigroup element stored in the $A$ part. We will care about transducers that compute total functions, which corresponds to the property  that for every input string, the last register valuation is in the $A$ part of $A+1$.

\begin{lemma}\label{lem:functorial-sst-complete}
    The models defined in Definitions~\ref{def:usual-sst} and~\ref{def:functorial-sst} define the same (total) string-to-semigroup functions.
\end{lemma}



\subsubsection{Coproducts and views}
\label{sec:coproducts-and-views}

Apart from the more abstract transducer model from Definition~\ref{def:functorial-sst}, the other ingredient used  in the proof of the hard implication in Theorem~\ref{thm:regular-functions} will be \kl{coproducts} of semigroups, and some basic operations on them, as described in this section.

\AP The \intro{coproduct}\footnote{The name \emph{coproduct} is used because of the following universal property: if $f\colon A \to C$ and $g\colon B \to C$ are two semigroup homomorphisms, then there is a unique homomorphism 
      $A \oplus B \to C$
  that coincides with $f$ (resp.~$g$) on the subsemigroup consisting of words with a single letter from $A$ (resp.~$B$).}  of two semigroups $A$ and $B$, denoted by $A \oplus B$, is the semigroup whose elements are nonempty words over an alphabet that is the disjoint union of $A$ and $B$, restricted to words that are \emph{alternating} in the sense that two consecutive letters cannot belong to the same semigroup. The semigroup operation is defined in the expected way. We draw elements of a \kl{coproduct} using coloured boxes, with the following picture showing the product operation in the coproduct of two copies, \red{red} and \blue{blue}, of the semigroup $\set{a,b}^+$:
\begin{align*}
    (\red{\boxed{aba}} \cdot 
    \blue{\boxed{b}} \cdot 
    \red{\boxed{b}} \cdot 
    \blue{\boxed{aa}}) \cdot 
    (
        \blue{\boxed{abba}} \cdot 
        \red{\boxed{aabb}}
    )
    = 
\red{\boxed{aba}} \cdot 
    \blue{\boxed{b}} \cdot 
    \red{\boxed{b}} \cdot 
    \blue{\boxed{aaabba}} \cdot 
        \red{\boxed{aabb}}.
\end{align*}
A \kl{coproduct} can involve more than two semigroups; in the pictures this would correspond to more colours, subject to the condition that  consecutive boxes have different colours.
\begin{remark}
  The \kl{copyless} \kl{register updates} $u : R \to (A + R)^+$ of ordinary \sst{}s that are in \kl{normal form} (cf.~\Cref{sec:easy}) can be seen as maps $R \to A \oplus \displaystyle \bigoplus_{X\in R} \{X\}^+$.
\end{remark}

We write $1$ for the semigroup that has one element. This semigroup is unique up
to isomorphism and it is a \emph{terminal object} in the category of semigroups,
which means that it admits a unique homomorphism from every other semigroup $A$.
This unique homomorphism will be denoted by $!\colon A \to 1$. (It has no
connection with the factorial function on numbers.)

Consider the semigroup-to-set functors defined by (the underlying set of) a
\kl{coproduct} of several copies of their argument with several copies of $1$,
such as $A \mapsto A \oplus A \oplus A \oplus 1 \oplus 1$. In our proof, it will
be useful to see them as \kl{polynomial functors}, even though strictly speaking
they are not defined as sums of products. This identification is allowed by the
following observation (stated for $A\oplus1$ for convenience, but the same idea
applies in general).
\begin{proposition}\label{prop:coproduct-as-polynomial-functor}
  There is a family of bijections, natural in the semigroup $A$, between
  \begin{align*}
    A \oplus 1 \quad \text{and} \quad \coprod_{q \in 1 \oplus 1} A^{\text{\kl{dimension} of $q$}},
  \end{align*}
  where the \kl{dimension} of $q$ is the number of times that the
  first copy of $1$ appears in $q$.
\end{proposition}
\begin{proof}[Idea]
  Given $x\in A\oplus1$, we apply $!\colon A\to1$ to the elements of $A$ in $x$
  to determine the component $q$ of the \kl{polynomial functor} that contains
  the image of $x$ by the left-to-right bijection. This operation, a special
  case of what is called the \kl{shape} below, forgets those elements of $A$
  appearing in $x$, so we record them in a tuple living in $A^{\dim(q)}$. For
  example, $\red{\boxed{aba}} \cdot {\boxed{1}} \cdot \red{\boxed{aa}} \cdot
  \boxed{1}$ is sent to the tuple $(aba,aa)$ in the component $\red{\boxed{1}}
  \cdot {\boxed{1}} \cdot \red{\boxed{1}} \cdot \boxed{1}$.
\end{proof}

The crucial property of semigroups that will be used in our proof is \Cref{lem:views} below, which says that an element of a \kl{coproduct} can be reconstructed based on certain partial information. This information is described  using the following operations.

\begin{enumerate}
    \item \AP \textbf{Merging}. Consider a coproduct $A_1 \oplus \cdots \oplus A_n$, such that the same semigroup $A$ appears on all coordinates from a subset $I \subseteq \set{1,\ldots,n}$, and possibly on other coordinates as well. Define \intro{merging the parts from $I$} to be the function of type 
    \begin{align*}
        A_1 \oplus \cdots \oplus A_n \to  A \oplus \bigoplus_{i \not \in I} A_i
        \end{align*}
    that is defined in the expected way, and explained in the following picture. In the picture, \kl{merging} is applied to a coproduct of three copies of the semigroup $\set{a,b}^+$, indicated using colours \red{red}, black and \blue{blue}, and the merged coordinates are \red{red} and \blue{blue}:
        \begin{align*}
        \blue{\boxed{aba}} \cdot 
        \red{\boxed{b}} \cdot 
        \boxed{aa} \cdot 
        \red{\boxed{b}} \cdot 
        \blue{\boxed{aa}} \cdot 
        \red{\boxed{abba}} \cdot 
        \boxed{b}
        \quad \mapsto \quad  
        \myunderbrace{
            \violet{\boxed{abab}} \cdot 
        \boxed{aa} \cdot 
        \violet{\boxed{baaabba}} \cdot 
        \boxed{b}
        }{the merge of \red{red} and \blue{blue} is drawn in \violet{violet}}.\end{align*}    
        \item \AP \textbf {Shape.}  Define the \intro{shape operation} to be the function of type 
        \begin{align*}
        A_1 \oplus \cdots \oplus A_n \to 1 \oplus \cdots \oplus 1
        \end{align*}
        obtained by applying $!$ on every coordinate. The \kl{shape} says how many alternating blocks there are, and which semigroups they come from, as explained in the following picture:
        \begin{align*}
            \blue{\boxed{aba}} \cdot 
            \red{\boxed{b}} \cdot 
            \boxed{aa} \cdot 
            \red{\boxed{b}} \cdot 
            \blue{\boxed{aa}} \cdot 
            \red{\boxed{abba}} \cdot 
            \boxed{b}
            \quad \mapsto \quad  
            \blue{\boxed{1}} \cdot 
            \red{\boxed{1}} \cdot 
            \boxed{1} \cdot 
            \red{\boxed{1}} \cdot 
            \blue{\boxed{1}} \cdot 
            \red{\boxed{1}} \cdot 
            \boxed{1}.
        \end{align*}
        \item \AP \textbf{Views.} The final operation is the $i$-th \intro{view}
        \begin{align*}
        A_1 \oplus \cdots \oplus A_n \to 1 \oplus A_i.
        \end{align*}
        This operation applies $!$ to all coordinates other than $i$, and then it \kl{merges} all those coordinates. Here is a picture, in which we take the \kl{view} of the \blue{blue} coordinate:
        \begin{align*}
            \blue{\boxed{aba}} \cdot 
            \red{\boxed{b}} \cdot 
            \boxed{aa} \cdot 
            \red{\boxed{b}} \cdot 
            \blue{\boxed{aa}} \cdot 
            \red{\boxed{abba}} \cdot 
            \boxed{b}
            \quad \mapsto \quad  
            \blue{\boxed{aba}} \cdot 
            {\boxed{1}} \cdot 
            \blue{\boxed{aa}} \cdot 
            \boxed{1}.
        \end{align*}
\end{enumerate}
The key observation is that an element of a \kl{coproduct} is fully determined
from its \kl{shape} and \kl{views}, as stated in the following lemma.
It seems to contain the essential property of semigroups that makes the
construction work. We expect our theorem to also be true for other algebraic
structures for which the lemma is true; however, the lemma seems to fail in
certain settings. Concrete examples will be discussed in the conclusion (\Cref{sec:conclusion}).
\begin{lemma}
\label{lem:views} Let $A_1,\ldots,A_n$ be semigroups. The \intro{deconstruction} function of type
\begin{align*}
A_1 \oplus \cdots \oplus A_n \longrightarrow (1 \oplus A_1) \times \cdots \times (1 \oplus A_n) \times (1 \oplus \cdots \oplus 1),
\end{align*}
which is obtained by combining the \kl{views} for all $i \in \set{1,\ldots,n}$ and the \kl{shape}, is injective. 
\end{lemma}
\AP We prove this by exhibiting an explicit partial left inverse: a
\intro{reconstruction} function of type
\[ \qquad (1 \oplus A_1) \times \cdots \times (1 \oplus A_n) \times (1 \oplus
  \cdots \oplus 1) \longrightarrow (A_1 \oplus \cdots \oplus A_n) + 1 \]
deconstruction followed by reconstruction maps every element of $A_1
\oplus \cdots \oplus A_n$ to itself. The idea is to start with the \kl{shape}
and replace the entries from $1$ with the elements appearing in the
\kl{views} in the right order. Rather than a formal definition, we illustrate
this on an example (in the three views, we omit the boxes around the 1s to avoid
visual cluttering):
\[\arraycolsep=1.4pt\def\arraystretch{1.2}
  \begin{array}{ccccccc}
    \blue{\boxed{aba}}
    & & 1 & &
    \blue{\boxed{aa}} & \multicolumn{2}{c}{1} \\
    1 &
    \red{\boxed{b}} &  
    1 & 
    \red{\boxed{b}} & 
    1 & 
      \red{\boxed{abba}} & 1\\
     \multicolumn{2}{c}{1} & 
    \boxed{aa} & \multicolumn{3}{c}{1} & 
    \boxed{b}\\
    \blue{\boxed{1}} &
    \red{\boxed{1}} &
    \boxed{1} &
    \red{\boxed{1}} &
    \blue{\boxed{1}} &
    \red{\boxed{1}} &
    \boxed{1}
  \end{array}
  \qquad\mapsto\qquad \blue{\boxed{aba}} \cdot 
  \red{\boxed{b}} \cdot 
  \boxed{aa} \cdot 
  \red{\boxed{b}} \cdot 
  \blue{\boxed{aa}} \cdot 
  \red{\boxed{abba}} \cdot 
  \boxed{b}
\]
Besides proving \Cref{lem:views}, this reconstruction function also enjoys the
following property, which can be seen from the definition and
\Cref{prop:coproduct-as-polynomial-functor}.
\begin{proposition}\label{prop:reconstruction}
  When each $A_i$ is either $A$ or $1$, \kl{reconstruction} can be seen as a \kl{copyless natural function} between \kl{polynomial functors} in $A$.
\end{proposition}
