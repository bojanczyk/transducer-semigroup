
\subsection{From a transducer semigroup to a regular function}
\label{sec:hard}
We now turn to the difficult implication $(\ref{it:trans-semig-regular}) \Rightarrow (\ref{it:regular})$ in Theorem~\ref{thm:regular-functions}. 
% The proof is presented in a way which, if sometimes slightly verbose, makes it easier to see how it can be adapted to other algebraic structures instead of semigroups (such as forest algebras).

\subparagraph*{Functorial streaming string transducers.}
\label{sec:abstract-sst} 
The assumption of the implication uses an abstract model (transducer semigroups), while the conclusion uses a concrete operational model (streaming string transducers). To bridge the gap, we use an intermediate model, similar to streaming string transducers, but a bit more abstract. The abstraction arises by using polynomial functors instead of registers, as described below. 

Define a \emph{polynomial functor} to be a semigroup-to-set functor of the form
\begin{align*}
A \quad \mapsto \quad \coprod_{q \in Q} A^{\text{dimension of } q},
\end{align*}
where $Q$ is some possibly infinite set, called the \emph{components}, with each  component having an associated \emph{dimension} in $\set{0,1,\ldots}$. The symbol $\coprod$ stands for disjoint union of sets. This functor does not take into account the semigroup structure of the input semigroup, since the output is seen only as a set.
On morphisms, the functor works in the expected way, i.e.~coordinate-wise.  

A \emph{finite polynomial functor} is one with finitely many components. For example, $A \mapsto A^2 + A^2 + A$ is a finite polynomial functor. 
A finite polynomial functor can be seen as a mild generalization of the construction which maps a semigroup $A$ to the set $A^R$ of register valuations for some fixed set $R$ of register names.  In the generalization, we allow a variable number of registers, depending on some finite information (the component). 

Having defined a more abstract notion of ``register valuations'', namely finite polynomial functors, we now define a more abstract notion of ``register updates''.  The first condition for such updates is that they do not look inside the register contents; this condition is captured by naturality as described in the following definition. 



\begin{definition}[Natural functions]\label{def:natural-functions}
    Let $\functor$  and $\functorg$ be polynomial functors, let $A$ be a semigroup. A function\footnote{This function is not necessarily a semigroup homomorphism. In fact, it would not even make sense call it a homomorphism, since the functors $\functor$ and $\functorg$ produce sets and not semigroups.}  $f : \functor A \to \functorg A$ is called \emph{natural} if it can be extended to natural transformation of type $\functor \Rightarrow \functorg$. This means that there is a family of functions, with one function
    $
    f_A : \functor A \to \functorg A
    $
    for each semigroup $A$, such  that $f=f_A$, and the the  diagram
    \[
    \begin{tikzcd}
    \functor A 
    \ar[r,"\functor h"]
    \ar[d,"f_A"']
    & 
    \functor B 
    \ar[d,"f_B"]
    \\
    \functorg A 
    \ar[r,"h"]
    &
    \functorg B
    \end{tikzcd}
    \]
    commutes for every semigroup homomorphism $h$.
\end{definition}

% Intuitively speaking, naturality says that the function is not allowed to look inside the semigroup elements that are stored in a polynomial functor $\functor A$, but it is allowed to manipulate them using the semigroup operation.

\begin{example}
    Consider the polynomial functors 
    \begin{align*}
    \functor A = A^* = 1 + A^1 + A^2 + \cdots  \qquad \functorg A = A + 1,     \end{align*} 
    where $1$ represents the singleton set $A^0$.
An example of a natural transformation between these two functors is the function which maps a nonempty list in $A^*$ to the product of its elements, and which maps the empty list to the unique element of $1$. A non-example is the function that maps a list $[a_1,\ldots,a_n] \in A^*$ to the leftmost element $a_i$ that is an idempotent in the semigroup, and returns $1$ if such an element does not exist. The reason why the non-example is not natural is that a semigroup homomorphism can map a non-idempotent to an idempotent.
\end{example}
% The naturality condition is an abstraction of the condition (a) that was mentioned before, namely that register contents are not inspected. For example, as we will see below, a natural transformations from the functor $\functor A = A^2$ to itself is $(a,b) \mapsto (abbaa,bbba)$, while 
% \begin{align*}
% (a,b) \mapsto \begin{cases}
%     (1,1) & \text{if the semigroup $A$ has an identity element $1$, and $a=1$}\\
%     (a,b) & \text{otherwise.}
% \end{cases}
% \end{align*}
% is not a natural transformation.


Apart from naturality, we will want our register updates to be copyless. 

\begin{definition}[Copyless natural function] \label{def:copyless} A natural function $f : \functor A \to \functorg A$ is  called \emph{copyless} if it arises from some natural transformation with the following property:  when instantiated to the semigroup\footnote{The choice of the semigroup $(\Nat,+)$ in the \Cref{def:copyless} is not particularly important. For example, the same notion of copylessness would arise if instead of $(\Nat,+)$, we used the semigroup $\set{0,1}$ with addition up to threshold $1$ (i.e.~the only way to get zero is to add two zeros). In the appendix, we present a more syntactic characterization of copyless natural transformations, which will be used later on when proving equivalence with streaming string transducers. 
    } $(\Nat,+)$, the corresponding function of type $\functor \Nat \to \functorg \Nat$ 
    does not increase the norm. Here, the norm of an element in a polynomial functor $\functor \Nat$ or $\functorg \Nat$ is defined to be the sum of numbers that appear in it.
\end{definition}





Having defined functions that are natural and copyless, we now describe the more abstract model of streaming string transducers that is be used in our proof. The main difference is that instead of register valuations and updates given by some finite set of register names, we have two abstract finite polynomial functors, together with an explicitly given application function. Another minor difference is that we allow the model to define partial functions; this will be useful in the proof.


\begin{definition}\label{def:functorial-sst}
    The syntax of a functorial streaming string transducer is given by:
    \begin{itemize}
        \item A finite \emph{input alphabet} $\Sigma$ and an \emph{output semigroup $A$}.
    \item Two finite polynomial functors $\functorr$ and $\functoru$, called the \emph{register} and \emph{update} functors, together with a function of type $\functorr A \times \functoru A \to \functorr A$, called \emph{appliction}, which is natural and copyless.
    \item A distinguished \emph{initial register valuation}  in $\functorr A$.
    \item A \emph{final function} of type $\functorr A \to A + 1$, which is natural and copyless.
    \item An \emph{update oracle}, which is a rational function of type $\Sigma^* \to (\functoru A)^*$.
    \end{itemize}
\end{definition}

The semantics of the transducer is a partial function of type 
$\Sigma^* \to A$ defined as follows. As in Definition~\ref{def:usual-sst}, for every input string we use  the initial register valuation, the application function and the update oracle to define a sequence of register valuations in $\functor A$. We then apply the final function to the last register valuation, yielding a result in $A+1$.  If this result is in the $1$ part, then the output of the transducer is undefined, otherwise the output of the transducer is the semigroup element stored in the $A$ part. We will care about transducers that compute total functions, which corresponds to the property  that for every input string, the last register valuation is in the $A$ part of $A+1$.

\begin{lemma}\label{lem:functorial-sst-complete}
    The models defined in Definitions~\ref{def:usual-sst} and~\ref{def:functorial-sst} define the same (total) string-to-semigroup functions.
\end{lemma}



\subparagraph*{Coproducts and views.}
\label{sec:coproducts-and-views}

Apart from the more abstract transducer model from Definition~\ref{def:functorial-sst}, the other ingredient used  in the proof of the hard implication in Theorem~\ref{thm:regular-functions} will be coproducts of semigroups, and some basic operations on them, as described in this section.

We write $1$ for the semigroup that has one element. This semigroup is unique up to isomorphism and it is a \emph{terminal object} in the category of semigroups, which means that it admits a unique homomorphism from every other semigroup $A$. This unique homomorphism will be denoted by $! : A \to 1$. It has no connection with the factorial function on numbers. 

The \emph{coproduct}  of two semigroups $A$ and $B$, denoted by $A \oplus B$, is the semigroup whose elements are nonempty words over an alphabet that is the disjoint union of $A$ and $B$, restricted to words that are \emph{alternating} in the sense that two consecutive letters cannot belong to the same semigroup. The semigroup operation is defined in the expected way. We draw elements of a coproduct using coloured boxes, with the following picture showing the product operation in the coproduct of two copies, \red{red} and \blue{blue}, of the semigroup $\set{a,b}^+$:
\begin{align*}
    (\red{\boxed{aba}} \cdot 
    \blue{\boxed{b}} \cdot 
    \red{\boxed{b}} \cdot 
    \blue{\boxed{aa}}) \cdot 
    (
        \blue{\boxed{abba}} \cdot 
        \red{\boxed{aa}} \cdot 
        \red{\boxed{bb}}
    )
    = 
\red{\boxed{aba}} \cdot 
    \blue{\boxed{b}} \cdot 
    \red{\boxed{b}} \cdot 
    \blue{\boxed{aaabba}} \cdot 
        \red{\boxed{aa}} \cdot 
        \red{\boxed{bb}}.
\end{align*}
A coproduct can involve more than two semigroups; in the pictures this would correspond to more colours, subject to the condition that  consecutive boxes have different colours.

% The name \emph{coproduct} is used because of the following universal property: if
% \begin{align*}
% f : A \to C \qquad \text{and} \qquad g : B \to C
% \end{align*}
% are two semigroup homomorphisms, then
% there is a unique homomorphism 
% \[
% \begin{tikzcd}
% f \text{ or } g : A \oplus B \to C
% \end{tikzcd}
% \]
% that coincides with $f$ (resp.~$g$) on the subsemigroup of $A \oplus B$ consisting of words with a single letter from $A$ (resp.~$B$).






% \begin{example}\label{ex:coproduct-as-polynomial-functor}
    The polynomial functors that we  use in our proof will arise using coproducts with the singleton semigroup $1$. Consider  the semigroup-to-set functor $A \mapsto A \oplus 1$, which maps a semigroup to the underlying set of its coproduct with the singleton semigroup. Although not defined as a polynomial functor, this functor is isomorphic to a polynomial functor. This is because for every semigroup $A$ there is a bijective correspondence between the sets
    \begin{align}\label{eq:polynomial-representation-of-coproduct}
    A \oplus 1 \quad \text{and} \quad \coprod_{q \in 1 \oplus 1} A^{\text{dimension of $q$}},
    \end{align}
    where the dimension of $q$ is defined to be the number of times that the first copy of $1$ appears in $q$. Furthermore, this bijection is natural, and thus we can speak of $A \oplus 1$ as being a polynomial functor. This remark applies to similar constructions, which involve a coproduct of several copies of $A$ with several copies of $1$, such as $A \oplus A \oplus A \oplus 1 \oplus 1$.
    
    %Mikolaj: I eliminated parts of this discussion, since they can be inserted (in the reader's head) at the places where they are used
    % bijection in~\eqref{eq:polynomial-representation-of-coproduct} is natural, and therefore there is a natural bijection between the functor $(-) \oplus 1$ and some polynomial functor. Also, if two polynomial functors are connected by a natural bijection, then they are the same, up to renaming of the components, and therefore the representation in~\eqref{eq:polynomial-representation-of-coproduct} is unique up to renaming of components. By uniqueness, we will simply speak of $A \oplus 1$ as being a polynomial functor.  In a similar way, functors such as $A \mapsto A \oplus 1 \oplus A$ are also polynomial.
    % Noting that $\oplus$ makes sense as an operation on sets without a semigroup structure (it builds a set of nonempty alternating sequences), we have more generally that, if $\functor$ is a polynomial functor, then so is $A \mapsto \functor A \oplus 1$. 
% \end{example}







% \begin{example}\label{ex:copyless-on-coproducts}
%     Consider the functor \enquote{$A\mapsto$ underlying set of $A \oplus 1$} from Example~\ref{ex:coproduct-as-polynomial-functor}, which was shown to be a polynomial functor. The natural transformation in $A$
%     \begin{align*}
%     (A \oplus 1)\times (A \oplus 1) \longrightarrow A \oplus 1
%     \end{align*}
%     which describes the semigroup operation in the coproduct $A \oplus 1$ is copyless.
% \end{example}


% \subsubsection{Views}
% \label{sec:views}

The crucial property of semigroups that will be used in our proof is described in Lemma~\ref{lem:views} below, which says that a coproduct can be reconstructed based on certain partial information. This partial information is described  using the following operations on coproducts.

\begin{enumerate}
    \item \textbf{Merging}. Consider a coproduct $A_1 \oplus \cdots \oplus A_n$, such that the same semigroup $A$ appears on all coordinates from a subset $I \subseteq \set{1,\ldots,n}$, and possibly on other coordinates as well. Define \emph{merging the parts from $I$} to be the function of type 
    \begin{align*}
        A_1 \oplus \cdots \oplus A_n \to  A \oplus \bigoplus_{i \not \in I} A_i
        \end{align*}
    that is defined in the expected way, and explained in the following picture. In the picture, merging is applied to  a coproduct of three copies of the semigroup $\set{a,b}^+$, indicated using colours \red{red}, black and \blue{blue}, and the merged coordinates are \red{red} and \blue{blue}:
        \begin{align*}
        \blue{\boxed{aba}} \cdot 
        \red{\boxed{b}} \cdot 
        \boxed{aa} \cdot 
        \red{\boxed{b}} \cdot 
        \blue{\boxed{aa}} \cdot 
        \red{\boxed{abba}} \cdot 
        \boxed{b}
        \quad \mapsto \quad  
        \myunderbrace{
            \violet{\boxed{abab}} \cdot 
        \boxed{aa} \cdot 
        \violet{\boxed{baaabba}} \cdot 
        \boxed{b}
        }{the merge of \red{red} and \blue{blue} is drawn in \violet{violet}}.\end{align*}    
        \item \textbf {Shape.}  Define the \emph{shape operation} to be the function of type 
        \begin{align*}
        A_1 \oplus \cdots \oplus A_n \to 1 \oplus \cdots \oplus 1
        \end{align*}
        obtained by applying $!$ on every coordinate. The shape says how many alternating blocks there are, and which semigroups they come from, as explained in the following picture:
        \begin{align*}
            \blue{\boxed{aba}} \cdot 
            \red{\boxed{b}} \cdot 
            \boxed{aa} \cdot 
            \red{\boxed{b}} \cdot 
            \blue{\boxed{aa}} \cdot 
            \red{\boxed{abba}} \cdot 
            \boxed{b}
            \quad \mapsto \quad  
            \blue{\boxed{1}} \cdot 
            \red{\boxed{1}} \cdot 
            \boxed{1} \cdot 
            \red{\boxed{1}} \cdot 
            \blue{\boxed{1}} \cdot 
            \red{\boxed{1}} \cdot 
            \boxed{1}.
        \end{align*}
        \item \textbf{Views.} The final operation is the $i$-th view 
        \begin{align*}
        A_1 \oplus \cdots \oplus A_n \to 1 \oplus A_i.
        \end{align*}
        This operation applies $!$ to all coordinates other than $i$, and then it merges all those coordinates. Here is a picture, in which we take the view of the \blue{blue} coordinate:
        \begin{align*}
            \blue{\boxed{aba}} \cdot 
            \red{\boxed{b}} \cdot 
            \boxed{aa} \cdot 
            \red{\boxed{b}} \cdot 
            \blue{\boxed{aa}} \cdot 
            \red{\boxed{abba}} \cdot 
            \boxed{b}
            \quad \mapsto \quad  
            \blue{\boxed{aba}} \cdot 
            {\boxed{1}} \cdot 
            \blue{\boxed{aa}} \cdot 
            \boxed{1}.
        \end{align*}
        
\end{enumerate}


% Another crucial property of the coproduct of semigroups is that a coproduct can be reconstructed from certain partial information, as described below. 
% Consider a coproduct $A_1 \cdots A_n$.


The key observation is that an element of a coproduct can be reconstructed from its shape and views, as stated in the following lemma. 

\begin{lemma}
\label{lem:views} Let $A_1,\ldots,A_n$ be semigroups. The function of type
\begin{align*}
A_1 \oplus \cdots \oplus A_n \to (1 \oplus A_1) \times \cdots \times (1 \oplus A_n) \times (1 \oplus \cdots \oplus 1),
\end{align*}
which is obtained by combining the views for all $i \in \set{1,\ldots,n}$ and the shape, is injective. 
\end{lemma}
\begin{proof}
    The input can be reconstructed from the output as follows.
    Start with the shape, and replace the entries from $1$ with the semigroup elements used in the views.
\end{proof}


This lemma seems to contain the essential property of semigroups that makes the construction work. Our theorem will also be true for other algebraic structures for which the lemma is true, such as forest algebras. However, the lemma seems to fail for certain algebraic structures, such as groups, even if we allow $1$ to be replaced by some fixed finite group. Another example where the lemma seems to fail is the monad of weighted sums of words (i.e.~this monad corresponds to weighted automata).

