

\section{The regular functions}
\label{sec:reg-char}
The two constructions in Theorems~\ref{thm:all-functions} and~\ref{thm:reco-reflecting-functions} were rather straightforward, and amounted to little more than symbol pushing. In this section, we present a more advanced  characterization, which concerns  the regular functions, i.e.~string-to-string functions recognized by streaming string transducers and their equivalent models. In the characterization, we require that the functor is finiteness preserving, i.e.~it maps finite semigroups to finite semigroups. It turns out that the naturality of the output mechanism interacts with the condition that 
the functor is finiteness preserving, resulting in a strong restriction on the expressive power.

\begin{myexample}
    Consider the powerset functor $\powerset A$ from Example~\ref{ex:functors}. This is a finiteness preserving functor, since the powerset of a finite set is also finite. One could imagine that using the powerset functor we could construct some transducer semigroup which recognizes functions that are not regular, e.g.~because they have exponential growth. It turns out that this is impossible, because there is no possible output mechanism, i.e.~no natural transformation 
    \[
    \begin{tikzcd}
    \powerset A 
    \ar[r,"\outfun_A"]
    &
    A.
    \end{tikzcd}
    \]
    There reason why there is no such natural transformation is that it require some kind of choice, which would contradict naturality. More formally, let $A$ be a semigroup with two elements, with a trivial semigroup operation defined by $xy=x$. The output mechanism needs to choose some element $a \in A$ to the full set $A \in \powerset A$. However, none of the two choices is right, because if we take any semigroup homomorphism $f : A \to A$ such that $f(A)=A$, then  naturality of the output mechanism implies that $a=f(a)$. If $f$ is the homomorphism that swaps the two elements, then we get a contradiction.
\end{myexample}


We now state the main theorem of this paper. Unlike the previous characterizations, the statement is stated in terms of functions between free semigroups $\Sigma^+$ and $\Gamma^+$, because the models defining regular functions are defined for string-to-string functions, and not transformations of abstract semigroups. (Some of the models, such as streaming string transducers or two-way automata, easily make sense when the output is an abstract semigroup, but the string structure of the input semigroup seems to be essential for all the models.) 

% Among the functors described in Example~\ref{ex:functors},  ``reverse'' and ``powerset''  are finiteness preserving, in the sense that if they are applied to a finite semigroup, then the result is also a finite semigroup. The ``tuple'' functor $A^n$ is finiteness preserving if and only if the exponent $n$ is finite. The ``list'' functor $A^+$ is not finiteness preserving. 


\begin{theorem}\label{thm:regular-functions}
    Let  $\Sigma$ and $\Gamma$  be finite alphabets. The following conditions are equivalent for a function $f : \Sigma^+ \to \Gamma^+$, which is not necessarily a semigroup homomorphism:
    \begin{enumerate}
        \item \label{it:regular} $f$ is  a regular function, as defined in  Section~\ref{sec:easy}. 
        \item \label{it:trans-semig-regular}$f$ is recognized by a transducer semigroup $(\functor,\outfun)$ such that for every finite semigroup $A$, the semigroup $\functor A$ is also finite. 
    \end{enumerate}
\end{theorem}


Before proceeding with the proof, we comment on the role of empty strings. Regular functions are usually defined for possibly empty strings, i.e.~functions of type $\Sigma^* \to \Gamma^*$. We use nonempty strings, because it will be more convenient to work with semigroups, and the free semigroup construction produces nonempty strings. To extend the construction to functions that can output possibly empty strings, while still working with semigroups, we could modify the type of the output mechanism to be 
\begin{align*}
\outfun_A : \functor A \to \myunderbrace{A + 1}{disjoint union of underlying set of $A$ \\ with one extra element representing the empty word},
\end{align*}
under this modification the same proof as presented below would give us exactly the regular functions with possibly empty outputs. The empty string as an input is less important, since we can always  extend  the source code of a transducer that inputs nonempty strings with an extra line which says how to handle an empty input string. 


The left-to-right implication is relatively straightforward, and presented in Section~\ref{sec:easy}, together with the definition of streaming string transducers. The main part of the proof  is devoted to the right-to-left implication. The proof is presented in a way which, if sometimes slightly verbose, makes it easier to see how it can be adapted to other algebraic structures instead of semigroups (such as forest algebras).

\subsection{Defininition of streaming string transducers}
\label{sec:sst-definition}
In this section, we formally describe the \kl{regular functions}, using a model based on \kl{streaming string transducers} (\sst).  This model, like our proof of Theorem~\ref{thm:regular-functions}, covers a slightly more general case, namely string-to-semigroup functions instead of only string-to-string functions. These are functions of type $\Sigma^* \to A$ where $\Sigma$ is a finite alphabet and $A$ is an arbitrary semigroup.  The purpose of this generalization is to make notation more transparent, since the fact that the output semigroup consists of strings will not play any role in our proof.

% The model is a minor variation on streaming string transducers, which use registers to store elements of the output semigroup.
\AP The model uses \intro{registers} to store elements of the output semigroup. We begin by describing notation for registers and their updates. Suppose that $R$ is a finite set of \kl{register names}, and $A$ is a semigroup called the \emph{output semigroup}. We consider two sets 
\begin{center}
    \intro{register valuations}: $(R \to A)$
  \qquad
    \intro{register updates}: $(R \to (A+R)^+)$
\end{center}
Below we show two examples of \kl{register updates}, presented as assignments, using two \kl{registers} $X,Y$ and the semigroup $A = a^*$. (The right-hand sides are the values in $(A+R)^+$.)
\begin{align*}
    \myunderbrace{
    \begin{array}{rcl}
        X &:=& aYaXaaa\\
    Y &:=& XaaXaa
    \end{array}
    }{copyful}
    \qquad 
    \myunderbrace{
    \begin{array}{rcl}
        X &:=& aaYaaXaaa\\
    Y &:=& aaa
    \end{array}
    }{copyless}
    \end{align*}
\AP The crucial property is being \intro{copyless} -- a \kl{register update} is called \kl{copyless} if every \kl{register name} appears in at most one right-hand side of the \kl{update}, and in that right-hand side it appears at most once. 
The main operation on these sets is \emph{application}: a \kl{register update} $u$
can be applied to a \kl{register valuation} $v$, giving a new \kl{register valuation} $vu$. 


\AP In our model of \kl{streaming string tranducers}, the registers will be updated by a
stream of \kl{register updates} that is produced by a \kl{rational function}, defined as
follows. Intuitively speaking, a \kl{rational function} corresponds to an automaton
that produces one output letter for each input position, with the output letter
depending on regular properties of the input position within the input string.
More formally:
\begin{definition}
  A \intro{rational function} of type $\Sigma^* \to X^*$ -- where $\Sigma$ is
  a finite alphabet but $X$ can be any set -- is a length-preserving\footnote{Often in the literature, rational
    functions are not required to be length-preserving, see
    e.g.~\cite[p.~525]{sakarovitch2009elements}, but in this paper, we only need
    the length-preserving case.} function with the following property: for some~family\footnote{The family $(f_a)_{a\in\Sigma}$ is very close to what is called an \emph{(Eilenberg) bimachine} in the literature.}
  \[ \qquad\qquad f_a \colon \myunderbrace{\Sigma^* \times \Sigma^*}{\qquad equipped with componentwise multiplication} \to \Gamma\quad \text{for $a \in \Sigma$ of \kl{recognizable} functions,} \]
  for every input $a_1 \dots a_n$ and $i\in\{1,\dots,n\}$, the $i$-th output letter is $f_{a_i}(a_1 \dots a_{i-1},\; a_{i+1} \dots a_n)$.
\end{definition}
Note that the range of a rational function with codomain $X^*$ may contain only
finitely many \enquote{letters} from $X$, so it can always be
seen as a string function over finite alphabets.

\AP Having defined \kl{register updates} and \kl{rational functions}, we are ready to introduce the machine model used in this paper as the reference definition of \intro{regular functions}.

\begin{definition}\label{def:usual-sst}
    The syntax of a \intro{streaming string transducer} (\sst) is given by:
\begin{itemize}
    \item A finite \emph{input alphabet} $\Sigma$ and an \emph{output semigroup $A$}.
    \item A finite set $R$ of \emph{register names}. All \kl{register valuations} and \kl{updates} below use $R$ and $A$.
    \item A designated \intro{initial register valuation}, and a \intro{final
        output pattern} in $R^+$ (that does not need to be copyless, though
      adding this restriction would not affect the expressive power).
    \item An \intro{update oracle}, which is a \kl{rational function} of type 
        $\Sigma^* \to (\text{copyless register updates})^*$.
\end{itemize}
\end{definition}
The semantics of the \sst{} is a function of type $\Sigma^* \to A$ defined as follows. When given an input string, the \sst{} begins in the designated \kl{initial register valuation}. Next, it applies all \kl{updates} produced by the \kl{update oracle}, in left-to-right order. Finally, the output of the \sst{} is obtained by combining the last register values according to the \kl{final output pattern}.

\begin{example}
  We define an \sst{} that computes the function of \Cref{ex:prefix-suffix}. It has two registers $X$ and $Y$, whose initial valuation is $X=Y=\varepsilon$, and the final output pattern is $YX$. The update associated to an input letter $\ell\in\{a,b,c\}$ at position $i$ is:
  \begin{itemize}
    \item if the position $i$ is part of the longest $c$-free prefix, then $X := X\ell$, otherwise $X:=X$;
    \item if the position $i$ is part of the longest $c$-free suffix, then $Y := Y\ell$, otherwise $Y:=Y$.
  \end{itemize}
  This sequence of updates can be produced by a rational function generated by a family of functions $(f_\ell)_{\ell\in\{a,b,c\}}$ that are \kl{recognized} by $\mathbb{B}^2$, where $\mathbb{B}$ is the monoid of booleans with conjunction (rephrase the conditions as \enquote{there is no $c$ to the left (resp.~right) of $i$}).
\end{example}

In a \kl{rational function}, the label of the $i$-th output position is allowed to depend on letters of the input string that are on both sides of the $i$-th input position; this corresponds to regular lookaround in a streaming string transducer. Therefore, the model described above is easily seen to be equivalent to copyless \sst{}s with regular lookaround, which are one of the equivalent models defining the regular string-to-string functions, see~\cite[Section~IV.C]{AlurFT12}.

\subsection{From a regular function to a transducer semigroup}
\label{sec:easy}

\AP Having defined the transducer model, we prove the easy implication in
\Cref{thm:regular-functions}. It is apparent from \Cref{def:usual-sst} that every
\kl{regular function} can be decomposed as a \kl{rational function} followed by a function
computed by a \kl{streaming string transducer} whose $i$-th \kl{register update} depends
only on the $i$-th input letter -- let us call that a \intro{local} \sst. Thanks
to closure under composition (\Cref{prop:composition}), we only need to handle
these two special cases: we show that \kl{finiteness-preserving} \kl{transducer
  semigroups} recognize
\begin{itemize}
\item all \kl{rational functions} in \Cref{sec:rational};
\item and all \kl{local} \kl{streaming string transducers} in \Cref{sec:local}.
\end{itemize}

\subsubsection{Recognizing rational functions by transducer semigroups}
\label{sec:rational}

Consider a \kl{rational function}, generated by the family $(f_a)_{a\in\Sigma}$ of \kl{recognizable functions} of type
$\Sigma^* \times \Sigma^* \to \Gamma$. By
definition of recognizability, each $f_a$ decomposes into
\[ \Sigma^* \times \Sigma^* \xrightarrow{\;h_a\;} B_a \xrightarrow{\;g_a\;} \Gamma
  \qquad\text{where $h_a$ is a semigroup homomorphism and $B_a$ is finite.} \]
One can check that every $f_a$ then factors through a monoid morphism to the
finite monoid
\[ \prod_{a \in \Sigma} h_a(\Sigma^* \times \Sigma^*) \]
Thus, without loss of generality, we may assume for the rest of the proof that
all of the above semigroups $B_a$ are equal to a common finite monoid $B$ and
that each semigroup homomorphism $h_a$ is in fact a monoid morphism.

For any semigroup $A$, we let\footnote{A construction similar in
  spirit to the classical \emph{two-sided semidirect
    product}~\cite[\S6]{rhodes1989kernel}.} $\functor A = B \times (B\to A) \times B$, endowed with the following semigroup operation:
\begin{align*}
  (\ell_1,\varphi_1,r_1) \cdot (\ell_2,\varphi_2,r_2) = \Big(\ell_1\ell_2,\; \big(b  \mapsto \varphi_1(br_2) \cdot \varphi_2(\ell_1b)\big),\; r_1r_2\Big).
\end{align*}
The construction $\functor$ is extended to morphisms by considering $B\to A$ as the set of $B$-indexed tuples (cf.~\Cref{ex:functors}) of
elements of $A$. To get a \kl{transducer semigroup}, we take the \kl{output mechanism} to
be $(\ell,\varphi,r) \mapsto \varphi(e)$ where $e \in B$ is the neutral element.

Our \kl{rational function} is then recognized by the unique monoid homomorphism of
type $\Sigma^* \to \functor(\Gamma^*)$ (indeed, $\functor$ preserves monoids)
which maps $a \in \Sigma$ to
$\big(h_a(a,\varepsilon),g_a,h_a(\varepsilon,a)\big)$.

\subsubsection{From a local SST to a transducer semigroup}
\label{sec:local}

\AP Suppose now that a string-to-semigroup function $f\colon \Sigma^* \to A$ is
computed by some \kl{local} \kl{streaming string transducer}. In the proof below, when referring to \kl{register valuations} and \kl{register updates}, we refer to those that use the \kl{registers} and output semigroup of the fixed transducer. We say that a \kl{register update} is in \intro{normal form} if, in every right-hand side, one cannot find two consecutive letters from the semigroup $A$.
Any \kl{register update} can be \intro{normalized}, i.e.~converted into one that is in \kl{normal form}, by using the semigroup operation to merge consecutive elements of the output semigroup in the right-hand sides.
Here is an example, which uses three registers $X,Y,Z$ and the semigroup $A = (\set{0,1}, \cdot)$:
\begin{align*}
  \myunderbrace{
  \begin{array}{rcl}
    X &:=& 01Y1111X111\\
    Y &:=& 01011
  \end{array}
  }{not in \kl{normal form}}
  \qquad \xrightarrow{\;\text{\kl{normalization}}\;} \qquad
  \myunderbrace{
  \begin{array}{rcl}
    X &:=& 0Y1X1\\
    Y &:=& 0
  \end{array}
  }{in \kl{normal form}}
\end{align*}
The \kl{register updates} before and after \kl{normalization} act in the same way on
\kl{register valuations}. If an \kl{update} is \kl{copyless} and in \kl{normal form}, then the
combined length of all right-hand sides is at most three times the number of
registers. Therefore, if a semigroup $A$ is finite, then the set of \kl{copyless}
\kl{register updates} in \kl{normal form}, call it $\functoru A$, is also finite.
(However, there are infinitely many copyful register updates even when $A$ is
finite.) This set $\functoru A$ can be equipped with a composition operation
\begin{align*}
    u_1,u_2 \in \functoru A  \quad \mapsto \quad u_1u_2 \in \functoru A,
\end{align*}
which is defined in the same way as applying a register update to a register
valuation, except that we normalize at the end. This composition operation is
associative, and  compatible with applying \kl{register updates} to \kl{register valuations}, in the sense that $(vu_1)u_2 = v(u_1u_2)$ holds for every \kl{valuation} $v$ and all \kl{updates} $u_1$ and $u_2$. Therefore, $A \mapsto \functoru A$ is a \kl{finiteness-preserving} semigroup-to-semigroup functor (with the natural extension to morphisms, where the homomorphism is applied to every semigroup element in a right-hand side). 

The functor $\functoru$ described above is almost but not quite the functor that
will be used in the \kl{transducer semigroup} that we will define to prove the easy
implication in Theorem~\ref{thm:regular-functions}. That functor $\functor$ will
also take into account the \kl{initial register valuation}:
\[ \functor A =  \functoru A \times \myunderbrace{(R \to A)}{\qquad\qquad\qquad\qquad\qquad endowed with the trivial \kl{left zero} semigroup structure} \quad\text{with componentwise multiplication \& action on morphisms} \]
Given $(u,v) \in \functor A$, the \kl{output mechanism} in the \kl{transducer semigroup}
applies the \kl{register update} $u$ to the \kl{register valuation} $v$, and then multiplies together the register values given by the resulting \kl{valuation} $vu$ according to the \kl{final output pattern}.
Using this, we can recognize $f$ via the homomorphism that sends each input letter to:
\begin{itemize}
\item the \kl{register update} that this letter determines (our \sst being \kl{local}) in the first component;
\item the designated \kl{initial register valuation} in the second component.
\end{itemize}





\subsection{Term operations and natural transformations}
We now turn to proving the right-to-left implication in Theorem~\ref{thm:regular-functions}.

Let us begin with some notation. We write $1$ for the singleton semigroup; it is a \emph{terminal object}, that is, it admits a unique homomorphism from every other semigroup $A$, which will be denoted by $! : A \to 1$ (the notation has no connection with the factorial function on numbers). Another construction that will be used heavily in the proof is the \emph{coproduct} of two semigroups $A$ and $B$, which is denoted by $A \oplus B$. This is a semigroup which consists of words over an alphabet that is the disjoint union of $A$ and $B$, restricted to words which are nonempty and alternating in the sense that two consecutive elements cannot belong to the same semigroup. The semigroup operation is defined in the expected way. The coproduct deserves its name due to the following universal property: for every pair of semigroup homomorphisms
\begin{align*}
f : A \to C \qquad \text{and} \qquad g : B \to C
\end{align*}
there is a unique semigroup homomorphism
\[
\begin{tikzcd}
f \text{ or } g : A \oplus B \to C
\end{tikzcd}
\]
that coincides with $f$ (resp.\ $g$) on the subsemigroup of $A \oplus B$ consisting of words with a single letter from $A$ (resp.\ $B$).

Next, we introduce some terminology that will be used in the proof, concerning  polynomial functors, and copyless operations between them. They will be used as the register structure for an \sst in our proof. 

\subsubsection{Polynomial functors}
%The kinds of polynomial functors that we use in the proof are functors from semigroups to sets.
Define a \emph{polynomial functor} to be a functor from the category of semigroups to the category of sets, which is of the form
\begin{align*}
A \quad \mapsto \quad \coprod_{q \in Q} A^{\text{dimension of } q},
\end{align*}
where $Q$ is some possibly infinite set, called the \emph{components}, with each  component having an associated \emph{dimension} in $\set{0,1,\ldots}$. On morphisms, the functor works in the expected way, i.e.~coordinate-wise.  A \emph{finite polynomial functor} is one that has finitely many components. 

\begin{example}\label{ex:coproduct-as-polynomial-functor}
    A crucial property that will be used in our proof is that the functor
    \begin{align*}
    A \mapsto \text{underlying set of}\ \myunderbrace{A \oplus 1}{\text{coproduct with the singleton semigroup}}
    \end{align*}
    is in fact a polynomial functor (but not a finite polynomial functor). Noting that $\oplus$ makes sense as an operation on sets without a semigroup structure, we have more generally that, if $\functor$ is a polynomial functor, then so is
    \begin{align*}
        A \mapsto \myunderbrace{\functor A \oplus 1}{\text{set of alternating sequences between $\functor A$ and $1$}}
    \end{align*}
    
    This is because for every semigroup $A$ there is a bijective correspondence 
    \begin{align}\label{eq:polynomial-representation-of-coproduct}
    A \oplus 1 \quad \simeq \quad \coprod_{q \in 1 \oplus 1} A^{\text{dimension of $q$}},
    \end{align}
    where the dimension of $q$ is defined to be the number of times that the first copy of $1$ appears in $q$. Furthermore, the bijective correspondence in~\eqref{eq:polynomial-representation-of-coproduct} is natural in $A$, and therefore there is a natural bijection between the functor $A \oplus 1$ and some polynomial functor. Also, if two polynomial functors are connected by a natural bijection, then they are the same, up to renaming of the components, and therefore the representation in~\eqref{eq:polynomial-representation-of-coproduct} is unique up to renaming of components. By uniqueness, we will simply speak of $A \oplus 1$ as being a polynomial functor. 
\end{example}




\subsubsection{Copyless natural transformations.}  Among all natural  transformations between polynomial functors, we will be interested mainly in those that are  \emph{copyless}. To define this notion, we will first observe that  every natural transformation between polynomial functors  arises from some syntactic description, and within this syntactic description, the copyless restriction can easily be phrased. 

We begin with \emph{monomial functors}, i.e.~polynomial functors with one component. 
Consider two monomial functors 
\begin{align*}
\functor A = A^k \qquad 
\functorg A = A^\ell \qquad \text{where $k,\ell \in \set{0,1,\ldots}$.}
\end{align*}
What is the possible form of a natural transformation between these functors? One way to create such a natural transformation is to take a  function  of type 
\begin{align*}
\set{1,\ldots,\ell} \to \set{1,\ldots,k}^+,
\end{align*}
which will be called the \emph{syntactic description} of the natural transformation, 
and to define the natural transformation as follows: for a semigroup $A$ the natural transformations gives the function  that inputs $\bar a \in A^k$ and outputs the following tuple $A^\ell$:
\[
\begin{tikzcd}
    [column sep=2.3cm]
\set{1,\ldots,\ell}
\arrow[r, "\text{syntactic description}"]
&
\set{1,\ldots,k}^+ 
\ar[r,"\text{substitute $\bar a$}"]
& 
A^+
\ar[r,"\text{semigroup operation}"]
&
A.
\end{tikzcd}
\]
Every natural transformation between monomial functors arises this way. To see this, the syntactic description is recovered by using the natural transformation for the free semigroup $A=\set{1,\ldots,k}^+$, and applying it for the identity valuation 
\begin{align*}
x \in \set{1,\ldots,k} \quad \mapsto \quad [x] \in \set{1,\ldots,k}^+.
\end{align*}


The advantage of the syntactic description, which is unique, is that it allows us to define the  \emph{copyless restriction}:  (*)   we say that a  syntactic description
\begin{align*}
    \alpha: \set{1,\ldots,\ell} \to \set{1,\ldots,k}^+
    \end{align*}
is copyless if  every letter  from $\set{1,\ldots,k}$ appears in at most one word $\alpha(x)$, and in that word it appears at most once. An equivalent condition can be phrased semantically: (**) if we use the natural transformation in the semigroup $A = \Nat$, then the corresponding function $\Nat^k \to \Nat^\ell$ is non-expansive, i.e.\ the norm of its output is at most the norm of its input, where the norm of a vector is the sum of its coordinates. 

We now define what it means to be copyless for a natural transformation between arbitrary polynomial functors 
\begin{align*}
\functor A = \coprod_{q \in Q} A^{\dim q} \qquad 
\functorg A = \coprod_{p \in P} A^{\dim p},
\end{align*}
which are not necessarily monomial. Such natural transformations also admit syntactic descriptions: for every input component $q$, there is some designated output component $p$, and a natural transformation $A^{\dim q} \to A^{\dim p}$.  The set of possible syntactic descriptions is
\begin{align*}
\prod_{q \in Q} \coprod_{p \in P} \dim p \to (\dim q)^+.
\end{align*}
Again, one can show that all natural transformations arise this way. The natural transformation is called copyless if for every $q$, the corresponding natural transformation between monomial functors is copyless. 


\begin{example}\label{ex:copyless-on-coproducts}
    Consider the functor \enquote{underlying set of $A \oplus 1$} that was discussed in Example~\ref{ex:coproduct-as-polynomial-functor}, and shown to be a polynomial functor. The natural transformation 
    \begin{align*}
    (A \oplus 1)\times (A \oplus 1) \to A \oplus 1
    \end{align*}
    which describes the semigroup operation in the coproduct $A \oplus 1$ is copyless.
\end{example}


\subsubsection{Views}
\label{sec:views}

\newcommand{\combine}{\mathrm{combine}}
\newcommand{\view}{\mathrm{view}}

We now describe a crucial property of the coproduct of semigroups, which is behind the proof of Theorem~\ref{thm:regular-functions}. The idea is that an element of a binary coproduct can be uniquely defined from its views onto the individual coordinates, as defined below. 
For two semigroups $A$ and $B$, define the \emph{$A$-view} and \emph{$B$-view} to be the homomorphisms
\[
    \view_A = \id_A \oplus ~!~ : A \oplus B \to A \oplus 1 \qquad 
    \view_B = ~!~ \oplus \id_B : A \oplus B \to 1 \oplus B
\]
These are natural transformations in $A$ and $B$. The important property of views is that they give complete information about the coproduct, i.e.~if we have all views then we can reconstruct an element of the coproduct.
%furthermore this reconstruction almost corresponds to an isomorphism of polynomial functors, so in particular it is copyless.
This is stated in the following lemma. 

\begin{lemma}
\label{lem:views}
The map $A \oplus B \to (A \oplus 1) \times (1 \oplus B)$ obtained by pairing $\view_A$ and $\view_B$ is \emph{injective} and natural in $A$ and $B$.
\end{lemma}
\begin{proof}
    Straightforward.
\end{proof}
\tito{Fixed. The correct generalization to $n=3$ is certainly something like $(A \oplus 1 \oplus 1) \times (1 \oplus B \oplus 1) \times (1 \oplus 1 \oplus C)$. The statement doesn't include a copyless partial inverse something to avoid having to talk about polynomial functors in several variables.}

This lemma seems to contain the essential property of semigroups that makes the construction work. Our theorem will also be true for other algebraic structures in the lemma is true, such as forest algebras. However, the lemma seems to fail for certain algebraic structures, such as groups, even if we allow $1$ to be replaced by some fixed finite group. Another example where the lemma seems to fail is the monad of weighted sums of words (i.e.~this monad corresponds to weighted automata).
\tito{I wonder if the important thing is not more simply that $A \oplus B$ is a polynomial bifunctor}

\subsubsection{Functorial streaming string transducers}
\label{sec:functorial-sst}
We now describe the last ingredient in our proof, which is a more abstract variant  of streaming string transducers (\sst) that is described in Definition~\ref{def:functorial-sst}. 

Before presenting the abstract definition, we discuss how it differs  from the usual of \sst. The first difference, which is least important, there is some abstract output semigroup $A$ instead of a free semigroup $\Gamma^+$; this generalization is only meant to have cleaner notation. The second, and more important, difference is that, instead of having a fixed number of registers, we allow the register structure to be a finite polynomial functor such as 
\begin{align*}
\functorr A = A^3 + A^2 + A^2 + A + 1.
\end{align*}
The idea is that the register structure already contains the states; with the states corresponding to components in the disjoint union, and with  different states using different numbers of registers. The final difference is that the transducer is allowed to have a look at regular properties of the string on both sides of the head:  when the head is over some position in the input string, then the way in which the registers are is decided based on  some recognizable property of 
\begin{align*}
\myunderbrace{\Sigma^*}{before \\ head } \times 
\myunderbrace{\Sigma}{under \\ head } \times 
\myunderbrace{\Sigma^*}{after \\ head }.
\end{align*}
(As usual, the register update must be copyless.)
By a recognizable property of the above we mean a function that inputs elements of the above set of triples  and outputs elements of some set $X$, which can be be decomposed as 
\[
\begin{tikzcd}
    \Sigma^* \times \Sigma \times \Sigma^* 
    \ar[r,"h \times \id \times h"] 
    &
    M \times \Sigma \times M
\ar[r,"f"]
& 
X
\end{tikzcd}
\] 
for some homomorphism $h$ into a finite monoid and some function $f$. In a sense, this model has two features that replace states: the disjoint unions in the register structure, and the recognizable property. Each of these features alone would be enough, but for our intended application having both features will give a cleaner construction.

Here is the formal definition of our \sst model.

\begin{definition}\label{def:functorial-sst}
    A functorial \sst is defined by:
    \begin{itemize}
    \item a finite  input alphabet $\Sigma$;
    \item a (not necessarily finite) output semigroup $A$;
    \item two finite polynomial functors $\functorr, \functors$ , called the \emph{register functor} and the \emph{update functors}, along with  three copyless natural transformations \begin{align*}
    \vdash\;: 1 \to \functorr A \qquad \delta : \functorr A \times \functors A \to \functorr A \qquad \myunderbrace{\dashv\;: \functorr A \to A}{can be partial}
    \end{align*}
    \tito{why make it partial?}
    \item an \emph{oracle}, which is a recognizable function 
    \begin{align*}
    o : \Sigma^* \times \Sigma \times \Sigma^* \to \functors A.
    \end{align*}
    
    \end{itemize}
\end{definition}

The function computed by a functorial \sst is the partial function of type 
\begin{align*}
\Sigma^+ \to A
\end{align*}
that is defined as follows. Consider some input string in $\Sigma^+$. The machine moves its head along all input positions, and computes for each one a register valuation in $\functorr A$. In the initial register valuation corresponding to $i=0$, when no positions were processed yet, the register valuation  is obtained by applying $\vdash$ to the unique element $1$. For $i > 0$, the $i$-th register valuation is obtained by applying $\delta$ to the pair which consists of the $(i-1)$-st register valuation and the result of applying the oracle to input string with the $i$-th position distinguished. Finally, the output of the \sst is obtained by applying the output term operation $\dashv$ to the last  register valuation.

\begin{lemma}\label{lem:functorial-sst-complete}
    A function of type $\Sigma^+ \to A$ is regular, in the standard sense, if and only if it is computed by a functorial \sst. 
\end{lemma}
\begin{proof}
    A normal \sst with states $Q$ and $k$ registers can be seen as a functorial \sst with a register functor of the form 
    \begin{align*}
    \functorr A = \coprod_{q \in Q} (A + 1)^k.
    \end{align*}
    The oracle function does not look at any properties of the input string (all appropriate information is remembered in the implicit state from the register functor), and it simply outputs all elements of the output semigroup that might potentially be used in an update, with sufficient copies 
    \begin{align*}
    \functors A = A^\ell
    \end{align*}
    to make the function $\delta$ copyless. 

    Consider now the other implication in the lemma, which is the one that we use in this proof. In the case when the register functor is $A^k$ for some $k$,  a functorial \sst is a special case of an \sst with regular lookahead; and regular lookahead can be eliminated~\cite[Lemma 13.6]{bojanczyk_automata_2018}. To accommodate more general polynomial functors as register functors, we observe that the component in a polynomial functor can be stored in the state of an \sst, and the register values can be stored in $A^k$ for sufficiently large $k$. 
\end{proof}


\subsection{Proof}

We now present proof of the right-to-left implication in Theorem~\ref{thm:regular-functions}. Consider some transducer semigroup, with the functor being $\functor$ and the output transformation being $\outfun$. We will show that for every finite alphabet $\Sigma$, every semigroup $A$ (not necessarily a free semigroup $\Gamma^+$), the composition 
\[
    \begin{tikzcd}
    \Sigma^+ 
    \ar[r,"h"]
    &
    \functor A
    \ar[r,"\outfun_A"]
    &
    A
    \end{tikzcd}
    \]
is recognized by a functorial \sst.


We begin by introducing some notation, which allows us to track which parts of an output come from which parts of an input. The new notation will be explained using the running example of the duplicating function 
\begin{align*}
f : \set{a,b}^+ \to \set{a,b}^+
\end{align*}
which is recognized by the 
transducer semigroup 
in which the functor $\functor$ is the identity, the output mechanism is duplication, and the homomorphism 
\begin{align*}
h : \set{a,b}^+ \to \functor \set{a,b}^+ = \set{a,b}^+
\end{align*}
is the identity.

%  Let us extend the output mechanism from semigroups to co-products of semigroups. This is done as follows, for semigroups $A_1,\ldots,A_n$, define the corresponding output mechanism to be the composition of the following functions 
% \[
% \begin{tikzcd}
% \functor A_1 \oplus \cdots \oplus \functor A_n
% \ar[d, "\functor \iota_1 \oplus \cdots \oplus \functor \iota_n"]\\
% \functor (A_1 \oplus \cdots \oplus A_n) \oplus \cdots \oplus \functor (A_1 \oplus \cdots \oplus A_n)
% \ar[d, "\outfun_{A_1} \oplus \cdots \oplus \outfun_{A_n}"]\\
% A_1 \oplus \cdots \oplus A_n
% \end{tikzcd}
% \]
% We use the name \emph{co-product output mechanism} for the function described above. 
For semigroups $A_1,\ldots,A_n$, define the \emph{factorized output function} to be the 
function of type 
\begin{align*}
\functor A_1 \times \cdots \times \functor A_n \to A_1 \oplus \cdots \oplus A_n
\end{align*}
that is obtained by composing the three functions described below
\tito{I believe the previous def was wrong (not consistent with the example), I've changed it to a non-equivalent one that looks like it does what we want}
\[
\begin{tikzcd}
\functor A_1 \times \cdots \times \functor A_n
\ar[d,"\functor(\text{co-projection}) \times \cdots \times \functor(\text{co-projection})"]
\\
\functor(A_1 \oplus \cdots \oplus A_n)
\times
\cdots
\times 
\functor(A_1 \oplus \cdots \oplus A_n)
\ar[d,"\text{semigroup operation}"]
\\
\functor(A_1 \oplus \cdots \oplus A_n)
\ar[d, "\outfun_{A_1 \oplus \cdots \oplus A_n}"]\\ 
A_1 \oplus \cdots \oplus A_n.
\end{tikzcd}
\]

Let us illustrate the factorized output function on our running example, with 
\begin{align*}
A_1 = 1 \qquad A_2 = \set{a,b}^+.
\end{align*}
If we apply the factorized output function to 
\begin{align*}
(1, abbb)  \in \functor A_1 \times \functor A_2 = A_1 \times A_2
\end{align*}
then the output will be 
\begin{align*}
1 abbb 1 abbb.
\end{align*}

If we fix $n$, then the factorized output function is a natural in the sense that the following diagram commutes 
\[
\begin{tikzcd}
    [column sep=2cm]
\functor A_1 \times \cdots \times \functor A_n
\ar[r,"\text{factorized output}"]
\ar[d,"\functor h_1 \times \cdots \times \functor h_n"']
&
A_1 \oplus \cdots \oplus A_n
\ar[d,"h_1 \oplus \cdots \oplus h_n"]
\\
\functor B_1 \times \cdots \times \functor B_n
\ar[r,"\text{factorized output}"']
&
B_1 \oplus \cdots \oplus B_n
\end{tikzcd}
\]
for every semigroup homomorphisms $h_i : A_i \to B_i$. In other words, the factorized output function is a natural 
transformation of type 
\[\begin{tikzcd}
    [column sep=1cm]
    {\text{Semigroups}^n} && {\text{Sets}}
    \arrow[""{name=0, anchor=center}, "{(A_1,\ldots,A_n)} \mapsto \text{underlying set of } \functor A_1 \times \cdots \times \functor A_n", curve={height=-18pt}, from=1-1, to=1-3]
    \arrow[""{name=1, anchor=center, inner sep=0}, "{(A_1,\ldots,A_n)} \mapsto \text{underlying set of } A_1 \oplus \cdots \oplus A_n"', curve={height=18pt}, from=1-1, to=1-3]
    \arrow[ shorten <=5pt, shorten >=5pt, Rightarrow, from=0, to=1]
\end{tikzcd}\]
The reason for naturality is that each of the three steps in the definition of the factorized output is itself a natural transformation, and natural transformations compose. 


% By abuse of notation, we allow some -- but not all -- of the arguments in the factorized output to be empty; in this case the empty arguments are ignored, but the output type is still a co-product of $n$ semigroups. For example, if the input to the factorized output function is 
% \begin{align*}
% (1, \varepsilon) \in 1 \oplus \set{a,b}^+
% \end{align*}
% then the factorized  output is  
% \begin{align*}
% 1 1     \in 1 \oplus \set{a,b}^+.
% \end{align*}
    

For  input strings $w_1,\ldots,w_n \in \Sigma^+$ let use write 
\begin{align*}
\tuple{w_1| \cdots | w_n} \in \myunderbrace{A \oplus \cdots \oplus A}{$n$ times}
\end{align*}
to be the result of first applying $h$ to all the strings, and then applying the factorized output function. By abuse of notation, we allow some, but not all of the input strings to be empty, in which case the appropriate values in the coproduct are ignored, but the output type is not changed. In our running example, we have
\begin{align*}
    \tuple{abbbbb | \varepsilon | bbabaaa} =  
    \myunderbrace{abbbbb}{in first \\ copy of \\ $\set{a,b}^+$}
     \myunderbrace{bbabaaa}{in third \\ copy of \\ $\set{a,b}^+$}
     \myunderbrace{abbbbb}{in first \\ copy of \\ $\set{a,b}^+$}
     \myunderbrace{bbabaaa}{in third \\ copy of \\ $\set{a,b}^+$}.
    \end{align*}


We also use a similar notation but with some strings underlined; in the underlined case, to the non-underlined strings we apply $h$, and to the underlined strings we apply 
\[
\begin{tikzcd}
\Sigma^+ 
\ar[r,"h"]
&
\functor A 
\ar[r,"\functor !"]
&
\functor 1.
\end{tikzcd}
\]



The following lemma is the key part of our construction. We consider $A \mapsto A \oplus 1$ and $A \mapsto 1 \oplus A \oplus 1$ as a polynomial functors as explained in Example~\ref{ex:coproduct-as-polynomial-functor}.
\begin{lemma}\label{lem:compute-next-configuration}
    There exists a \emph{copyless} natural transformation of polynomial functors
    \begin{align*}
    \delta_A : (A \oplus 1) \times (1 \oplus A \oplus 1) \to A \oplus 1
    \end{align*} 
    and some function 
    \begin{align*}
    o : \Sigma^* \times \Sigma \times \Sigma^*  \to  1 \oplus A \oplus 1
    \end{align*}
    which is recognizable in the sense that was defined in Section~\ref{sec:functorial-sst}, such that  for every strings $w,v \in \Sigma^*$ and every letter $a \in \Sigma$ we have 
    \begin{align*}
    \tuple{wa|\underline v}  = \delta_A(\tuple{w|\underline{av}}, o(w,a,v)).
    \end{align*}
\end{lemma}
\begin{proof}
    We begin with the following claim, which shows that there is no difference if we merge parts before or after applying the factorized output operation. It is obtained by expanding the latter's definition, using the naturality of $\outfun$ and the fact that  both $h$ and merging the first two \enquote{copies of $A$} in $A \oplus A \oplus 1$ (by copairing coprojections) are homomorphisms.
    \begin{claim}\label{claim:merge-factorized-output}
        The following diagram commutes 
        \[
        \begin{tikzcd}
            [column sep=3cm]
        \Sigma^* \times \Sigma \times \Sigma^* 
        \ar[r,"{(w,a,v) \mapsto \tuple{w|a|\underline v}}"]
        \ar[d,"{(w,a,v) \mapsto (wa,v)}"']
        &
        A \oplus A \oplus 1
        \ar[d,"\text{merge first two coordinates}"]
        \\
        \Sigma^+ \times \Sigma^* 
        \ar[r,"{(wa,v) \mapsto \tuple{wa|\underline v}}"']
        &
        A  \oplus 1
        \end{tikzcd}
        \]
    \end{claim}
    \tito{I checked the claim for the new definition of factorized output function, but I don't know if the details are worth spelling out (it's purely mechanical)}

    Thanks to the above claim, we know that  the desired value $\tuple{wa|\underline v}$ is obtained from  $\tuple{w|a| \underline v}$
    by applying a copyless natural transformation, namely merging the first two coordinates. 
     Lemma~\ref{lem:views} gives us a natural injection
    \[ \iota_A : A \oplus (A \oplus 1) \xrightarrow{\;\text{pairing of $\view_A$ and $\view_{A \oplus 1}$}\;} (A \oplus 1) \times (1 \oplus A \oplus 1) \]
    and, using again the naturality of the components in the factorized output function, we have
    \[ \iota(\tuple{w|a| \underline v}) = (\tuple{w|\underline{av}},\tuple{\underline w| a| \underline v}) \]
    The inverse of $\iota_A$, which is partial, can be extended to a total function
    \[ \rho_A : (A \oplus 1) \times (1 \oplus A \oplus 1) \to A \oplus (A \oplus 1) \]
    by mapping all elements outside the range of $\iota$ to the unique element of $A \oplus (A \oplus 1)$ that does not involve $A$. This is natural in $A$ and copyless.
        \tito{TODO justify natural and copyless}
    
    To conclude, by setting
    \[ \delta_A = (\text{merge two $A$s in $A \oplus A \oplus 1$}) \circ \rho_A \qquad o(w,a,v) = \tuple{\underline w| a| \underline v} \]
    we satisfy the equation in the lemma statement. It remains to check that the map $o$ is recognizable. This is the case because $\tuple{\underline w| a| \underline v}$ depends only on the letter $a$ and the images of $w$ and $v$ under the homomorphism $\functor ! \circ h$, which has finite codomain. This last step is crucial: this is where we use the assumption that $\functor 1$ is finite.
\end{proof}


Thanks to the above lemma, we can construct a device which is almost a functorial \sst as described in Section~\ref{sec:functorial-sst} and which recognizes our desired function $w \mapsto \tuple{w}$. We say ``almost'', because the device will use register and update functors that are  infinite polynomial functors; this construction will be later improved so that it becomes finite.  The register and update functors are 
\begin{align*}
\functorr A  = A \oplus 1 \qquad \functors A = 1 \oplus A \oplus 1
\end{align*}
As mentioned above, these are  not a finite polynomial functors; we will resolve this problem shortly. Beyond that, the construction is immediate: if we take the register update term $\delta$ and the oracle $o$ as in the above lemma, and we define the initial register value to be 
\begin{align*}
\myunderbrace{\tuple{\varepsilon | \underline{a_1 \cdots a_n}}}{this does not depend on the input string $a_1 \cdots a_n$, \\ it is equal to the unique element of $1 \oplus A$ that does not use $A$}
\end{align*}).
then the resulting machine will have the property that  if the input string is $a_1 \cdots a_n$, then its $i$-th configuration is 
\begin{align*}
\tuple{a_1 \cdots a_{i}| \underline{a_{i+1} \cdots a_n}}.
\end{align*}
In particular, the last configuration is $\tuple{a_1 \cdots a_n | \varepsilon}$, which is the same as the output when viewed as an element of $A \oplus 1$. Therefore, to get the output, we should apply the partial copyless natural transformation of type $A \oplus 1 \to A$ which is the one-sided inverse of the embedding of $A$ in $A \oplus 1$.  
\tito{It's a bit inelegant since our categorical definition yields total functions}

The only remaining issue with our construction is that the two functors, i.e.~the register functor and the update functors, are not finite polynomial functors. This problem is overcome by observing that not all of $1 \oplus A$ need be used for the register values, only a small part of it, and likewise for the update functor. More formally, 
consider the natural bijection
\begin{align*}
A \oplus 1  \quad \cong \quad 
\myunderbrace{\coprod_{q \in 1 \oplus 1}  A^{\dim q}}{call this the \emph{register representation} of $A \oplus 1$}
\end{align*}
that was discussed in Example~\ref{ex:coproduct-as-polynomial-functor}. If we apply it to an element of the form
\begin{align*}
\tuple{w|\underline {v}} \in A \oplus 1,
\end{align*}
then the corresponding component will be $\tuple {\underline w | \underline v}$. Since the latter depends only  on $\underline w$ and $\underline v$, and these take values in the finite semigroup $\functor 1$, it follows that there are only finitely many components of $A \oplus 1$ that will be used to represent values from of the form $\tuple {w | \underline v}$. Therefore, instead of using $\functorr A$ to be all of $A \oplus 1$, we can restrict it to those finitely many components, giving thus a finite polynomial functor. The same argument can be applied to the update functor $\functors A$.