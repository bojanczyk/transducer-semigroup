\section{The regular functions}
\label{sec:reg-char}
The two straightforward constructions in Theorems~\ref{thm:all-functions} and~\ref{thm:reco-reflecting-functions}  amount to little more than symbol pushing. In this section, we present a more substantial characterization, which is the main result of this paper.
% In this characterization, we use functors that are finiteness-preserving.
This characterization concerns finiteness-preserving functors.
% (above rephrased mainly to avoid vertical alignment of two occurrences of "finiteness-preserving")
This is a strengthening of the condition from Theorem~\ref{thm:reco-reflecting-functions}: if the functor $\functor$ in a transducer semigroup is finiteness-preserving, then for every finite semigroup $A$, the output function $\functor A \to A$ will be recognizable, since all functions from a finite semigroup are trivially recognizable.  However, the condition is strictly stronger, as witnessed by Example~\ref{ex:squaring}, which is recognizability reflecting (cf.~\Cref{ex:squaring-reco-refl}) but not finiteness preserving. As we will see, the stronger condition will characterize exactly the regular string-to-string functions.

The following counterexample illustrates the non-trivial interaction between naturality of the output mechanism interacts and the requirement that 
the functor is finiteness preserving.

\begin{example}
    Consider the powerset functor $\powerset$ from Example~\ref{ex:functors}. It is finiteness-preserving, since the powerset of a finite semigroup is also finite. One could imagine that using powersets, one could construct a transducer semigroup that recognizes functions that are not regular, e.g.~because they have exponential growth (unlike regular functions, which have linear growth). It turns out that this is impossible, because there is no possible output mechanism, i.e.~no natural transformation of type $\powerset A \to A$, as we explain below.

    The issue is that the naturality condition disallows choosing elements from a subset.  To see why, consider a semigroup $A$ with two elements, with the trivial left zero semigroup structure. For this semigroup, the output mechanism of type $\powerset A \to A$ would need to choose some element $a \in A$ when given as input the full set $A \in \powerset A$. However, none of the two choices is right, because swapping the two elements of $A$ is an automorphism of the semigroup $A$, which maps the full set to itself, but does not map any element to itself.
\end{example}

We now state the main theorem of this paper. 
% Unlike the previous characterizations, it concerns the functions from a free monoid $\Sigma^*$ over a finite alphabet to a semigroup $A$, because the models defining regular functions operate on strings, and not on abstract semigroups. Some of the models, such as streaming string transducers (\sst) or two-way automata, easily make sense when the output is an abstract semigroup, but the string structure of the input seems to be essential for all the models. 

% Among the functors described in Example~\ref{ex:functors},  ``reverse'' and ``powerset''  are finiteness preserving, in the sense that if they are applied to a finite semigroup, then the result is also a finite semigroup. The ``tuple'' functor $A^n$ is finiteness preserving if and only if the exponent $n$ is finite. The ``list'' functor $A^+$ is not finiteness preserving. 

\begin{theorem}\label{thm:regular-functions}
    The following conditions are equivalent for every string-to-string function:
    \begin{enumerate}
        \item \label{it:regular} it is a regular string-to-string function;
        \item \label{it:trans-semig-regular}it is recognized by a transducer semigroup  in which the functor is finiteness preserving. 
    \end{enumerate}
\end{theorem}

\noindent Here is the plan for the rest of this section:
\begin{description}
    \item[Section~\ref{sec:sst-definition}] gives a formal definition of regular functions
    \item[Section~\ref{sec:easy}] proves the easy  implication in the theorem, namely  $(\ref{it:regular}) \Rightarrow (\ref{it:trans-semig-regular})$
    \item[Section~\ref{sec:hard}] proves the hard  implication in the theorem, namely  $(\ref{it:regular}) \Leftarrow (\ref{it:trans-semig-regular})$
\end{description}

Before continuing, we remark on one advantage of the characterization, namely a
straightforward proof of closure under composition. In contrast, for some (but
not all) existing models defining regular string-to-string functions,
composition requires a non-trivial construction -- examples include two-way
transducers~\cite[Theorem 2]{ChytilJ77} or copyless \sst~\cite[Theorem
1]{composingSST}.
\begin{proposition}\label{prop:composition}
  Functions recognized by finiteness-preserving transducer semigroups are closed
  under composition.
\end{proposition}
\begin{proof}
  This is because finiteness-preserving functors are closed under composition,
  natural output functions are also closed under composition, and naturality
  means by definition that the output functions \enquote{commute} in a suitable
  sense with functors.\footnote{More precisely, consider the following diagram,
    where the upper path describes the composition of two functions recognized
    transducer semigroups $(\functor,\outfun)$ and $(\functor',\outfun')$,
    respectively:
    \[
      \begin{tikzcd}[ampersand replacement=\&, column sep=2cm]
        \Sigma^* 
        \ar[r,"h"]
        \& 
        \functor(\Gamma^*)
        \ar[r,"\outfun_{\Gamma^*}"]
        \ar[d,"\functor h'"]
        \&
        \Gamma^* \ar[d,"h'"]\\
        \&
        \functor\functor'(\Pi^*) \ar[r,"\outfun_{\functor'(\Pi^*)}"]
        \& 
        \functor'(\Pi^*)
        \ar[r,"\outfun'_{(\Pi^*)}"]
        \&
        \Pi^*
      \end{tikzcd}
    \]
    The rectangle in the middle commutes by naturality, and therefore the upper path is equal to the lower path. The latter describes a function recognized by $(\functor\functor', \outfun_{\functor'(-)} \circ \outfun')$.}
\end{proof}



 
\subsection{Defininition of streaming string transducers}
\label{sec:sst-definition}
In this section, we formally describe the \kl{regular functions}, using a model based on \kl{streaming string transducers} (\sst).  This model, like our proof of Theorem~\ref{thm:regular-functions}, covers a slightly more general case, namely string-to-semigroup functions instead of only string-to-string functions. These are functions of type $\Sigma^* \to A$ where $\Sigma$ is a finite alphabet and $A$ is an arbitrary semigroup.  The purpose of this generalization is to make notation more transparent, since the fact that the output semigroup consists of strings will not play any role in our proof.

% The model is a minor variation on streaming string transducers, which use registers to store elements of the output semigroup.
\AP The model uses \intro{registers} to store elements of the output semigroup. We begin by describing notation for registers and their updates. Suppose that $R$ is a finite set of \kl{register names}, and $A$ is a semigroup called the \emph{output semigroup}. We consider two sets 
\begin{center}
    \intro{register valuations}: $(R \to A)$
  \qquad
    \intro{register updates}: $(R \to (A+R)^+)$
\end{center}
Below we show two examples of \kl{register updates}, presented as assignments, using two \kl{registers} $X,Y$ and the semigroup $A = a^*$. (The right-hand sides are the values in $(A+R)^+$.)
\begin{align*}
    \myunderbrace{
    \begin{array}{rcl}
        X &:=& aYaXaaa\\
    Y &:=& XaaXaa
    \end{array}
    }{copyful}
    \qquad 
    \myunderbrace{
    \begin{array}{rcl}
        X &:=& aaYaaXaaa\\
    Y &:=& aaa
    \end{array}
    }{copyless}
    \end{align*}
\AP The crucial property is being \intro{copyless} -- a \kl{register update} is called \kl{copyless} if every \kl{register name} appears in at most one right-hand side of the \kl{update}, and in that right-hand side it appears at most once. 
The main operation on these sets is \emph{application}: a \kl{register update} $u$
can be applied to a \kl{register valuation} $v$, giving a new \kl{register valuation} $vu$. 


\AP In our model of \kl{streaming string tranducers}, the registers will be updated by a
stream of \kl{register updates} that is produced by a \kl{rational function}, defined as
follows. Intuitively speaking, a \kl{rational function} corresponds to an automaton
that produces one output letter for each input position, with the output letter
depending on regular properties of the input position within the input string.
More formally:
\begin{definition}
  A \intro{rational function} of type $\Sigma^* \to X^*$ -- where $\Sigma$ is
  a finite alphabet but $X$ can be any set -- is a length-preserving\footnote{Often in the literature, rational
    functions are not required to be length-preserving, see
    e.g.~\cite[p.~525]{sakarovitch2009elements}, but in this paper, we only need
    the length-preserving case.} function with the following property: for some~family\footnote{The family $(f_a)_{a\in\Sigma}$ is very close to what is called an \emph{(Eilenberg) bimachine} in the literature.}
  \[ \qquad\qquad f_a \colon \myunderbrace{\Sigma^* \times \Sigma^*}{\qquad equipped with componentwise multiplication} \to \Gamma\quad \text{for $a \in \Sigma$ of \kl{recognizable} functions,} \]
  for every input $a_1 \dots a_n$ and $i\in\{1,\dots,n\}$, the $i$-th output letter is $f_{a_i}(a_1 \dots a_{i-1},\; a_{i+1} \dots a_n)$.
\end{definition}
Note that the range of a rational function with codomain $X^*$ may contain only
finitely many \enquote{letters} from $X$, so it can always be
seen as a string function over finite alphabets.

\AP Having defined \kl{register updates} and \kl{rational functions}, we are ready to introduce the machine model used in this paper as the reference definition of \intro{regular functions}.

\begin{definition}\label{def:usual-sst}
    The syntax of a \intro{streaming string transducer} (\sst) is given by:
\begin{itemize}
    \item A finite \emph{input alphabet} $\Sigma$ and an \emph{output semigroup $A$}.
    \item A finite set $R$ of \emph{register names}. All \kl{register valuations} and \kl{updates} below use $R$ and $A$.
    \item A designated \intro{initial register valuation}, and a \intro{final
        output pattern} in $R^+$ (that does not need to be copyless, though
      adding this restriction would not affect the expressive power).
    \item An \intro{update oracle}, which is a \kl{rational function} of type 
        $\Sigma^* \to (\text{copyless register updates})^*$.
\end{itemize}
\end{definition}
The semantics of the \sst{} is a function of type $\Sigma^* \to A$ defined as follows. When given an input string, the \sst{} begins in the designated \kl{initial register valuation}. Next, it applies all \kl{updates} produced by the \kl{update oracle}, in left-to-right order. Finally, the output of the \sst{} is obtained by combining the last register values according to the \kl{final output pattern}.

\begin{example}
  We define an \sst{} that computes the function of \Cref{ex:prefix-suffix}. It has two registers $X$ and $Y$, whose initial valuation is $X=Y=\varepsilon$, and the final output pattern is $YX$. The update associated to an input letter $\ell\in\{a,b,c\}$ at position $i$ is:
  \begin{itemize}
    \item if the position $i$ is part of the longest $c$-free prefix, then $X := X\ell$, otherwise $X:=X$;
    \item if the position $i$ is part of the longest $c$-free suffix, then $Y := Y\ell$, otherwise $Y:=Y$.
  \end{itemize}
  This sequence of updates can be produced by a rational function generated by a family of functions $(f_\ell)_{\ell\in\{a,b,c\}}$ that are \kl{recognized} by $\mathbb{B}^2$, where $\mathbb{B}$ is the monoid of booleans with conjunction (rephrase the conditions as \enquote{there is no $c$ to the left (resp.~right) of $i$}).
\end{example}

In a \kl{rational function}, the label of the $i$-th output position is allowed to depend on letters of the input string that are on both sides of the $i$-th input position; this corresponds to regular lookaround in a streaming string transducer. Therefore, the model described above is easily seen to be equivalent to copyless \sst{}s with regular lookaround, which are one of the equivalent models defining the regular string-to-string functions, see~\cite[Section~IV.C]{AlurFT12}.

\subsection{From a regular function to a transducer semigroup}
\label{sec:easy}

\AP Having defined the transducer model, we prove the easy implication in
\Cref{thm:regular-functions}. It is apparent from \Cref{def:usual-sst} that every
\kl{regular function} can be decomposed as a \kl{rational function} followed by a function
computed by a \kl{streaming string transducer} whose $i$-th \kl{register update} depends
only on the $i$-th input letter -- let us call that a \intro{local} \sst. Thanks
to closure under composition (\Cref{prop:composition}), we only need to handle
these two special cases: we show that \kl{finiteness-preserving} \kl{transducer
  semigroups} recognize
\begin{itemize}
\item all \kl{rational functions} in \Cref{sec:rational};
\item and all \kl{local} \kl{streaming string transducers} in \Cref{sec:local}.
\end{itemize}

\subsubsection{Recognizing rational functions by transducer semigroups}
\label{sec:rational}

Consider a \kl{rational function}, generated by the family $(f_a)_{a\in\Sigma}$ of \kl{recognizable functions} of type
$\Sigma^* \times \Sigma^* \to \Gamma$. By
definition of recognizability, each $f_a$ decomposes into
\[ \Sigma^* \times \Sigma^* \xrightarrow{\;h_a\;} B_a \xrightarrow{\;g_a\;} \Gamma
  \qquad\text{where $h_a$ is a semigroup homomorphism and $B_a$ is finite.} \]
One can check that every $f_a$ then factors through a monoid morphism to the
finite monoid
\[ \prod_{a \in \Sigma} h_a(\Sigma^* \times \Sigma^*) \]
Thus, without loss of generality, we may assume for the rest of the proof that
all of the above semigroups $B_a$ are equal to a common finite monoid $B$ and
that each semigroup homomorphism $h_a$ is in fact a monoid morphism.

For any semigroup $A$, we let\footnote{A construction similar in
  spirit to the classical \emph{two-sided semidirect
    product}~\cite[\S6]{rhodes1989kernel}.} $\functor A = B \times (B\to A) \times B$, endowed with the following semigroup operation:
\begin{align*}
  (\ell_1,\varphi_1,r_1) \cdot (\ell_2,\varphi_2,r_2) = \Big(\ell_1\ell_2,\; \big(b  \mapsto \varphi_1(br_2) \cdot \varphi_2(\ell_1b)\big),\; r_1r_2\Big).
\end{align*}
The construction $\functor$ is extended to morphisms by considering $B\to A$ as the set of $B$-indexed tuples (cf.~\Cref{ex:functors}) of
elements of $A$. To get a \kl{transducer semigroup}, we take the \kl{output mechanism} to
be $(\ell,\varphi,r) \mapsto \varphi(e)$ where $e \in B$ is the neutral element.

Our \kl{rational function} is then recognized by the unique monoid homomorphism of
type $\Sigma^* \to \functor(\Gamma^*)$ (indeed, $\functor$ preserves monoids)
which maps $a \in \Sigma$ to
$\big(h_a(a,\varepsilon),g_a,h_a(\varepsilon,a)\big)$.

\subsubsection{From a local SST to a transducer semigroup}
\label{sec:local}

\AP Suppose now that a string-to-semigroup function $f\colon \Sigma^* \to A$ is
computed by some \kl{local} \kl{streaming string transducer}. In the proof below, when referring to \kl{register valuations} and \kl{register updates}, we refer to those that use the \kl{registers} and output semigroup of the fixed transducer. We say that a \kl{register update} is in \intro{normal form} if, in every right-hand side, one cannot find two consecutive letters from the semigroup $A$.
Any \kl{register update} can be \intro{normalized}, i.e.~converted into one that is in \kl{normal form}, by using the semigroup operation to merge consecutive elements of the output semigroup in the right-hand sides.
Here is an example, which uses three registers $X,Y,Z$ and the semigroup $A = (\set{0,1}, \cdot)$:
\begin{align*}
  \myunderbrace{
  \begin{array}{rcl}
    X &:=& 01Y1111X111\\
    Y &:=& 01011
  \end{array}
  }{not in \kl{normal form}}
  \qquad \xrightarrow{\;\text{\kl{normalization}}\;} \qquad
  \myunderbrace{
  \begin{array}{rcl}
    X &:=& 0Y1X1\\
    Y &:=& 0
  \end{array}
  }{in \kl{normal form}}
\end{align*}
The \kl{register updates} before and after \kl{normalization} act in the same way on
\kl{register valuations}. If an \kl{update} is \kl{copyless} and in \kl{normal form}, then the
combined length of all right-hand sides is at most three times the number of
registers. Therefore, if a semigroup $A$ is finite, then the set of \kl{copyless}
\kl{register updates} in \kl{normal form}, call it $\functoru A$, is also finite.
(However, there are infinitely many copyful register updates even when $A$ is
finite.) This set $\functoru A$ can be equipped with a composition operation
\begin{align*}
    u_1,u_2 \in \functoru A  \quad \mapsto \quad u_1u_2 \in \functoru A,
\end{align*}
which is defined in the same way as applying a register update to a register
valuation, except that we normalize at the end. This composition operation is
associative, and  compatible with applying \kl{register updates} to \kl{register valuations}, in the sense that $(vu_1)u_2 = v(u_1u_2)$ holds for every \kl{valuation} $v$ and all \kl{updates} $u_1$ and $u_2$. Therefore, $A \mapsto \functoru A$ is a \kl{finiteness-preserving} semigroup-to-semigroup functor (with the natural extension to morphisms, where the homomorphism is applied to every semigroup element in a right-hand side). 

The functor $\functoru$ described above is almost but not quite the functor that
will be used in the \kl{transducer semigroup} that we will define to prove the easy
implication in Theorem~\ref{thm:regular-functions}. That functor $\functor$ will
also take into account the \kl{initial register valuation}:
\[ \functor A =  \functoru A \times \myunderbrace{(R \to A)}{\qquad\qquad\qquad\qquad\qquad endowed with the trivial \kl{left zero} semigroup structure} \quad\text{with componentwise multiplication \& action on morphisms} \]
Given $(u,v) \in \functor A$, the \kl{output mechanism} in the \kl{transducer semigroup}
applies the \kl{register update} $u$ to the \kl{register valuation} $v$, and then multiplies together the register values given by the resulting \kl{valuation} $vu$ according to the \kl{final output pattern}.
Using this, we can recognize $f$ via the homomorphism that sends each input letter to:
\begin{itemize}
\item the \kl{register update} that this letter determines (our \sst being \kl{local}) in the first component;
\item the designated \kl{initial register valuation} in the second component.
\end{itemize}





\subsection{Term operations and natural transformations}
We now turn to proving the right-to-left implication in Theorem~\ref{thm:regular-functions}.

Let us begin with some notation. We write $1$ for the singleton semigroup; it is a \emph{terminal object}, that is, it admits a unique homomorphism from every other semigroup $A$, which will be denoted by $! : A \to 1$ (the notation has no connection with the factorial function on numbers). Another construction that will be used heavily in the proof is the \emph{coproduct} of two semigroups $A$ and $B$, which is denoted by $A \oplus B$. This is a semigroup which consists of words over an alphabet that is the disjoint union of $A$ and $B$, restricted to words which are nonempty and alternating in the sense that two consecutive elements cannot belong to the same semigroup. The semigroup operation is defined in the expected way. The coproduct deserves its name due to the following universal property: for every pair of semigroup homomorphisms
\begin{align*}
f : A \to C \qquad \text{and} \qquad g : B \to C
\end{align*}
there is a unique semigroup homomorphism
\[
\begin{tikzcd}
f \text{ or } g : A \oplus B \to C
\end{tikzcd}
\]
that coincides with $f$ (resp.\ $g$) on the subsemigroup of $A \oplus B$ consisting of words with a single letter from $A$ (resp.\ $B$).

Next, we introduce some terminology that will be used in the proof, concerning  polynomial functors, and copyless operations between them. They will be used as the register structure for an \sst in our proof. 

\subsubsection{Polynomial functors}
%The kinds of polynomial functors that we use in the proof are functors from semigroups to sets.
Define a \emph{polynomial functor} to be a functor from the category of semigroups to the category of sets, which is of the form
\begin{align*}
A \quad \mapsto \quad \coprod_{q \in Q} A^{\text{dimension of } q},
\end{align*}
where $Q$ is some possibly infinite set, called the \emph{components}, with each  component having an associated \emph{dimension} in $\set{0,1,\ldots}$. On morphisms, the functor works in the expected way, i.e.~coordinate-wise.  A \emph{finite polynomial functor} is one that has finitely many components. 

\begin{example}\label{ex:coproduct-as-polynomial-functor}
    A crucial property that will be used in our proof is that the functor
    \begin{align*}
    A \mapsto \text{underlying set of}\ \myunderbrace{A \oplus 1}{\text{coproduct with the singleton semigroup}}
    \end{align*}
    is in fact a polynomial functor (but not a finite polynomial functor). Noting that $\oplus$ makes sense as an operation on sets without a semigroup structure, we have more generally that, if $\functor$ is a polynomial functor, then so is
    \begin{align*}
        A \mapsto \myunderbrace{\functor A \oplus 1}{\text{set of alternating sequences between $\functor A$ and $1$}}
    \end{align*}
    
    This is because for every semigroup $A$ there is a bijective correspondence 
    \begin{align}\label{eq:polynomial-representation-of-coproduct}
    A \oplus 1 \quad \simeq \quad \coprod_{q \in 1 \oplus 1} A^{\text{dimension of $q$}},
    \end{align}
    where the dimension of $q$ is defined to be the number of times that the first copy of $1$ appears in $q$. Furthermore, the bijective correspondence in~\eqref{eq:polynomial-representation-of-coproduct} is natural in $A$, and therefore there is a natural bijection between the functor $A \oplus 1$ and some polynomial functor. Also, if two polynomial functors are connected by a natural bijection, then they are the same, up to renaming of the components, and therefore the representation in~\eqref{eq:polynomial-representation-of-coproduct} is unique up to renaming of components. By uniqueness, we will simply speak of $A \oplus 1$ as being a polynomial functor. 
\end{example}




\subsubsection{Copyless natural transformations.}  Among all natural  transformations between polynomial functors, we will be interested mainly in those that are  \emph{copyless}. To define this notion, we will first observe that  every natural transformation between polynomial functors  arises from some syntactic description, and within this syntactic description, the copyless restriction can easily be phrased. 

We begin with \emph{monomial functors}, i.e.~polynomial functors with one component. 
Consider two monomial functors 
\begin{align*}
\functor A = A^k \qquad 
\functorg A = A^\ell \qquad \text{where $k,\ell \in \set{0,1,\ldots}$.}
\end{align*}
What is the possible form of a natural transformation between these functors? One way to create such a natural transformation is to take a  function  of type 
\begin{align*}
\set{1,\ldots,\ell} \to \set{1,\ldots,k}^+,
\end{align*}
which will be called the \emph{syntactic description} of the natural transformation, 
and to define the natural transformation as follows: for a semigroup $A$ the natural transformations gives the function  that inputs $\bar a \in A^k$ and outputs the following tuple $A^\ell$:
\[
\begin{tikzcd}
    [column sep=2.3cm]
\set{1,\ldots,\ell}
\arrow[r, "\text{syntactic description}"]
&
\set{1,\ldots,k}^+ 
\ar[r,"\text{substitute $\bar a$}"]
& 
A^+
\ar[r,"\text{semigroup operation}"]
&
A.
\end{tikzcd}
\]
Every natural transformation between monomial functors arises this way. To see this, the syntactic description is recovered by using the natural transformation for the free semigroup $A=\set{1,\ldots,k}^+$, and applying it for the identity valuation 
\begin{align*}
x \in \set{1,\ldots,k} \quad \mapsto \quad [x] \in \set{1,\ldots,k}^+.
\end{align*}


The advantage of the syntactic description, which is unique, is that it allows us to define the  \emph{copyless restriction}:  (*)   we say that a  syntactic description
\begin{align*}
    \alpha: \set{1,\ldots,\ell} \to \set{1,\ldots,k}^+
    \end{align*}
is copyless if  every letter  from $\set{1,\ldots,k}$ appears in at most one word $\alpha(x)$, and in that word it appears at most once. An equivalent condition can be phrased semantically: (**) if we use the natural transformation in the semigroup $A = \Nat$, then the corresponding function $\Nat^k \to \Nat^\ell$ is non-expansive, i.e.\ the norm of its output is at most the norm of its input, where the norm of a vector is the sum of its coordinates. 

We now define what it means to be copyless for a natural transformation between arbitrary polynomial functors 
\begin{align*}
\functor A = \coprod_{q \in Q} A^{\dim q} \qquad 
\functorg A = \coprod_{p \in P} A^{\dim p},
\end{align*}
which are not necessarily monomial. Such natural transformations also admit syntactic descriptions: for every input component $q$, there is some designated output component $p$, and a natural transformation $A^{\dim q} \to A^{\dim p}$.  The set of possible syntactic descriptions is
\begin{align*}
\prod_{q \in Q} \coprod_{p \in P} \dim p \to (\dim q)^+.
\end{align*}
Again, one can show that all natural transformations arise this way. The natural transformation is called copyless if for every $q$, the corresponding natural transformation between monomial functors is copyless. 


\begin{example}\label{ex:copyless-on-coproducts}
    Consider the functor \enquote{underlying set of $A \oplus 1$} that was discussed in Example~\ref{ex:coproduct-as-polynomial-functor}, and shown to be a polynomial functor. The natural transformation 
    \begin{align*}
    (A \oplus 1)\times (A \oplus 1) \to A \oplus 1
    \end{align*}
    which describes the semigroup operation in the coproduct $A \oplus 1$ is copyless.
\end{example}


\subsubsection{Views}
\label{sec:views}

\newcommand{\combine}{\mathrm{combine}}
\newcommand{\view}{\mathrm{view}}

We now describe a crucial property of the coproduct of semigroups, which is behind the proof of Theorem~\ref{thm:regular-functions}. The idea is that an element of a binary coproduct can be uniquely defined from its views onto the individual coordinates, as defined below. 
For two semigroups $A$ and $B$, define the \emph{$A$-view} and \emph{$B$-view} to be the homomorphisms
\[
    \view_A = \id_A \oplus ~!~ : A \oplus B \to A \oplus 1 \qquad 
    \view_B = ~!~ \oplus \id_B : A \oplus B \to 1 \oplus B
\]
These are natural transformations in $A$ and $B$. The important property of views is that they give complete information about the coproduct, i.e.~if we have all views then we can reconstruct an element of the coproduct.
%furthermore this reconstruction almost corresponds to an isomorphism of polynomial functors, so in particular it is copyless.
This is stated in the following lemma. 

\begin{lemma}
\label{lem:views}
The map $A \oplus B \to (A \oplus 1) \times (1 \oplus B)$ obtained by pairing $\view_A$ and $\view_B$ is \emph{injective} and natural in $A$ and $B$.
\end{lemma}
\begin{proof}
    Straightforward.
\end{proof}
\tito{Fixed. The correct generalization to $n=3$ is certainly something like $(A \oplus 1 \oplus 1) \times (1 \oplus B \oplus 1) \times (1 \oplus 1 \oplus C)$. The statement doesn't include a copyless partial inverse something to avoid having to talk about polynomial functors in several variables.}

This lemma seems to contain the essential property of semigroups that makes the construction work. Our theorem will also be true for other algebraic structures in the lemma is true, such as forest algebras. However, the lemma seems to fail for certain algebraic structures, such as groups, even if we allow $1$ to be replaced by some fixed finite group. Another example where the lemma seems to fail is the monad of weighted sums of words (i.e.~this monad corresponds to weighted automata).
\tito{I wonder if the important thing is not more simply that $A \oplus B$ is a polynomial bifunctor}

\subsubsection{Functorial streaming string transducers}
\label{sec:functorial-sst}
We now describe the last ingredient in our proof, which is a more abstract variant  of streaming string transducers (\sst) that is described in Definition~\ref{def:functorial-sst}. 

Before presenting the abstract definition, we discuss how it differs  from the usual of \sst. The first difference, which is least important, there is some abstract output semigroup $A$ instead of a free semigroup $\Gamma^+$; this generalization is only meant to have cleaner notation. The second, and more important, difference is that, instead of having a fixed number of registers, we allow the register structure to be a finite polynomial functor such as 
\begin{align*}
\functorr A = A^3 + A^2 + A^2 + A + 1.
\end{align*}
The idea is that the register structure already contains the states; with the states corresponding to components in the disjoint union, and with  different states using different numbers of registers. The final difference is that the transducer is allowed to have a look at regular properties of the string on both sides of the head:  when the head is over some position in the input string, then the way in which the registers are is decided based on  some recognizable property of 
\begin{align*}
\myunderbrace{\Sigma^*}{before \\ head } \times 
\myunderbrace{\Sigma}{under \\ head } \times 
\myunderbrace{\Sigma^*}{after \\ head }.
\end{align*}
(As usual, the register update must be copyless.)
By a recognizable property of the above we mean a function that inputs elements of the above set of triples  and outputs elements of some set $X$, which can be be decomposed as 
\[
\begin{tikzcd}
    \Sigma^* \times \Sigma \times \Sigma^* 
    \ar[r,"h \times \id \times h"] 
    &
    M \times \Sigma \times M
\ar[r,"f"]
& 
X
\end{tikzcd}
\] 
for some homomorphism $h$ into a finite monoid and some function $f$. In a sense, this model has two features that replace states: the disjoint unions in the register structure, and the recognizable property. Each of these features alone would be enough, but for our intended application having both features will give a cleaner construction.

Here is the formal definition of our \sst model.

\begin{definition}\label{def:functorial-sst}
    A functorial \sst is defined by:
    \begin{itemize}
    \item a finite  input alphabet $\Sigma$;
    \item a (not necessarily finite) output semigroup $A$;
    \item two finite polynomial functors $\functorr, \functors$ , called the \emph{register functor} and the \emph{update functors}, along with  three copyless natural transformations \begin{align*}
    \vdash\;: 1 \to \functorr A \qquad \delta : \functorr A \times \functors A \to \functorr A \qquad \myunderbrace{\dashv\;: \functorr A \to A}{can be partial}
    \end{align*}
    \tito{why make it partial?}
    \item an \emph{oracle}, which is a recognizable function 
    \begin{align*}
    o : \Sigma^* \times \Sigma \times \Sigma^* \to \functors A.
    \end{align*}
    
    \end{itemize}
\end{definition}

The function computed by a functorial \sst is the partial function of type 
\begin{align*}
\Sigma^+ \to A
\end{align*}
that is defined as follows. Consider some input string in $\Sigma^+$. The machine moves its head along all input positions, and computes for each one a register valuation in $\functorr A$. In the initial register valuation corresponding to $i=0$, when no positions were processed yet, the register valuation  is obtained by applying $\vdash$ to the unique element $1$. For $i > 0$, the $i$-th register valuation is obtained by applying $\delta$ to the pair which consists of the $(i-1)$-st register valuation and the result of applying the oracle to input string with the $i$-th position distinguished. Finally, the output of the \sst is obtained by applying the output term operation $\dashv$ to the last  register valuation.

\begin{lemma}\label{lem:functorial-sst-complete}
    A function of type $\Sigma^+ \to A$ is regular, in the standard sense, if and only if it is computed by a functorial \sst. 
\end{lemma}
\begin{proof}
    A normal \sst with states $Q$ and $k$ registers can be seen as a functorial \sst with a register functor of the form 
    \begin{align*}
    \functorr A = \coprod_{q \in Q} (A + 1)^k.
    \end{align*}
    The oracle function does not look at any properties of the input string (all appropriate information is remembered in the implicit state from the register functor), and it simply outputs all elements of the output semigroup that might potentially be used in an update, with sufficient copies 
    \begin{align*}
    \functors A = A^\ell
    \end{align*}
    to make the function $\delta$ copyless. 

    Consider now the other implication in the lemma, which is the one that we use in this proof. In the case when the register functor is $A^k$ for some $k$,  a functorial \sst is a special case of an \sst with regular lookahead; and regular lookahead can be eliminated~\cite[Lemma 13.6]{bojanczyk_automata_2018}. To accommodate more general polynomial functors as register functors, we observe that the component in a polynomial functor can be stored in the state of an \sst, and the register values can be stored in $A^k$ for sufficiently large $k$. 
\end{proof}


\subparagraph{Factorized output.}
Now, consider some transducer semigroup, with the functor being $\functor$, and fix a string-to-semigroup function $f\colon \Sigma^* \to A$ that decomposes as some homomorphism $h\colon \Sigma^* \to \functor A$ followed by the output function of type $\functor A \to A$. 

For semigroups $A_1,\ldots,A_n$, define the \emph{vectorial output function} to be the 
function% of type 
\begin{align*}
\qquad\functor A_1 \times \cdots \times \functor A_n \longrightarrow A_1 \oplus \cdots \oplus A_n
\end{align*}
that is obtained by composing the three functions described below
\[
\begin{tikzcd}
\functor A_1 \times \cdots \times \functor A_n
\ar[d,"\functor(\text{co-projection}) \times \cdots \times \functor(\text{co-projection})"]
\\
\functor(A_1 \oplus \cdots \oplus A_n)
\times
\cdots
\times 
\functor(A_1 \oplus \cdots \oplus A_n)
\ar[d,"\text{semigroup operation}"]
\\
\functor(A_1 \oplus \cdots \oplus A_n)
\ar[d, "\text{output mechanism for $A_1 \oplus \cdots \oplus A_n$}"]\\ 
A_1 \oplus \cdots \oplus A_n.
\end{tikzcd}
\]

    To illustrate the definitions in this section, we use a running example with the transducer semigroup  from \Cref{ex:duplicator} for the duplicating function. The functor is the identity $\functor A = A$, and the output mechanism is $a \mapsto aa$. The duplicating function on $\set{a,b}^*$ is obtained by composing the identity homomorphism on $\set{a,b}^* = 
\functor(\set{a,b}^*)$
    with the output function.
    Here is an example of  the vectorial output function, applied to $A_1=1$ and $A_2=\set{a,b}^*$:
\begin{align*}
    (1,abbb) \in \functor 1 \times \functor \set{a,b}^* \qquad \mapsto \qquad 
\red{\boxed{1}}\  \boxed{abbb} \ \red{\boxed{1}} \  \boxed{abbb} \in 1 \oplus \set{a,b}^*.
\end{align*}
The  vectorial output function is natural in all of its arguments, which means that  
\[
\begin{tikzcd}
    [column sep=4cm]
\functor A_1 \times \cdots \times \functor A_n
\ar[r,"\text{vectorial output function}"]
\ar[d,"\functor h_1 \times \cdots \times \functor h_n"']
&
A_1 \oplus \cdots \oplus A_n
\ar[d,"h_1 \oplus \cdots \oplus h_n"]
\\
\functor B_1 \times \cdots \times \functor B_n
\ar[r,"\text{vectorial output function}"]
&
B_1 \oplus \cdots \oplus B_n
\end{tikzcd}
\]
commutes
for all semigroup homomorphisms $h_1,\ldots,h_n$. This is because each of the three steps in the definition of the vectorial output function is itself a natural transformation, and natural transformations compose.  Naturality of the first two steps is easy to check, while for the last step we use the assumption that the (non-vectorial) output function is natural.
% In other words, the factorized output function is a natural 
% transformation of type 
% \[\begin{tikzcd}
%     [column sep=1cm]
%     {\text{Semigroups}^n} && {\text{Sets}}
%     \arrow[""{name=0, anchor=center}, "{(A_1,\ldots,A_n)} \mapsto \text{underlying set of } \functor A_1 \times \cdots \times \functor A_n", curve={height=-18pt}, from=1-1, to=1-3]
%     \arrow[""{name=1, anchor=center, inner sep=0}, "{(A_1,\ldots,A_n)} \mapsto \text{underlying set of } A_1 \oplus \cdots \oplus A_n"', curve={height=18pt}, from=1-1, to=1-3]
%     \arrow[ shorten <=5pt, shorten >=5pt, Rightarrow, from=0, to=1]
% \end{tikzcd}\]



% By abuse of notation, we allow some -- but not all -- of the arguments in the factorized output to be empty; in this case the empty arguments are ignored, but the output type is still a co-product of $n$ semigroups. For example, if the input to the factorized output function is 
% \begin{align*}
% (1, \varepsilon) \in 1 \oplus \set{a,b}^+
% \end{align*}
% then the factorized  output is  
% \begin{align*}
% 1 1     \in 1 \oplus \set{a,b}^+.
% \end{align*}

Let us return to our function $f = \outfun_A \circ h$ recognized by our transducer semigroup $(\functor,\outfun)$.
%Using the vectorial output mechanism, we will be able to track the origins in the output of the function $f$, with respect to some partition of the input string into several nonempty parts.
For strings $w_1,\ldots,w_n \in \Sigma^*$, define the corresponding \emph{factorized output} to be the result of first applying the semigroup homomorphism $h : \Sigma^* \to \functor A$ to all the strings, and then applying the vectorial output function; we denote it by 
\begin{align*}
\tuple{w_1| \cdots | w_n} \in \myunderbrace{A \oplus \cdots \oplus A}{$n$ times},
\end{align*}
Here is the factorized output illustrated in our running example (we use colours to distinguish which of the three parts of the input is used):
\begin{align*}
        \tuple{\red{abb} | \varepsilon | \blue{bbaba}} =  
        \red{\boxed{abb}} \ 
        \boxed{\varepsilon} \ 
        \blue{\boxed{bbaba}} \ 
        \red{\boxed{abb}} \ 
        \boxed{\varepsilon} \
        \blue{\boxed{bbaba}} \in \red{\set{a,b}^*} \oplus \set{a,b}^*  \oplus \blue{\set{a,b}^*}.
\end{align*}
As we can see above, when the output semigroup is a free monoid, the factorized output morally tells us \enquote{which part of the output string comes from which part in the input string}.
\begin{remark}
This is similar to the idea of \emph{origin semantics}~\cite{bojanczykTransducersOriginInformation2014} of regular functions (see also~\cite[Section~5]{MuschollPuppis}). Indeed, our definition of factorized output is inspired by a similar tool of the same name that appears in the study of origin semantics~\cite[Section~2]{bojanczykTransducersOriginInformation2014}.
\end{remark}

We also use a similar notation but with some input strings underlined, e.g.~the input could be $\tuple{\red{\underline{abb}} | \varepsilon | \blue{bbaba}}$ with an underline for the first red part. In the underlined case, before applying the vectorial output function, we apply $h$ to the non-underlined strings and
% \[
% \begin{tikzcd}
% \Sigma^*
% \ar[r,"h"]
% &
% \functor A 
% \ar[r,"\functor !"]
% &
% \functor 1.
% \end{tikzcd}
% \]
$(\functor! \circ h) \colon \Sigma^* \to \functor1$
to the underlined strings. In our running example, we have
    \begin{align*}
        \tuple{\red{\underline{abb}} | \varepsilon | \blue{bbaba}} =  
        \red{\boxed{1}} \ 
        \boxed{\varepsilon} \
        \blue{\boxed{bbaba}} \ 
        \red{\boxed{1}} \ 
        \boxed{\varepsilon} \
        \blue{\boxed{bbaba}}.
        \end{align*}

\subparagraph*{Proof of $(\ref{it:trans-semig-regular}) \Rightarrow (\ref{it:regular})$ in \Cref{thm:regular-functions}.}
We have now collected all necessary ingredients to prove this hard direction of
the equivalence. Therefore, our goal is now to show that the function $f\colon
\Sigma^* \to A$ that we have previously fixed is computed by some functorial
streaming string transducer as in Definition~\ref{def:functorial-sst}.
We will see that this can be done merely assuming that \emph{$\functor 1$ is finite} -- a particular instance of the assumption in (\ref{it:trans-semig-regular}) that $\functor$ is finiteness-preserving.

The idea is that we want the \sst to maintain the following invariant: \emph{after processing the first $i$ letters in an input string $a_1 \cdots a_n$, the register valuation is equal to the factorized output $\tuple{a_1 \cdots a_i | \underline{a_{i+1} \cdots a_n}}$}. This way, after processing all input letters, the last valuation $\tuple{a_1 \cdots a_n | \underline{\varepsilon}}$ is very close to the output $f(a_1\dots a_n) = \tuple{a_1 \dots a_n} \in A = \text{1-ary coproduct}$.

The naive choice for the register functor is then $\functorr' : A \mapsto A \oplus 1$, since $\tuple{w|\underline{v}} \in A\oplus1$ for all $w,v\in\Sigma^*$ by definition. However, while $\functorr'$ can be seen as a polynomial semigroup-to-set functor, whose components are indexed by $1 \oplus 1$ (cf.~\Cref{rem:coproduct-as-polynomial-functor}), it is not \emph{finite} polynomial (the set $1\oplus1$ is infinite). That said, we have by naturality (cf.~appendix) that:
\begin{claim}
  The component index for $\tuple{w|\underline{v}} \in \functorr'A$ is $\tuple{\underline{w}|\underline{v}} \in 1 \oplus 1$.
\end{claim}
This index is determined by definition by the values of $(\functor! \circ h) \colon \Sigma^* \to \functor1$ on $w$ and $v$. \emph{Since $\functor1$ is finite,} the $\tuple{w|\underline{v}}$ for $w,v$ ranging over $\Sigma^*$ live in finitely many components. We take our register functor  $\functorr A \subset \functorr' A$ to be the finite polynomial functor consisting of these \enquote{useful} components, plus the unique component that does not use $A$ (it will serve as a \enquote{null value}).

To design the register updates, the key is the following lemma. It shall be proved later using the machinery of views on coproducts that we have introduced for this very purpose.
\begin{lemma}\label{lem:compute-next-configuration}
  There is a {copyless} natural function
  \[
    \delta\colon (A \oplus 1) \times (1 \oplus A \oplus 1) \to A \oplus 1
    \qquad\text{such that}\qquad
    \tuple{wa|\underline v} = \delta(\tuple{w|\underline {av}},\tuple{\underline w | a|\underline v})
  \]
  for every pair of strings $w,v \in \Sigma^*$ and every letter $a \in \Sigma$.
\end{lemma}
This leads us to use the update functor $\functoru : A \mapsto 1 \oplus A \oplus 1$ and to define the application of updates to registers, of type $\functorr A \times (1 \oplus A \oplus 1) \to \functorr A$, to be $\delta$ followed by the map $A \oplus 1 \to \functorr A$ which sends the components in $\functorr A$ to themselves and the rest to the \enquote{null value}.
As an immediate consequence of the lemma, the desired invariant holds if we use
\begin{itemize}
  \item the initial function $w \mapsto \tuple{\varepsilon|\underline{w}}$,
  \item and the update oracle $a_1 \dots a_n \mapsto \tuple{\underline\varepsilon|a_1|\underline{a_2 \dots a_n}} \dots
  \tuple{\underline{a_1 \dots a_{n-1}}|a_n|\underline\varepsilon}$.
\end{itemize}
To fit \Cref{def:functorial-sst}, we have to check that the initial function is recognizable and that the update oracle is a rational function; by definition, the latter amounts to saying that for any $a\in\Sigma$, the function $(w,v) \in (\Sigma^*)^2 \mapsto \tuple{\underline{w}|a|\underline{v}}$ is recognizable. According to the definition of factorized output, the initial function factors through the semigroup homomorphism $\functor! \circ h$, \emph{whose codomain $\functor1$ is finite,} therefore the initial function is recognizable. The other recognizability condition holds for a similar reason.

What remains is to define a way to turn the last register valuation
$\tuple{w|\underline\varepsilon}$ into the output $\tuple{w} = f(w)$ for any input string
$w$, and to prove \Cref{lem:compute-next-configuration}. For both of these
goals, it is useful to consider a sort of left inverse to the \emph{deconstruction} into views and shape of \Cref{lem:views} -- that is, a \emph{reconstruction} function of type
\[ (1 \oplus A_1) \times \cdots \times (1 \oplus A_n) \times (1 \oplus \cdots
  \oplus 1) \longrightarrow (A_1 \oplus \cdots \oplus A_n) + \myunderbrace{1}{default value when the input is not in the image of the deconstruction\qquad\qquad\qquad\qquad\qquad\qquad} \]
such that deconstruction followed by reconstruction maps every element of $A_1 \oplus \cdots \oplus A_n$ to itself. The injectivity proof in \Cref{lem:views} actually provides such a reconstruction function.
And in the case of a binary coproduct $A \oplus A$, any of the two views determines the shape, so we have a binary reconstruction of type $(A\oplus1)\times(1\oplus A) \to (A\oplus A)+1$.

To finish building our functorial \sst, let our final functor be $\functork : A \mapsto (1 \oplus A)\times A$, our final value function be $w\in\Sigma^* \mapsto (\tuple{\underline{w}|\varepsilon},\;\text{some arbitrary fixed value in}\ A)$ -- it is recognizable because $\functor1$ is finite, again -- and our final output function $\functorr A \times \functork A \to A$ be defined as follows: apply binary reconstruction to get some value in $((A\oplus A)+1)\times A$; if the left component is in $A\oplus A$, merge it to produce an output in $A$; otherwise, return the right component. One can check that this output function applied to $\tuple{w|\underline\varepsilon}$ and $(\tuple{\underline{w}|\varepsilon},\;\text{anything})$ returns $\tuple{w}$ -- this is similar to \Cref{claim:merge-factorized-output} below -- so that our functorial \sst indeed computes $f$.

This being done, let us discharge our only remaining subgoal.

\begin{proof}[Proof of \Cref{lem:compute-next-configuration}]
    We use the following claim, which is proved using naturality of the output mechanism (as detailed in the appendix).
    \begin{claim}\label{claim:merge-factorized-output}
        $\tuple{wa|\underline v}$ is obtained from $\tuple{w|a|\underline v}$ by merging the first two parts.
    \end{claim}
    % \begin{proof}[Proof sketch]
    %     % The following diagram commutes 
    %     % \[
    %     % \begin{tikzcd}
    %     %     [column sep=3cm]
    %     % \Sigma^* \times \Sigma \times \Sigma^* 
    %     % \ar[r,"{(w,a,v) \mapsto \tuple{w|a|\underline v}}"]
    %     % \ar[d,"{(w,a,v) \mapsto (wa,v)}"']
    %     % &
    %     % A \oplus A \oplus 1
    %     % \ar[d,"\text{merge first two coordinates}"]
    %     % \\
    %     % \Sigma^+ \times \Sigma^* 
    %     % \ar[r,"{(wa,v) \mapsto \tuple{wa|\underline v}}"']
    %     % &
    %     % A  \oplus 1
    %     % \end{tikzcd}
    %     % \]
    %  Using naturality of the output mechanism, and $\outfun$ and the fact that  both $h$ and merging the first two \enquote{copies of $A$} in $A \oplus A \oplus 1$ (by copairing coprojections) are homomorphisms.
    % \end{proof}

    Since merging the first two parts is a copyless natural function,  the above
    claim shows that  the factorized output $\tuple{wa|\underline v}$ is
    obtained from $\tuple{w|a|\underline v}$ by a copyless natural function.
    In turn, $\tuple{w|a|\underline v}$ can be obtained by applying the reconstruction function to the following four items (the equalities below are proved analogously to Claim~\ref{claim:merge-factorized-output}):
     \begin{enumerate}
        \item \label{it:first-view}First view of $\tuple{w|a|\underline v}$, which is equal to $\tuple{w|\underline {av}}$ -- this is the first argument of the function $\delta$ that we want to define.
        \item  \label{it:second-view} Second view of $\tuple{w|a|\underline v}$, which is obtained by merging the first and third parts in $\tuple{\underline w|a|\underline v}$.
        \item \label{it:third-view}   Third view of $\tuple{w|a|\underline v}$, which is equal to $\tuple{\underline {wa}|\underline{v}}$.
        \item   \label{it:shape} Shape of~$\tuple{w|a|\underline v}$, which is equal to $\tuple{\underline w|\underline a| \underline v}$.
     \end{enumerate}
     (Strictly speaking, this reconstruction returns an element of $(1 \oplus A \oplus 1)+1$, and we should postcompose it by a map $(1 \oplus A \oplus 1)+1 \to 1 \oplus A \oplus 1$ that sends the right summand $1$ to some default value that does not use $A$.)
     
     To complete the proof of the lemma, it remains to justify that the last three items in the above enumeration can be collectively obtained from the second argument of $\delta$, namely $\tuple{\underline w|a|\underline v}$, by applying some copyless natural function. Each item is obtained separately by applying a natural function. Furthermore, the second item is obtained in a copyless way, while the last two items do not use $A$ at all, and therefore they are obtained in a copyless way for trivial reasons, even when combined with the second item.
\end{proof}


% \begin{lemma}
%     There is a letter-to-letter rational function 
%     \begin{align*}
%     \rho : \Sigma^* \to (1 \oplus A \oplus 1)^*
%     \end{align*}
%     such that for every input string $a_1 \cdots a_n$, the $i$-th letter of the output is 
%     \begin{align*}
%         \tuple{\underline{a_1 \cdots a_{i-1}}|a_i| \underline{ a_{i+1} \cdots a_n}} \in 1 \oplus A \oplus 1.
%     \end{align*}
% \end{lemma}
% \begin{proof}
%     For an interval $\set{i,\ldots,j}\subseteq \set{1,\ldots,n}$, we define 
%     The $i$-th letter produced by the rational function is 
%     \begin{align*}
%         \tuple{\underline{a_1 \cdots a_{i-1}}|a_i| \underline{ a_{i+1} \cdots a_n}} \in 1 \oplus A \oplus 1.
%     \end{align*}
% We begin by showing that these letters can indeed be produced by a letter-to-letter rational function. To see this, we observe that the above depends only on the letter $a_i$, as well as the images of the strings $a_1 \cdots a_{i-1}$ and $a_{i+1} \cdots a_n$ under the semigroup homomorphism
% \begin{align*}
% \functor ! : \Sigma^* \to \functor 1.
% \end{align*}
% Since the target of this homomorphism is a finite semigroup, by the assumption that the functor is finiteness preserving, it follows that the 
% \end{proof}
