

\section{The regular functions}
\label{sec:reg-char}
The two constructions in Theorems~\ref{thm:all-functions} and~\ref{thm:reco-reflecting-functions} were rather straightforward, and amounted to little more than symbol pushing. In this section, we present a characterization of the regular functions, i.e.~string-to-string functions recognized by streaming string transducers and their equivalent models. In the characterization, we require that the functor is finiteness preserving, i.e.~it maps finite semigroups to finite semigroups. It turns out that the naturality of the output mechanism interacts with the condition that 
the functor is finiteness preserving; resulting in a strong restriction on the expressive power. To see this interaction, consider the following example

\begin{myexample}
    Consider the powerset functor $\powerset A$ from Example~\ref{ex:functors}. This is a finiteness preserving functor, since the powerset of a finite set is also finite. One could imagine that using the powerset functor we could construct some transducer semigroup which recognizes functions that are not regular, e.g.~because they have exponential growth. It turns out that this is impossible, because there is no possible output mechanism, i.e.~no natural transformation 
    \[
    \begin{tikzcd}
    \powerset A 
    \ar[r,"\outfun_A"]
    &
    A.
    \end{tikzcd}
    \]
    There reason why there is no such natural transformation is that it require some kind of choice, which would contradict naturality. More formally, let $A$ be a semigroup with two elements, with a trivial semigroup operation defined by $xy=x$. The output mechanism needs to choose some element $a \in A$ to the full set $A \in \powerset A$. However, none of the two choices is right, because if we take any semigroup homomorphism $f : A \to A$ such that $f(A)=A$, then  naturality of the output mechanism implies that $a=f(a)$. If $f$ is the homomorphism that swaps the two elements, then we get a contradiction.
\end{myexample}


We now state the main theorem of this paper. Unlike the previous characterizations, the statement is stated in terms of functions between free semigroups $\Sigma^+$ and $\Gamma^+$, because the models defining regular functions are defined for string-to-string functions, and not transformations of abstract semigroups. (Some of the models, such as streaming string transducers or two-way automata, easily make sense when the output is an abstract semigroup, but the string structure of the input semigroup seems to be essential for all the models.) 

% Among the functors described in Example~\ref{ex:functors},  ``reverse'' and ``powerset''  are finiteness preserving, in the sense that if they are applied to a finite semigroup, then the result is also a finite semigroup. The ``tuple'' functor $A^n$ is finiteness preserving if and only if the exponent $n$ is finite. The ``list'' functor $A^+$ is not finiteness preserving. 


\begin{theorem}\label{thm:regular-functions}
    Let  $\Sigma$ and $\Gamma$  be finite alphabets. The following conditions are equivalent for a function $f : \Sigma^+ \to \Gamma^+$, which is not necessarily a semigroup homomorphism:
    \begin{enumerate}
        \item \label{it:regular} $f$ is  a regular function, as defined in  Section~\ref{sec:easy}. 
        \item \label{it:trans-semig-regular}$f$ is recognized by a transducer semigroup  such that for every finite semigroup $A$, the semigroup $\functor A$ is also finite. 
    \end{enumerate}
\end{theorem}


Before proceeding with the proof, we comment on the role of empty strings. Regular functions are usually defined for possibly empty strings, i.e.~functions of type $\Sigma^* \to \Gamma^*$. We use nonempty strings, because it will be more convenient to work with semigroups, and the free semigroup construction produces nonempty strings. To extend the construction to functions that can output possibly empty strings, while still working with semigroups, we could modify the type of the output mechanism to be 
\begin{align*}
\outfun_A : \functor A \to \myunderbrace{A + 1}{disjoint union of underlying set of $A$ \\ with one extra element representing the empty word},
\end{align*}
under this modification the same proof as presented below would give us exactly the regular functions with possibly empty outputs. The empty string as an input is less important, since we can always  extend  the source code of a transducer that inputs nonempty strings with an extra line which says how to handle an empty input string. 


The left-to-right implication is relatively straightforward, and presented in Section~\ref{sec:easy}, together with the definition of streaming string transducers. The main part of the proof  is devoted to the right-to-left implication. The proof is presented in a way which, if sometimes slightly verbose, makes it easier to see how it can be adapted to other algebraic structures instead of semigroups (such as forest algebras).

\subsection{From a regular function to a transducer semigroup}
\label{sec:easy}

(TODO fill in)

\tito{Note: state-dependent memory SSTs are needed to put semigroup elements (not just monoid elements) in registers}


\subsection{From a transducer semigroup to a regular function}
\label{sec:hard}
We now turn to the difficult implication $(\ref{it:trans-semig-regular}) \Rightarrow (\ref{it:regular})$ in Theorem~\ref{thm:regular-functions}. 
% The proof is presented in a way which, if sometimes slightly verbose, makes it easier to see how it can be adapted to other algebraic structures instead of semigroups (such as forest algebras).

\subparagraph*{Functorial streaming string transducers.}
\label{sec:abstract-sst} 
The assumption of the implication uses an abstract model (transducer semigroups), while the conclusion uses a concrete operational model (streaming string transducers). To bridge the gap, we use an intermediate model, similar to streaming string transducers, but a bit more abstract. The abstraction arises by using polynomial functors instead of registers, as described below. 

Define a \emph{polynomial functor} to be a semigroup-to-set functor of the form
\begin{align*}
A \quad \mapsto \quad \coprod_{q \in Q} A^{\text{dimension of } q},
\end{align*}
where $Q$ is some possibly infinite set, called the \emph{components}, with each  component having an associated \emph{dimension} in $\set{0,1,\ldots}$. The symbol $\coprod$ stands for disjoint union of sets. This functor does not take into account the semigroup structure of the input semigroup, since the output is seen only as a set.
On morphisms, the functor works in the expected way, i.e.~coordinate-wise.  

A \emph{finite polynomial functor} is one with finitely many components. For example, $A \mapsto A^2 + A^2 + A$ is a finite polynomial functor. 
A finite polynomial functor can be seen as a mild generalization of the construction which maps a semigroup $A$ to the set $A^R$ of register valuations for some fixed set $R$ of register names.  In the generalization, we allow a variable number of registers, depending on some finite information (the component). 

Having defined a more abstract notion of ``register valuations'', namely finite polynomial functors, we now define a more abstract notion of ``register updates''.  The first condition for such updates is that they do not look inside the register contents; this condition is captured by naturality as described in the following definition. 



\begin{definition}[Natural functions]\label{def:natural-functions}
    Let $\functor$  and $\functorg$ be polynomial functors, let $A$ be a semigroup. A function\footnote{This function is not necessarily a semigroup homomorphism. In fact, it would not even make sense call it a homomorphism, since the functors $\functor$ and $\functorg$ produce sets and not semigroups.}  $f : \functor A \to \functorg A$ is called \emph{natural} if it can be extended to natural transformation of type $\functor \Rightarrow \functorg$. This means that there is a family of functions, with one function
    $
    f_A : \functor A \to \functorg A
    $
    for each semigroup $A$, such  that $f=f_A$, and the the  diagram
    \[
    \begin{tikzcd}
    \functor A 
    \ar[r,"\functor h"]
    \ar[d,"f_A"']
    & 
    \functor B 
    \ar[d,"f_B"]
    \\
    \functorg A 
    \ar[r,"h"]
    &
    \functorg B
    \end{tikzcd}
    \]
    commutes for every semigroup homomorphism $h$.
\end{definition}

% Intuitively speaking, naturality says that the function is not allowed to look inside the semigroup elements that are stored in a polynomial functor $\functor A$, but it is allowed to manipulate them using the semigroup operation.

\begin{example}
    Consider the polynomial functors 
    \begin{align*}
    \functor A = A^* = 1 + A^1 + A^2 + \cdots  \qquad \functorg A = A + 1,     \end{align*} 
    where $1$ represents the singleton set $A^0$.
An example of a natural transformation between these two functors is the function which maps a nonempty list in $A^*$ to the product of its elements, and which maps the empty list to the unique element of $1$. A non-example is the function that maps a list $[a_1,\ldots,a_n] \in A^*$ to the leftmost element $a_i$ that is an idempotent in the semigroup, and returns $1$ if such an element does not exist. The reason why the non-example is not natural is that a semigroup homomorphism can map a non-idempotent to an idempotent.
\end{example}
% The naturality condition is an abstraction of the condition (a) that was mentioned before, namely that register contents are not inspected. For example, as we will see below, a natural transformations from the functor $\functor A = A^2$ to itself is $(a,b) \mapsto (abbaa,bbba)$, while 
% \begin{align*}
% (a,b) \mapsto \begin{cases}
%     (1,1) & \text{if the semigroup $A$ has an identity element $1$, and $a=1$}\\
%     (a,b) & \text{otherwise.}
% \end{cases}
% \end{align*}
% is not a natural transformation.


Apart from naturality, we will want our register updates to be copyless. 

\begin{definition}[Copyless natural function] \label{def:copyless} A natural function $f : \functor A \to \functorg A$ is  called \emph{copyless} if it arises from some natural transformation with the following property:  when instantiated to the semigroup\footnote{The choice of the semigroup $(\Nat,+)$ in the \Cref{def:copyless} is not particularly important. For example, the same notion of copylessness would arise if instead of $(\Nat,+)$, we used the semigroup $\set{0,1}$ with addition up to threshold $1$ (i.e.~the only way to get zero is to add two zeros). In the appendix, we present a more syntactic characterization of copyless natural transformations, which will be used later on when proving equivalence with streaming string transducers. 
    } $(\Nat,+)$, the corresponding function of type $\functor \Nat \to \functorg \Nat$ 
    does not increase the norm. Here, the norm of an element in a polynomial functor $\functor \Nat$ or $\functorg \Nat$ is defined to be the sum of numbers that appear in it.
\end{definition}





Having defined functions that are natural and copyless, we now describe the more abstract model of streaming string transducers that is be used in our proof. The main difference is that instead of register valuations and updates given by some finite set of register names, we have two abstract finite polynomial functors, together with an explicitly given application function. Another minor difference is that we allow the model to define partial functions; this will be useful in the proof.


\begin{definition}\label{def:functorial-sst}
    The syntax of a functorial streaming string transducer is given by:
    \begin{itemize}
        \item A finite \emph{input alphabet} $\Sigma$ and an \emph{output semigroup $A$}.
    \item Two finite polynomial functors $\functorr$ and $\functoru$, called the \emph{register} and \emph{update} functors, together with a function of type $\functorr A \times \functoru A \to \functorr A$, called \emph{appliction}, which is natural and copyless.
    \item A distinguished \emph{initial register valuation}  in $\functorr A$.
    \item A \emph{final function} of type $\functorr A \to A + 1$, which is natural and copyless.
    \item An \emph{update oracle}, which is a rational function of type $\Sigma^* \to (\functoru A)^*$.
    \end{itemize}
\end{definition}

The semantics of the transducer is a partial function of type 
$\Sigma^* \to A$ defined as follows. As in Definition~\ref{def:usual-sst}, for every input string we use  the initial register valuation, the application function and the update oracle to define a sequence of register valuations in $\functor A$. We then apply the final function to the last register valuation, yielding a result in $A+1$.  If this result is in the $1$ part, then the output of the transducer is undefined, otherwise the output of the transducer is the semigroup element stored in the $A$ part. We will care about transducers that compute total functions, which corresponds to the property  that for every input string, the last register valuation is in the $A$ part of $A+1$.

\begin{lemma}\label{lem:functorial-sst-complete}
    The models defined in Definitions~\ref{def:usual-sst} and~\ref{def:functorial-sst} define the same (total) string-to-semigroup functions.
\end{lemma}



\subparagraph*{Coproducts and views.}
\label{sec:coproducts-and-views}

Apart from the more abstract transducer model from Definition~\ref{def:functorial-sst}, the other ingredient used  in the proof of the hard implication in Theorem~\ref{thm:regular-functions} will be coproducts of semigroups, and some basic operations on them, as described in this section.

We write $1$ for the semigroup that has one element. This semigroup is unique up to isomorphism and it is a \emph{terminal object} in the category of semigroups, which means that it admits a unique homomorphism from every other semigroup $A$. This unique homomorphism will be denoted by $!\colon A \to 1$. It has no connection with the factorial function on numbers. 

The \emph{coproduct}  of two semigroups $A$ and $B$, denoted by $A \oplus B$, is the semigroup whose elements are nonempty words over an alphabet that is the disjoint union of $A$ and $B$, restricted to words that are \emph{alternating} in the sense that two consecutive letters cannot belong to the same semigroup. The semigroup operation is defined in the expected way. We draw elements of a coproduct using coloured boxes, with the following picture showing the product operation in the coproduct of two copies, \red{red} and \blue{blue}, of the semigroup $\set{a,b}^+$:
\begin{align*}
    (\red{\boxed{aba}} \cdot 
    \blue{\boxed{b}} \cdot 
    \red{\boxed{b}} \cdot 
    \blue{\boxed{aa}}) \cdot 
    (
        \blue{\boxed{abba}} \cdot 
        \red{\boxed{aa}} \cdot 
        \red{\boxed{bb}}
    )
    = 
\red{\boxed{aba}} \cdot 
    \blue{\boxed{b}} \cdot 
    \red{\boxed{b}} \cdot 
    \blue{\boxed{aaabba}} \cdot 
        \red{\boxed{aa}} \cdot 
        \red{\boxed{bb}}.
\end{align*}
A coproduct can involve more than two semigroups; in the pictures this would correspond to more colours, subject to the condition that  consecutive boxes have different colours.
\begin{remark}
  The copyless register updates $u : R \to (A + R)^+$ of ordinary \sst{}s that are in normal form according to the definition of \Cref{sec:easy} can be seen as maps
  \[ R \to A \oplus \bigoplus_{X\in R} \{X\}^+ \]
\end{remark}
% The name \emph{coproduct} is used because of the following universal property: if
% \begin{align*}
% f : A \to C \qquad \text{and} \qquad g : B \to C
% \end{align*}
% are two semigroup homomorphisms, then
% there is a unique homomorphism 
% \[
% \begin{tikzcd}
% f \text{ or } g : A \oplus B \to C
% \end{tikzcd}
% \]
% that coincides with $f$ (resp.~$g$) on the subsemigroup of $A \oplus B$ consisting of words with a single letter from $A$ (resp.~$B$).






\begin{remark}[used in our proof]
    \label{rem:coproduct-as-polynomial-functor}
    %The polynomial functors that we  use in our proof will arise using coproducts with the singleton semigroup $1$. 
    Consider the semigroup-to-set functor $A \mapsto A \oplus 1$, which maps a semigroup to the underlying set of its coproduct with the singleton semigroup. Although not defined as a polynomial functor, this functor is isomorphic to one. This is because for every semigroup $A$ there is a bijective correspondence between the sets
    \begin{align*}%\label{eq:polynomial-representation-of-coproduct}
    A \oplus 1 \quad \text{and} \quad \coprod_{q \in 1 \oplus 1} A^{\text{dimension of $q$}},
    \end{align*}
    where the dimension of $q$ is defined to be the number of times that the first copy of $1$ appears in $q$. Furthermore, this bijection is natural, and thus we can speak of $A \oplus 1$ as being a polynomial functor. This remark applies to similar constructions, which involve a coproduct of several copies of $A$ with several copies of $1$, such as $A \oplus A \oplus A \oplus 1 \oplus 1$.
    %Mikolaj: I eliminated parts of this discussion, since they can be inserted (in the reader's head) at the places where they are used
    % bijection in~\eqref{eq:polynomial-representation-of-coproduct} is natural, and therefore there is a natural bijection between the functor $(-) \oplus 1$ and some polynomial functor. Also, if two polynomial functors are connected by a natural bijection, then they are the same, up to renaming of the components, and therefore the representation in~\eqref{eq:polynomial-representation-of-coproduct} is unique up to renaming of components. By uniqueness, we will simply speak of $A \oplus 1$ as being a polynomial functor.  In a similar way, functors such as $A \mapsto A \oplus 1 \oplus A$ are also polynomial.
    % Noting that $\oplus$ makes sense as an operation on sets without a semigroup structure (it builds a set of nonempty alternating sequences), we have more generally that, if $\functor$ is a polynomial functor, then so is $A \mapsto \functor A \oplus 1$. 
\end{remark}







% \begin{example}\label{ex:copyless-on-coproducts}
%     Consider the functor \enquote{$A\mapsto$ underlying set of $A \oplus 1$} from Example~\ref{ex:coproduct-as-polynomial-functor}, which was shown to be a polynomial functor. The natural transformation in $A$
%     \begin{align*}
%     (A \oplus 1)\times (A \oplus 1) \longrightarrow A \oplus 1
%     \end{align*}
%     which describes the semigroup operation in the coproduct $A \oplus 1$ is copyless.
% \end{example}


% \subsubsection{Views}
% \label{sec:views}

The crucial property of semigroups that will be used in our proof is described in Lemma~\ref{lem:views} below, which says that an element of a coproduct can be reconstructed based on certain partial information. This information is described  using the following operations.

\begin{enumerate}
    \item \textbf{Merging}. Consider a coproduct $A_1 \oplus \cdots \oplus A_n$, such that the same semigroup $A$ appears on all coordinates from a subset $I \subseteq \set{1,\ldots,n}$, and possibly on other coordinates as well. Define \emph{merging the parts from $I$} to be the function of type 
    \begin{align*}
        A_1 \oplus \cdots \oplus A_n \to  A \oplus \bigoplus_{i \not \in I} A_i
        \end{align*}
    that is defined in the expected way, and explained in the following picture. In the picture, merging is applied to  a coproduct of three copies of the semigroup $\set{a,b}^+$, indicated using colours \red{red}, black and \blue{blue}, and the merged coordinates are \red{red} and \blue{blue}:
        \begin{align*}
        \blue{\boxed{aba}} \cdot 
        \red{\boxed{b}} \cdot 
        \boxed{aa} \cdot 
        \red{\boxed{b}} \cdot 
        \blue{\boxed{aa}} \cdot 
        \red{\boxed{abba}} \cdot 
        \boxed{b}
        \quad \mapsto \quad  
        \myunderbrace{
            \violet{\boxed{abab}} \cdot 
        \boxed{aa} \cdot 
        \violet{\boxed{baaabba}} \cdot 
        \boxed{b}
        }{the merge of \red{red} and \blue{blue} is drawn in \violet{violet}}.\end{align*}    
        \item \textbf {Shape.}  Define the \emph{shape operation} to be the function of type 
        \begin{align*}
        A_1 \oplus \cdots \oplus A_n \to 1 \oplus \cdots \oplus 1
        \end{align*}
        obtained by applying $!$ on every coordinate. The shape says how many alternating blocks there are, and which semigroups they come from, as explained in the following picture:
        \begin{align*}
            \blue{\boxed{aba}} \cdot 
            \red{\boxed{b}} \cdot 
            \boxed{aa} \cdot 
            \red{\boxed{b}} \cdot 
            \blue{\boxed{aa}} \cdot 
            \red{\boxed{abba}} \cdot 
            \boxed{b}
            \quad \mapsto \quad  
            \blue{\boxed{1}} \cdot 
            \red{\boxed{1}} \cdot 
            \boxed{1} \cdot 
            \red{\boxed{1}} \cdot 
            \blue{\boxed{1}} \cdot 
            \red{\boxed{1}} \cdot 
            \boxed{1}.
        \end{align*}
        \item \textbf{Views.} The final operation is the $i$-th view 
        \begin{align*}
        A_1 \oplus \cdots \oplus A_n \to 1 \oplus A_i.
        \end{align*}
        This operation applies $!$ to all coordinates other than $i$, and then it merges all those coordinates. Here is a picture, in which we take the view of the \blue{blue} coordinate:
        \begin{align*}
            \blue{\boxed{aba}} \cdot 
            \red{\boxed{b}} \cdot 
            \boxed{aa} \cdot 
            \red{\boxed{b}} \cdot 
            \blue{\boxed{aa}} \cdot 
            \red{\boxed{abba}} \cdot 
            \boxed{b}
            \quad \mapsto \quad  
            \blue{\boxed{aba}} \cdot 
            {\boxed{1}} \cdot 
            \blue{\boxed{aa}} \cdot 
            \boxed{1}.
        \end{align*}
        
\end{enumerate}


% Another crucial property of the coproduct of semigroups is that a coproduct can be reconstructed from certain partial information, as described below. 
% Consider a coproduct $A_1 \cdots A_n$.


The key observation is that an element of a coproduct can be reconstructed from its shape and views, as stated in the following lemma. 

\begin{proposition}
\label{prop:views} Let $A_1,\ldots,A_n$ be semigroups. The \emph{deconstruction} function of type
\begin{align*}
A_1 \oplus \cdots \oplus A_n \longrightarrow (1 \oplus A_1) \times \cdots \times (1 \oplus A_n) \times (1 \oplus \cdots \oplus 1),
\end{align*}
which is obtained by combining the views for all $i \in \set{1,\ldots,n}$ and the shape, is injective. 
\end{proposition}
\begin{proof}
    The input can be reconstructed from the output as follows.
    Start with the shape, and replace the entries from $1$ with the semigroup elements used in the views.
\end{proof}
This lemma seems to contain the essential property of semigroups that makes the construction work. We expect our theorem to also be true for other algebraic structures for which the lemma is true; however, the lemma seems to fail for certain algebraic structures. Concrete examples will be discussed in the conclusion (\Cref{sec:conclusion}).

Like any injective function, deconstruction admits a partial left inverse: a function
    \[ \qquad (1 \oplus A_1) \times \cdots \times (1 \oplus A_n) \times (1 \oplus \cdots
  \oplus 1) \longrightarrow (A_1 \oplus \cdots \oplus A_n) + 1 \]
such that deconstruction followed by this function maps every element of $A_1 \oplus \cdots \oplus A_n$ to itself.
We call \emph{reconstruction} the unique left inverse that sends any input not in the image of deconstruction to the right component $1$. Then from the argument proving \Cref{prop:views}, one can also derive the following (which makes sense thanks to \Cref{rem:coproduct-as-polynomial-functor}).
\begin{lemma}\label{lem:reconstruction}
  When each $A_i$ is either $A$ or $1$, reconstruction can be seen as a \emph{copyless natural} function between polynomial functors in $A$.
\end{lemma}


\subsection{Proof}

We now present proof of the right-to-left implication in Theorem~\ref{thm:regular-functions}. Consider some transducer semigroup, with the functor being $\functor$ and the output transformation being $\outfun$. We will show that for every finite alphabet $\Sigma$, every semigroup $A$ (not necessarily a free semigroup $\Gamma^+$), the composition 
\[
    \begin{tikzcd}
    \Sigma^+ 
    \ar[r,"h"]
    &
    \functor A
    \ar[r,"\outfun_A"]
    &
    A
    \end{tikzcd}
    \]
is recognized by a functorial \sst.


We begin by introducing some notation, which allows us to track which parts of an output come from which parts of an input. The new notation will be explained using the running example of the duplicating function 
\begin{align*}
f : \set{a,b}^+ \to \set{a,b}^+
\end{align*}
which is recognized by the 
transducer semigroup 
in which the functor $\functor$ is the identity, the output mechanism is duplication, and the homomorphism 
\begin{align*}
h : \set{a,b}^+ \to \functor \set{a,b}^+ = \set{a,b}^+
\end{align*}
is the identity.

%  Let us extend the output mechanism from semigroups to co-products of semigroups. This is done as follows, for semigroups $A_1,\ldots,A_n$, define the corresponding output mechanism to be the composition of the following functions 
% \[
% \begin{tikzcd}
% \functor A_1 \oplus \cdots \oplus \functor A_n
% \ar[d, "\functor \iota_1 \oplus \cdots \oplus \functor \iota_n"]\\
% \functor (A_1 \oplus \cdots \oplus A_n) \oplus \cdots \oplus \functor (A_1 \oplus \cdots \oplus A_n)
% \ar[d, "\outfun_{A_1} \oplus \cdots \oplus \outfun_{A_n}"]\\
% A_1 \oplus \cdots \oplus A_n
% \end{tikzcd}
% \]
% We use the name \emph{co-product output mechanism} for the function described above. 
For semigroups $A_1,\ldots,A_n$, define the \emph{factorized output function} to be the 
function of type 
\begin{align*}
\functor A_1 \times \cdots \times \functor A_n \to A_1 \oplus \cdots \oplus A_n
\end{align*}
that is obtained by composing the  four functions described below
\[
\begin{tikzcd}
\functor A_1 \times \cdots \times \functor A_n
\ar[d,"\text{co-projection} \times \cdots \times \text{co-projection}"]
\\
(\functor A_1 \oplus \cdots \oplus \functor A_n)
\times
\cdots
\times 
(\functor A_1 \oplus \cdots \oplus \functor A_n)
\ar[d,"\text{semigroup operation in the co-product}"]
\\
\functor A_1 \oplus \cdots \oplus \functor A_n
\ar[d, "\functor (\text{co-projection}) \oplus \cdots \oplus \functor (\text{co-projection}) "]\\
\functor (A_1 \oplus \cdots \oplus A_n) \oplus \cdots \oplus \functor (A_1 \oplus \cdots \oplus A_n)
\ar[d, "\outfun_{A_1} \oplus \cdots \oplus \outfun_{A_n}"]\\ 
A_1 \oplus \cdots \oplus A_n.
\end{tikzcd}
\]

Let us illustrate the factorized output function on our running example, with 
\begin{align*}
A_1 = 1 \qquad A_2 = \set{a,b}^+.
\end{align*}
If we apply the factorized output function to 
\begin{align*}
(1, abbb)  \in \functor A_1 \times \functor A_2 = A_1 \times A_2
\end{align*}
then the output will be 
\begin{align*}
1 abbb 1 abbb.
\end{align*}

If we fix $n$, then the factorized output function is a natural in the sense that the following diagram commutes 
\[
\begin{tikzcd}
    [column sep=2cm]
\functor A_1 \times \cdots \times \functor A_n
\ar[r,"\text{factorized output}"]
\ar[d,"\functor h_1 \times \cdots \times \functor h_n"']
&
A_1 \oplus \cdots \oplus A_n
\ar[d,"h_1 \oplus \cdots \oplus h_n"]
\\
\functor B_1 \times \cdots \times \functor B_n
\ar[r,"\text{factorized output}"']
&
B_1 \oplus \cdots \oplus B_n
\end{tikzcd}
\]
for every semigroup homomorphisms $h_i : A_i \to B_i$. In other words, the factorized output function is a natural 
transformation of type 
\[\begin{tikzcd}
    [column sep=1cm]
    {\text{Semigroups}^n} && {\text{Sets}}
    \arrow[""{name=0, anchor=center}, "{(A_1,\ldots,A_n)} \mapsto \text{underlying set of } \functor A_1 \times \cdots \times \functor A_n", curve={height=-18pt}, from=1-1, to=1-3]
    \arrow[""{name=1, anchor=center, inner sep=0}, "{(A_1,\ldots,A_n)} \mapsto \text{underlying set of } A_1 \oplus \cdots \oplus A_n"', curve={height=18pt}, from=1-1, to=1-3]
    \arrow[ shorten <=5pt, shorten >=5pt, Rightarrow, from=0, to=1]
\end{tikzcd}\]
The reason for naturality is that each of the four steps in the definition of the factorized output is itself a natural transformation, and natural transformations compose. 


% By abuse of notation, we allow some -- but not all -- of the arguments in the factorized output to be empty; in this case the empty arguments are ignored, but the output type is still a co-product of $n$ semigroups. For example, if the input to the factorized output function is 
% \begin{align*}
% (1, \varepsilon) \in 1 \oplus \set{a,b}^+
% \end{align*}
% then the factorized  output is  
% \begin{align*}
% 1 1     \in 1 \oplus \set{a,b}^+.
% \end{align*}
    

For  input strings $w_1,\ldots,w_n \in \Sigma^+$ let use write 
\begin{align*}
\tuple{w_1| \cdots | w_n} \in \myunderbrace{A \oplus \cdots \oplus A}{$n$ times}
\end{align*}
to be the result of first applying $h$ to all the strings, and then applying the factorized output function. By abuse of notation, we allow some, but not all of the input strings to be empty, in which case the appropriate values in the co-product are ignored, but the output type is not changed. In our running example, we have 
\begin{align*}
    \tuple{abbbbb | \varepsilon | bbabaaa} =  
    \myunderbrace{abbbbb}{in first \\ copy of \\ $\set{a,b}^+$}
     \myunderbrace{bbabaaa}{in third \\ copy of \\ $\set{a,b}^+$}
     \myunderbrace{abbbbb}{in first \\ copy of \\ $\set{a,b}^+$}
     \myunderbrace{bbabaaa}{in third \\ copy of \\ $\set{a,b}^+$}.
    \end{align*}


We also use a similar notation but with some strings underlined; in the underlined case, to the non-underlined strings we apply $h$, and to the underlined strings we apply 
\[
\begin{tikzcd}
\Sigma^+ 
\ar[r,"h"]
&
\functor A 
\ar[r,"\functor !"]
&
\functor 1.
\end{tikzcd}
\]



The following lemma is the key part of our construction. The lemma speaks of a copyless natural transformation of type 
\begin{align*}
    (1 \oplus A) \times (1 \oplus A) \times (1 \oplus 1) \to 1 \oplus A.
\end{align*}
By copyless we mean that the natural transformation is copyless if its input and outputs are seen as polynomial functors, in the sense that is  explained in Example~\ref{ex:copyless-on-co-products}.

\begin{lemma}\label{lem:compute-next-configuration}
    There exists a  copyless natural transformation 
    \begin{align*}
    \delta : (1 \oplus A) \times (1 \oplus A) \times (1 \oplus 1) \to 1 \oplus A
    \end{align*} 
    and some function 
    \begin{align*}
    o : \Sigma^* \times \Sigma \times \Sigma^*  \to  (1 \oplus A) \times (1 \oplus 1)
    \end{align*}
    which is recognizable in the sense that was defined in Section~\ref{sec:functorial-sst}, such that  for every strings $w,v \in \Sigma^*$ and every letter $a \in \Sigma$ we have 
    \begin{align*}
    \tuple{wa|\underline v}  = u( \tuple{w|\underline{av}}, g(w,a,v)).
    \end{align*}
\end{lemma}
\begin{proof}
    We begin with the following claim, which shows that there is no difference if we merge parts before or after applying the factorized output operation.
    \begin{claim}\label{claim:merge-factorized-output}
        The following diagram commutes 
        \[
        \begin{tikzcd}
            [column sep=3cm]
        \Sigma^* \times \Sigma \times \Sigma^* 
        \ar[r,"{(w,a,v) \mapsto \tuple{w|a|\underline v}}"]
        \ar[d,"{(w,a,v) \mapsto (wa,v)}"']
        &
        A \oplus A \oplus 1
        \ar[d,"\text{merge first two coordinates}"]
        \\
        \Sigma^+ \times \Sigma^* 
        \ar[r,"{(wa,v) \mapsto \tuple{wa|\underline v}}"']
        &
        A  \oplus 1
        \end{tikzcd}
        \]
    \end{claim}

    Thanks to the above claim, we know that  the desired value $\tuple{wa|\underline v}$ is obtained from  $\tuple{w|a| \underline v}$
    by applying a copyless natural transformation, namely merging the first two coordinates. 
     By Lemma~\ref{lem:views}, we know that $\tuple{w|a| \underline v}$ is obtained by applying some copyless natural transformation to its three views 
     \begin{align*}
     \myunderbrace{\view_1(\tuple{w|a| \underline v}) }{in $1 \oplus A$}
    \qquad
    \myunderbrace{\view_2(\tuple{w|a| \underline v}) }{in $1 \oplus A$}
     \qquad 
     \myunderbrace{\view_3(\tuple{w|a| \underline v}) }{in $1 \oplus 1$}.
    \end{align*}
    By the same argument as in Claim~\ref{claim:merge-factorized-output}, the three views above are equal to, respectively
    \begin{align}
        \label{eq:three-views-after-post-processing}
       \tuple{w|\underline a \underline v}
       \qquad
\myunderbrace{\mathrm{merge}(\tuple{\underline w| a| \underline v}) }{here merging is the operation \\ 
that merges the two copies of $1$}
       \qquad 
        \tuple{\underline {wa}| \underline v}.
    \end{align}
    Summing up, we have shown that the desired value $\tuple{wa| \underline v}$ is obtained by applying some copyless natural transformation to the three values in~\eqref{eq:three-views-after-post-processing}. To complete the proof of the lemma, we  observe that  second and third values in~\eqref{eq:three-views-after-post-processing} depend only on the letter $a$ and the images of $w$ and $v$ under $h;\functor !$ and therefore can be obtained from $(w,a,v)$ in a recognizable way.
\end{proof}


Thanks to the above lemma, we can construct a device which is almost a functorial \sst as described in Section~\ref{sec:functorial-sst} and which recognizes our desired function $w \mapsto \tuple{w}$. We say ``almost'', because the device will use register and update functors that are  infinite polynomial functors; this construction will be later improved so that it becomes finite.  The register and update functors are 
\begin{align*}
\functorr A  = 1 \oplus A \qquad \functors A = (1 \oplus A) \times (1 \oplus 1).
\end{align*}
As mentioned above, these are  not a finite polynomial functors; we will resolve this problem shortly. Beyond that, there construction is immediate: if we take the register update term $\delta$ and the oracle $o$ as in the above lemma, and we define the initial register value to be 
\begin{align*}
\myunderbrace{\tuple{\varepsilon | \underline{a_1 \cdots a_n}}}{this does not depend on the input string $a_1 \cdots a_n$, \\ it is equal to the unique element of $1 \oplus A$ that does not use $A$}
\end{align*}
then the resulting machine will have the property that  if the input string is $a_1 \cdots a_n$, then its $i$-th configuration is 
\begin{align*}
\tuple{a_1 \cdots a_{i}| \underline{a_{i+1} \cdots a_n}}.
\end{align*}
In particular, the last configuration is $\tuple{a_1 \cdots a_n | \varepsilon}$, which is the same as the output when viewed as an element of $A \oplus 1$. Therefore, to get the output, we should apply the partial copyless natural transformation of type $A \oplus 1 \to A$ which is the one-sided inverse of the embedding of $A$ in $A \oplus 1$.  

The only remaining issue with our construction is that the two functors, i.e.~the register functor and the update functors, are not finite polynomial functors. This problem is overcome by observing that not all of $1 \oplus A$ need be used for the register values, only a small part of it, and likewise for the update functor. More formally, 
consider the natural isomorphism
\begin{align*}
1 \oplus A  \quad \simeq \quad 
\myunderbrace{\coprod_{q \in 1 \oplus 1}  A^{\dim q}}{call this the \emph{register representation} of $1 \oplus A$}
\end{align*}
that was discussed in Example~\ref{ex:co-product-as-polynomial-functor}. If we apply this isomorphism to an element of the form 
\begin{align*}
\tuple{w|\underline {v}} \in A \oplus 1,
\end{align*}
then the corresponding component will be $\tuple {\underline w | \underline v}$. Since the latter depends only  on $\underline w$ and $\underline v$, and these take values in the finite semigroup $\functor 1$, it follows that there are only finitely many components of $1 \oplus A$ that will be used to represent values from of the form $\tuple {w | \underline v}$. Therefore, instead of using $\functorr A$ to be all of $A \oplus 1$, we can restrict it to those finitely many components, giving thus a finite polynomial functor. The same argument can be applied to the update functor $\functors A$.