\section{Introduction}
\label{sec:intro}

The purpose of this paper is to give a characterization of the regular string-to-string functions. This is a fundamental class of functions, which has many equivalent descriptions: deterministic two-way automata~\cite[Note~4]{shepherdson1959reduction}, copyless streaming string transducers (\sst)~\cite[Section~3]{alurExpressivenessStreamingString2010} (or the earlier and very similar single use restricted macro tree transducers~\cite[Section~5]{MacroMSO}), \mso transductions~\cite[Theorem~13]{engelfrietMSODefinableString2001}, combinators~\cite[Section~2]{alur2014regular}, a functional programming language~\cite[Section~6]{bojanczykRegularFirstOrderList2018}, $\lambda$-calculus with linear types~\cite[Theorem~3]{LambdaTransducer} (see also~\cite[Claim~6.2]{IATLC} and~\cite[Theorem~1.2.3]{titoPhD}), decompositions \textit{à la} Krohn--Rhodes~\cite[Theorem~18, item~4]{bojanczykstefanski2020}, \ldots

The present paper adds a new characterization to the list, which uses minimal syntax, and refers only to basic concepts from algebra and category theory. We prove that a string-to-string function of type $\Sigma^* \to \Gamma^*$ is regular
%, in the sense that it is recognized by any of the equivalent models described in the previous paragraph,
if and only if it can be decomposed as 
\[
\begin{tikzcd}
    \Sigma^* 
    \ar[r,"h"]
    & 
    \functor(\Gamma^*)
    \ar[r,"\outfun_{\Gamma^*}"]
    &
    \Gamma^*
\end{tikzcd}
\]
where $\functor$ is some finiteness-preserving functor from the category of semigroups to itself, $h$~is some semigroup homomorphism, and the output function $\outfun_{\Gamma^*}$ is a natural transformation.
\tito{oops, forgot to specify that $\outfun$ goes from semigroups to sets}
%Here, a finiteness-preserving functor is one that maps finite semigroups to finite semigroups. 
%\tito{I figure this is explicited anyways in the dedicated section}

This result (\Cref{thm:regular-functions}) also extends to some other algebraic structures, such as trees modelled via forest algebra. However, our proof uses properties of the underlying algebraic structure which seem to fail for some structures such as groups or algebras corresponding to weighted automata; we do not know if our proof can be extended to  these structures, or even if the theorem itself is true.

%  this means that functors such as $A \mapsto A^2$ are allowed, but not $A \mapsto A^+$. 