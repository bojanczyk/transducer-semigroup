\subsection{From a transducer semigroup to a regular function}
\label{sec:hard}
We now turn to the difficult implication $(\ref{it:trans-semig-regular}) \Rightarrow (\ref{it:regular})$ in Theorem~\ref{thm:regular-functions}. 
% The proof is presented in a way which, if sometimes slightly verbose, makes it easier to see how it can be adapted to other algebraic structures instead of semigroups (such as forest algebras).

\subparagraph*{Functorial streaming string transducers.}
\label{sec:abstract-sst} 
The assumption of the implication uses an abstract model (transducer semigroups), while the conclusion uses a concrete operational model (streaming string transducers). To bridge the gap, we use an intermediate model, similar to streaming string transducers, but a bit more abstract. The abstraction arises by using polynomial functors instead of registers, as described below. 

Define a \emph{polynomial functor} to be a semigroup-to-set functor of the form
\begin{align*}
\qquad A \quad \mapsto \quad \coprod_{q \in Q} A^{\text{dimension of } q},
\end{align*}
where $Q$ is some possibly infinite set, called the \emph{components}, with each  component having an associated \emph{dimension} in $\set{0,1,\ldots}$. The symbol $\coprod$ stands for disjoint union of sets. This functor does not take into account the semigroup structure of the input semigroup, since the output is seen only as a set.
On morphisms, the functor works in the expected way, i.e.~coordinate-wise.  

A \emph{finite polynomial functor} is a polynomial functor with finitely many components -- for example, $A \mapsto A^2 + A^2 + A$. 
The notion of finite polynomial functor can be seen as a mild generalization of the construction which maps a semigroup $A$ to the set $A^R$ of register valuations for some fixed set $R$ of register names.  In the generalization, we allow a variable number of registers, depending on some finite information (the component). 

Having defined a more abstract notion of ``register valuations'', namely finite polynomial functors, we now define a more abstract notion of ``register updates''.  The first condition for such updates is that they do not look inside the register contents; this condition is captured by naturality as described in the following definition. 



\begin{definition}[Natural functions]\label{def:natural-functions}
    Let $\functor$  and $\functorg$ be polynomial functors, let $A$ be a semigroup. A function\footnote{This function is not necessarily a semigroup homomorphism. In fact, it would not even make sense call it a homomorphism, since the functors $\functor$ and $\functorg$ produce sets and not semigroups.}  $f : \functor A \to \functorg A$ is called \emph{natural} if it can be extended to natural transformation of type $\functor \Rightarrow \functorg$. This means that there is a family of functions, with one function
    $
    f_A : \functor A \to \functorg A
    $
    for each semigroup $A$, such  that $f=f_A$, and the the  diagram
    \[
    \begin{tikzcd}
    \functor A 
    \ar[r,"\functor h"]
    \ar[d,"f_A"']
    & 
    \functor B 
    \ar[d,"f_B"]
    \\
    \functorg A 
    \ar[r,"h"]
    &
    \functorg B
    \end{tikzcd}
    \]
    commutes for every semigroup homomorphism $h$.
\end{definition}

% Intuitively speaking, naturality says that the function is not allowed to look inside the semigroup elements that are stored in a polynomial functor $\functor A$, but it is allowed to manipulate them using the semigroup operation.

\begin{example}
    Consider the polynomial functors (where $1$ represents the singleton set $A^0$)
    \begin{align*}
    \functor A = A^* = 1 + A^1 + A^2 + \cdots  \qquad \functorg A = A + 1.     \end{align*} 
An example of a natural transformation between these two functors is the function which maps a nonempty list in $A^*$ to the product of its elements, and which maps the empty list to the unique element of $1$. A non-example is the function that maps a list $[a_1,\ldots,a_n] \in A^*$ to the leftmost element $a_i$ that is an idempotent in the semigroup, and returns $1$ if such an element does not exist. The reason why the non-example is not natural is that a semigroup homomorphism can map a non-idempotent to an idempotent.
\end{example}
% The naturality condition is an abstraction of the condition (a) that was mentioned before, namely that register contents are not inspected. For example, as we will see below, a natural transformations from the functor $\functor A = A^2$ to itself is $(a,b) \mapsto (abbaa,bbba)$, while 
% \begin{align*}
% (a,b) \mapsto \begin{cases}
%     (1,1) & \text{if the semigroup $A$ has an identity element $1$, and $a=1$}\\
%     (a,b) & \text{otherwise.}
% \end{cases}
% \end{align*}
% is not a natural transformation.


Apart from naturality, we will want our register updates to be copyless. 

\begin{definition}[Copyless natural function] \label{def:copyless} A natural function $f : \functor A \to \functorg A$ is  called \emph{copyless} if it arises from some natural transformation with the following property:  when instantiated to the semigroup\footnote{The choice of the semigroup $(\Nat,+)$ in the \Cref{def:copyless} is not particularly important. For example, the same notion of copylessness would arise if instead of $(\Nat,+)$, we used the semigroup $\set{0,1}$ with addition up to threshold $1$ (i.e.~the only way to get zero is to add two zeros). In the appendix, we present a more syntactic characterization of copyless natural transformations, which will be used later on when proving equivalence with streaming string transducers. 
    } $(\Nat,+)$, the corresponding function of type $\functor \Nat \to \functorg \Nat$ 
    does not increase the norm. Here, the norm of an element in a polynomial functor $\functor \Nat$ or $\functorg \Nat$ is defined to be the sum of numbers that appear in it.
\end{definition}

Having defined functions that are natural and copyless, we now describe the more
abstract model of \sst{}s used in our proof. The main difference is that instead
of register valuations and updates given by some finite set of register names,
we have two abstract polynomial functors, one of them finite polynomial,
together with an explicitly given application function. We also allow the
computation to be initialized and finalized in a more liberal way, that may depend on a
regular property of the input.

\begin{definition}\label{def:functorial-sst}
    The syntax of a functorial streaming string transducer is given by:
    \begin{itemize}
        \item A finite \emph{input alphabet} $\Sigma$ and an \emph{output semigroup $A$}.
    \item Two polynomial functors $\functorr$ and $\functoru$, called the \emph{register} and \emph{update} functors, with $\functorr$ \emph{finite}.
    \item A function of type $\functorr A \times \functoru A \to \functorr A$, called \emph{application}, which is natural and copyless.
    \item An \emph{initial function} $\Sigma^* \to \functorr A$ which is
      \emph{recognizable} (and therefore has finite range).
    \item A polynomial \emph{final functor} $\functork$, a \emph{final value function} $\Sigma^* \to \functork A$ which is recognizable, and a \emph{final output function} of type $\functorr A \times \functork A \to A$ which is natural and copyless.
    \item An \emph{update oracle}, which is a rational function of type $\Sigma^* \to (\functoru A)^*$.
    \end{itemize}
\end{definition}
The semantics of the functorial \sst is a function 
$\Sigma^* \to A$ defined as follows. Given an input string, we feed it to the
initial function to get an initial register valuation. As in
Definition~\ref{def:usual-sst}, we use the application function and the update
oracle to define inductively a sequence of register valuations in $\functor A$.
We then apply the final output function to the last register valuation and to the
final value (determined by the input), yielding a result in $A$.

In the appendix, we prove that this model is no more expressive than usual
copyless \sst{}s.

\begin{lemma}\label{lem:functorial-sst-complete}
    Definitions~\ref{def:usual-sst} and~\ref{def:functorial-sst} characterize the same string-to-semigroup functions.
\end{lemma}

