\section{Proof of~\Cref{thm:reco-reflecting-functions}}
    We prove a slight strenghening of the theorem, which concerns not only string-to-string functions, but also functions $f\colon A \to B$ between semigroups, not necessarily homomorphisms, such that the target semigroup $B$ is finitely generated.

    \subparagraph{(\ref{it:reco-refl}) $\Rightarrow$ (\ref{it:trans-semig-reco}).}  We  use a similar construction as in the proof of Theorem~\ref{thm:all-functions}. Let  $f\colon A \to B$  be \kl{recognizability reflecting}.  Define  a functor as follows: 
    \begin{align*}
        \functor C = A \times \text{(all semigroup homomorphisms of type $B \to C$)}.
    \end{align*}
    Similarly to Theorem~\ref{thm:all-functions}, the semigroup operation on the first coordinate of $\functor C$ is inherited from $A$, and on the second coordinate we use the \kl{left zero} semigroup structure,  where the product of $g$ and $h$ is $g$.     On morphisms, the functor is defined as in the proof of Theorem~\ref{thm:all-functions}. 

    The \kl{output mechanism}  is $(a,g) \mapsto g(f(a))$ -- think of it as function application with $f$ inserted as an interface. The \kl{transducer semigroup} thus defined recognizes the function $f$ via the homomorphism $a \in A  \mapsto  (a,\id)$.
    
    We now argue that the \kl{output mechanism} for a finite semigroup $C$ is a \kl{recognizable function}.
     To do show, we show that the inverse image of every $c \in C$ is a  recognizable subset of $\functor C$. This inverse image is
    \begin{align*}
    \qquad\qquad\qquad\qquad \bigcup_{\mathclap{g\colon B \to C\ \text{is a semigroup homomorphism}}} \; \{(a,g) \mid g(f(a))=c \}.
    \end{align*}
    Each of the sets in this union is \kl{recognizable}, by the assumption that $f$ is \kl{recognizability reflecting}.     By the assumptions that $B$ is finitely generated and that $C$ is finite, there are finitely semigroup homomorphisms $B \to C$, and therefore the union is finite. This implies that the inverse image of $c$ is recognizable, as a finite union of recognizable subsets of  $\functor C$.
    
    
\subparagraph{
     (\ref{it:reco-refl}) $\Leftarrow$ (\ref{it:trans-semig-reco}).} Take a function $f\colon A \to B$ that satisfies~(\ref{it:trans-semig-reco}), i.e.~it is a composition 
\begin{align*}
    A \xrightarrow{\;h\;} \functor B \xrightarrow{\;\outfun_B\;} B
\end{align*}
%\[\begin{tikzcd}
%    [column sep=2cm]
%	A & {\functor B} & B 
%	\arrow["h", from=1-1, to=1-2]
%	\arrow["\outfun_{B}", from=1-2, to=1-3]
%\end{tikzcd}\]
where $h$ is some homomorphism.
We want to show that $f$ is \kl{recognizability reflecting}. To prove this, let us
consider some \kl{recognizable function} from the output semigroup 
\begin{align*}
    B \xrightarrow{\;h'\;} C \xrightarrow{\;g\;} X
\end{align*}
where $C$ is a finite semigroup and $h'$ is a homomorphism.
%~a composition of some homomorphism from $B$ into a finite semigroup, followed by an arbitrary boolean-valued function
%\[
%\begin{tikzcd}
%    [column sep=2cm]
%B 
%\ar[r,"g"']
%&
%C
%\ar[r,"\text{accepting set}"']
%&
%\set{\text{yes,no}}
%\end{tikzcd}
%\]
We want to show that its precomposition $f$ is also recognizable. Consider the following diagram. 
\[\begin{tikzcd}
    [column sep=2cm]
	A & {\functor B} & & B \\
	& {\functor C} & & C & X\\
    & & D
	\arrow["h", from=1-1, to=1-2]
	\arrow["\outfun_B", from=1-2, to=1-4]
	\arrow["{\functor h'}"', from=1-2, to=2-2]
	\arrow["\outfun_C"', from=2-2, to=2-4]
	\arrow["h'", from=1-4, to=2-4]
	\arrow["g"', from=2-4, to=2-5]
    \arrow["\text{some homomorphism}"', from=2-2, to=3-3]
    \arrow["\text{some function}"', from=3-3, to=2-4]
\end{tikzcd}\]
The triangle, with $D$ \emph{finite}, describes  the assumption that the output function $\outfun_C$ is \kl{recognizable} when $C$ is finite.
The upper path from $A$ to $X$ describes the precomposition by $f$. 
 The  rectangle commutes by naturality of the output mechanism, and therefore the upper path describes the same function as the lower path from $A$ to $X$. The lower path is a \kl{recognizable function}, since the first three arrows are homomorphisms and $D$ is finite.
