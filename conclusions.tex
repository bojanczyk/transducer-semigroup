\section{Conclusions}\label{sec:conclusion}

In this paper, we have exhibited a concise algebraic characterization of the
\kl{regular string-to-string functions}, in the style of the definition of regular
languages using \kl{recognizability} by finite semigroups. To perform this extension from
languages to functions, we have relied on the basic concepts of category theory:
categories, functors, \kl{natural transformations}.

It should be noted that our use of categories is quite different in spirit from
many of the works that take a categorical perspective on automata-theoretic
results -- see for instance~\cite{ColcombetPetrisan}, whose introduction points
to many further references. In such works, the correspondence between concrete
automata models and their rephrasing as suitable (co)algebras or functors tends
to be straightforward, with the technical focus lying elsewhere (typically, in
generalizing constructions such as determinisation or minimisation). On the
contrary, we define a truly new transducer model whose equivalence with the
preexisting \kl{copyless} \kl{streaming string transducers} requires a
non-trivial proof.

An advantage of our characterization of the regular string functions is that, as
one would expect from an abstract result, it lends itself to generalizations.

\subparagraph{Semigroup-to-semigroup functions.}

The notion of recognition by a \kl{finiteness-preserving} \kl{transducer semigroup} makes
sense for functions between arbitrary semigroups. Furthermore, such functions
are closed under composition (the proof of \Cref{prop:composition} works as it
is). To check their robustness, it would be desirable to have a more concrete,
machine-like model capturing the same function class; possibly a variant of
\kl{streaming string transducers} where the underlying finite automaton is morally
\enquote{replaced} by a finite semigroup.

\subparagraph{More string functions.} Another direction is
characterizing other classes of string-to-string functions, such as the \kl{rational functions} or the polyregular functions~\cite{PolyregSurvey}. In this paper, we have discovered that,
somewhat mysteriously, combining two conditions -- naturality and preserving
finiteness -- characterizes exactly the \kl{regular functions}, which have linear
growth. Perhaps there is some way of tweaking the definitions to describe, say, some class with polynomial growth. For instance, the squaring function
(\Cref{ex:squaring}) seems to be recognized by a mixed-variance functor
$A \mapsto (A \to A) \times A$ with a dinatural output mechanism.

\subparagraph{Functions on other free algebras.} The definition of a \kl{transducer semigroup} can applied to other algebras, and not just semigroups. This may be
done by taking some monad $\monad$ and considering functions that can be
decomposed, for some endofunctor $\functor$ of the category of Eilenberg-Moore
algebras for the monad $\monad$ and some natural transformation $\outfun$, as
\[ \functort \Sigma \xrightarrow{\;\text{some $\functort$-algebra
      homomorphism}\;} \functor \functort \Gamma
  \xrightarrow{\;\outfun_{\functort \Gamma}\;} \functort \Gamma. \]
An example of this approach is forest algebras~\cite[Section 5]{bojanczyk_recobook}, which are algebras for describing trees. Preliminary work shows that, in the case of forest algebras, the suitable version of Theorem~\ref{thm:regular-functions} also holds, i.e.~the \kl{finiteness-preserving} functors lead to a characterization of the standard notion of regular tree-to-tree functions, namely \mso transductions (see~\cite{MacroMSO,FOTree}). We believe that these results apply even further, namely for graphs of bounded treewidth, modeled using suitable monads~\cite[Section 6]{bojanczyk_recobook}. The crucial property is that \Cref{prop:views}, about \kl{reconstructing} a \kl{coproduct} from its \kl{views}, holds for other monads than just the nonempty list monad for semigroups. Unfortunately, this lemma fails for some monads, such as the monad of formal linear combinations of strings that corresponds to weighted automata. In the future, we intend to conduct a more systematic investigation of the extent to which the characterizations from this paper can be generalized to other algebraic structures.

