\section{Conclusions}\label{sec:conclusion}

Another advantage of the model is that, as one would expect from an abstract result, it lends itself naturally to generalizations. 

The definition of a transducer semigroup can applied to other algebras, and not just semigroups, by taking some monad $\monad$, and considering functions that can be decomposed as 
\[
\begin{tikzcd}
    \functort \Sigma 
    \ar[r,"h"]
    & 
    \functor \functort \Gamma
    \ar[r,"\outfun_{\functort \Gamma}"]
    &
    \functort \Gamma,
\end{tikzcd}
\]
for some endofunctor $\functor$ in the category of Eilenberg-Moore algebras for the monad $\monad$, some algebra homomorphism $h$, and some natural transformation $\outfun$. An example of this approach is forest algebras~\cite[Section 5]{bojanczyk_recobook}, which are algebras for describing trees. Preliminary work shows that, in the case of forest algebras, the suitable version of Theorem~\ref{thm:regular-functions} also holds, i.e.~the finiteness-preserving functors lead to a characterization of the standard notion of regular tree-to-tree functions, namely \mso transductions (see~\cite{MacroMSO,FOTree}). We believe that these results apply even further, namely for graphs of bounded treewidth, modeled using suitable monads~\cite[Section 6]{bojanczyk_recobook}. The crucial property is that Lemma~\ref{lem:views}, about reconstructing a co-product from its views, holds for other monads than just the list monad for semigroups. Unfortunately, this lemma fails for some monads, such as the monad of linear combinations of strings that corresponds to weighted automata. In the future, we intend to conduct a more systematic investigation of the extent to which the characterizations from this paper can be generalized to other algebraic structures.

Another direction is characterizing other classes of string-to-string functions, such as the rational functions or the polyregular functions. In this paper, we have discovered that, somehow mysteriously, combining two conditions -- naturality and preserving finiteness -- characterizes exactly the regular functions, which have linear growth. Perhaps there is some way of tweaking the definitions, such that other classes of functions are described, e.g.~linear growth is replaced by polynomial growth.

\tito{Remarks about higher-order constructions and dinaturality?}