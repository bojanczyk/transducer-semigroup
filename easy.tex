\subsection{Defininition of streaming string transducers}
\label{sec:sst-definition}
In this section, we formally describe the \kl{regular functions}, using a model based on \kl{streaming string transducers} (\sst).  This model, like our proof of Theorem~\ref{thm:regular-functions}, covers a slightly more general case, namely string-to-semigroup functions instead of only string-to-string functions. These are functions of type $\Sigma^* \to A$ where $\Sigma$ is a finite alphabet and $A$ is an arbitrary semigroup.  The purpose of this generalization is to make notation more transparent, since the fact that the output semigroup consists of strings will not play any role in our proof.

% The model is a minor variation on streaming string transducers, which use registers to store elements of the output semigroup.
\AP The model uses \intro{registers} to store elements of the output semigroup. We begin by describing notation for registers and their updates. Suppose that $R$ is a finite set of \kl{register names}, and $A$ is a semigroup called the \emph{output semigroup}. We consider two sets 
\begin{center}
    \intro{register valuations}: $(R \to A)$
  \qquad
    \intro{register updates}: $(R \to (A+R)^+)$
\end{center}
Below we show two examples of \kl{register updates}, presented as assignments, using two \kl{registers} $X,Y$ and the semigroup $A = a^*$. (The right-hand sides are the values in $(A+R)^+$.)
\begin{align*}
    \myunderbrace{
    \begin{array}{rcl}
        X &:=& aYaXaaa\\
    Y &:=& XaaXaa
    \end{array}
    }{copyful}
    \qquad 
    \myunderbrace{
    \begin{array}{rcl}
        X &:=& aaYaaXaaa\\
    Y &:=& aaa
    \end{array}
    }{copyless}
    \end{align*}
\AP The crucial property is being \intro{copyless} -- a \kl{register update} is called \kl{copyless} if every \kl{register name} appears in at most one right-hand side of the \kl{update}, and in that right-hand side it appears at most once. 
The main operation on these sets is \emph{application}: a \kl{register update} $u$
can be applied to a \kl{register valuation} $v$, giving a new \kl{register valuation} $vu$. 


\AP In our model of \kl{streaming string tranducers}, the registers will be updated by a
stream of \kl{register updates} that is produced by a \kl{rational function}, defined as
follows. Intuitively speaking, a \kl{rational function} corresponds to an automaton
that produces one output letter for each input position, with the output letter
depending on regular properties of the input position within the input string.
More formally:
\begin{definition}
  A \intro{rational function} of type $\Sigma^* \to X^*$ -- where $\Sigma$ is
  a finite alphabet but $X$ can be any set -- is a length-preserving\footnote{Often in the literature, rational
    functions are not required to be length-preserving, see
    e.g.~\cite[p.~525]{sakarovitch2009elements}, but in this paper, we only need
    the length-preserving case.} function with the following property: for some~family\footnote{The family $(f_a)_{a\in\Sigma}$ is very close to what is called an \emph{(Eilenberg) bimachine} in the literature.}
  \[ \qquad\qquad f_a \colon \myunderbrace{\Sigma^* \times \Sigma^*}{\qquad equipped with componentwise multiplication} \to \Gamma\quad \text{for $a \in \Sigma$ of \kl{recognizable} functions,} \]
  for every input $a_1 \dots a_n$ and $i\in\{1,\dots,n\}$, the $i$-th output letter is $f_{a_i}(a_1 \dots a_{i-1},\; a_{i+1} \dots a_n)$.
\end{definition}
Note that the range of a rational function with codomain $X^*$ may contain only
finitely many \enquote{letters} from $X$, so it can always be
seen as a string function over finite alphabets.

\AP Having defined \kl{register updates} and \kl{rational functions}, we are ready to introduce the machine model used in this paper as the reference definition of \intro{regular functions}.

\begin{definition}\label{def:usual-sst}
    The syntax of a \intro{streaming string transducer} (\sst) is given by:
\begin{itemize}
    \item A finite \emph{input alphabet} $\Sigma$ and an \emph{output semigroup $A$}.
    \item A finite set $R$ of \emph{register names}. All \kl{register valuations} and \kl{updates} below use $R$ and $A$.
    \item A designated \intro{initial register valuation}, and a \intro{final
        output pattern} in $R^+$ (that does not need to be copyless, though
      adding this restriction would not affect the expressive power).
    \item An \intro{update oracle}, which is a \kl{rational function} of type 
        $\Sigma^* \to (\text{copyless register updates})^*$.
\end{itemize}
\end{definition}
The semantics of the \sst{} is a function of type $\Sigma^* \to A$ defined as follows. When given an input string, the \sst{} begins in the designated \kl{initial register valuation}. Next, it applies all \kl{updates} produced by the \kl{update oracle}, in left-to-right order. Finally, the output of the \sst{} is obtained by combining the last register values according to the \kl{final output pattern}.

\begin{example}
  We define an \sst{} that computes the function of \Cref{ex:prefix-suffix}. It has two registers $X$ and $Y$, whose initial valuation is $X=Y=\varepsilon$, and the final output pattern is $YX$. The update associated to an input letter $\ell\in\{a,b,c\}$ at position $i$ is:
  \begin{itemize}
    \item if the position $i$ is part of the longest $c$-free prefix, then $X := X\ell$, otherwise $X:=X$;
    \item if the position $i$ is part of the longest $c$-free suffix, then $Y := Y\ell$, otherwise $Y:=Y$.
  \end{itemize}
  This sequence of updates can be produced by a rational function generated by a family of functions $(f_\ell)_{\ell\in\{a,b,c\}}$ that are \kl{recognized} by $\mathbb{B}^2$, where $\mathbb{B}$ is the monoid of booleans with conjunction (rephrase the conditions as \enquote{there is no $c$ to the left (resp.~right) of $i$}).
\end{example}

In a \kl{rational function}, the label of the $i$-th output position is allowed to depend on letters of the input string that are on both sides of the $i$-th input position; this corresponds to regular lookaround in a streaming string transducer. Therefore, the model described above is easily seen to be equivalent to copyless \sst{}s with regular lookaround, which are one of the equivalent models defining the regular string-to-string functions, see~\cite[Section~IV.C]{AlurFT12}.

\subsection{From a regular function to a transducer semigroup}
\label{sec:easy}

\AP Having defined the transducer model, we prove the easy implication in
\Cref{thm:regular-functions}. It is apparent from \Cref{def:usual-sst} that every
\kl{regular function} can be decomposed as a \kl{rational function} followed by a function
computed by a \kl{streaming string transducer} whose $i$-th \kl{register update} depends
only on the $i$-th input letter -- let us call that a \intro{local} \sst. Thanks
to closure under composition (\Cref{prop:composition}), we only need to handle
these two special cases: we show that \kl{finiteness-preserving} \kl{transducer
  semigroups} recognize
\begin{itemize}
\item all \kl{rational functions} in \Cref{sec:rational};
\item and all \kl{local} \kl{streaming string transducers} in \Cref{sec:local}.
\end{itemize}

\subsubsection{Recognizing rational functions by transducer semigroups}
\label{sec:rational}

Consider a \kl{rational function}, generated by the family $(f_a)_{a\in\Sigma}$ of \kl{recognizable functions} of type
$\Sigma^* \times \Sigma^* \to \Gamma$. By
definition of recognizability, each $f_a$ decomposes into
\[ \Sigma^* \times \Sigma^* \xrightarrow{\;h_a\;} B_a \xrightarrow{\;g_a\;} \Gamma
  \qquad\text{where $h_a$ is a semigroup homomorphism and $B_a$ is finite.} \]
One can check that every $f_a$ then factors through a monoid morphism to the
finite monoid
\[ \prod_{a \in \Sigma} h_a(\Sigma^* \times \Sigma^*) \]
Thus, without loss of generality, we may assume for the rest of the proof that
all of the above semigroups $B_a$ are equal to a common finite monoid $B$ and
that each semigroup homomorphism $h_a$ is in fact a monoid morphism.

For any semigroup $A$, we let\footnote{A construction similar in
  spirit to the classical \emph{two-sided semidirect
    product}~\cite[\S6]{rhodes1989kernel}.} $\functor A = B \times (B\to A) \times B$, endowed with the following semigroup operation:
\begin{align*}
  (\ell_1,\varphi_1,r_1) \cdot (\ell_2,\varphi_2,r_2) = \Big(\ell_1\ell_2,\; \big(b  \mapsto \varphi_1(br_2) \cdot \varphi_2(\ell_1b)\big),\; r_1r_2\Big).
\end{align*}
The construction $\functor$ is extended to morphisms by considering $B\to A$ as the set of $B$-indexed tuples (cf.~\Cref{ex:functors}) of
elements of $A$. To get a \kl{transducer semigroup}, we take the \kl{output mechanism} to
be $(\ell,\varphi,r) \mapsto \varphi(e)$ where $e \in B$ is the neutral element.

Our \kl{rational function} is then recognized by the unique monoid homomorphism of
type $\Sigma^* \to \functor(\Gamma^*)$ (indeed, $\functor$ preserves monoids)
which maps $a \in \Sigma$ to
$\big(h_a(a,\varepsilon),g_a,h_a(\varepsilon,a)\big)$.

\subsubsection{From a local SST to a transducer semigroup}
\label{sec:local}

\AP Suppose now that a string-to-semigroup function $f\colon \Sigma^* \to A$ is
computed by some \kl{local} \kl{streaming string transducer}. In the proof below, when referring to \kl{register valuations} and \kl{register updates}, we refer to those that use the \kl{registers} and output semigroup of the fixed transducer. We say that a \kl{register update} is in \intro{normal form} if, in every right-hand side, one cannot find two consecutive letters from the semigroup $A$.
Any \kl{register update} can be \intro{normalized}, i.e.~converted into one that is in \kl{normal form}, by using the semigroup operation to merge consecutive elements of the output semigroup in the right-hand sides.
Here is an example, which uses three registers $X,Y,Z$ and the semigroup $A = (\set{0,1}, \cdot)$:
\begin{align*}
  \myunderbrace{
  \begin{array}{rcl}
    X &:=& 01Y1111X111\\
    Y &:=& 01011
  \end{array}
  }{not in \kl{normal form}}
  \qquad \xrightarrow{\;\text{\kl{normalization}}\;} \qquad
  \myunderbrace{
  \begin{array}{rcl}
    X &:=& 0Y1X1\\
    Y &:=& 0
  \end{array}
  }{in \kl{normal form}}
\end{align*}
The \kl{register updates} before and after \kl{normalization} act in the same way on
\kl{register valuations}. If an \kl{update} is \kl{copyless} and in \kl{normal form}, then the
combined length of all right-hand sides is at most three times the number of
registers. Therefore, if a semigroup $A$ is finite, then the set of \kl{copyless}
\kl{register updates} in \kl{normal form}, call it $\functoru A$, is also finite.
(However, there are infinitely many copyful register updates even when $A$ is
finite.) This set $\functoru A$ can be equipped with a composition operation
\begin{align*}
    u_1,u_2 \in \functoru A  \quad \mapsto \quad u_1u_2 \in \functoru A,
\end{align*}
which is defined in the same way as applying a register update to a register
valuation, except that we normalize at the end. This composition operation is
associative, and  compatible with applying \kl{register updates} to \kl{register valuations}, in the sense that $(vu_1)u_2 = v(u_1u_2)$ holds for every \kl{valuation} $v$ and all \kl{updates} $u_1$ and $u_2$. Therefore, $A \mapsto \functoru A$ is a \kl{finiteness-preserving} semigroup-to-semigroup functor (with the natural extension to morphisms, where the homomorphism is applied to every semigroup element in a right-hand side). 

The functor $\functoru$ described above is almost but not quite the functor that
will be used in the \kl{transducer semigroup} that we will define to prove the easy
implication in Theorem~\ref{thm:regular-functions}. That functor $\functor$ will
also take into account the \kl{initial register valuation}:
\[ \functor A =  \functoru A \times \myunderbrace{(R \to A)}{\qquad\qquad\qquad\qquad\qquad endowed with the trivial \kl{left zero} semigroup structure} \quad\text{with componentwise multiplication \& action on morphisms} \]
Given $(u,v) \in \functor A$, the \kl{output mechanism} in the \kl{transducer semigroup}
applies the \kl{register update} $u$ to the \kl{register valuation} $v$, and then multiplies together the register values given by the resulting \kl{valuation} $vu$ according to the \kl{final output pattern}.
Using this, we can recognize $f$ via the homomorphism that sends each input letter to:
\begin{itemize}
\item the \kl{register update} that this letter determines (our \sst being \kl{local}) in the first component;
\item the designated \kl{initial register valuation} in the second component.
\end{itemize}

