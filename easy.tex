\subsection{Defininition of streaming string transducers}
\label{sec:sst-definition}
In this section, we formally describe the regular functions, using a model based on streaming string transducers (\sst).  This model, like our proof of Theorem~\ref{thm:regular-functions}, covers a slightly more general case, namely string-to-semigroup functions instead of only string-to-string functions. These are functions of type $\Sigma^* \to A$ where $\Sigma$ is a finite alphabet and $A$ is an arbitrary semigroup.  The purpose of this generalization is to make notation more transparent, since the fact that the output semigroup consists of strings will not play any role in our proof.

% The model is a minor variation on streaming string transducers, which use registers to store elements of the output semigroup.
The model uses registers to store elements of the output semigroup. We begin by describing notation for registers and their updates. Suppose that $R$ is a finite set of \emph{register names}, and $A$ is a semigroup called the \emph{output semigroup}. We consider two sets 
\begin{align*}
    \text{register valuations:}\ (R \to A)
  \qquad
    \text{register updates:}\ (R \to (A+R)^+)
\end{align*}
Below we show two examples of register updates, presented as assignments, using two registers $X,Y$ and the semigroup $A = a^*$. (The right-hand sides are the values in $(A+R)^+$.)
\begin{align*}
    \myunderbrace{
    \begin{array}{rcl}
        X &:=& aYaXaaa\\
    Y &:=& XaaXaa
    \end{array}
    }{copyful}
    \qquad 
    \myunderbrace{
    \begin{array}{rcl}
        X &:=& aaYaaXaaa\\
    Y &:=& aaa
    \end{array}
    }{copyless}
    \end{align*}
The crucial property is being copyless -- a register update is called copyless if every register name appears in at most one right-hand side of the update, and in that right-hand side it appears at most once. 
The main operation on these sets is \emph{application}: a register update $u$
can be applied to a register valuation $v$, giving a new register valuation $vu$. 


In our model of streaming string tranducers, the registers will be updated by a
stream of register updates that is produced by a {rational function}, defined as
follows. Intuitively speaking, a rational function corresponds to an automaton
that produces one output letter for each input position, with the output letter
depending on regular properties of the input position within the input string.
More formally, a \emph{rational function} of type $\Sigma^* \to \Gamma^*$ is
defined to be a length-preserving\footnote{Often in the literature, rational
  functions are not required to be length-preserving, see
  e.g.~\cite[p.~525]{sakarovitch2009elements}, but in this paper, we only need
  the length-preserving case.} function such that for some family of
\emph{recognizable} functions\footnote{The family $(f_a)_{a\in\Sigma}$ is very
  close to what is called an \emph{(Eilenberg) bimachine} in the literature.}
\begin{align*}
 f_a \colon \Sigma^* \times \Sigma^* \to \Gamma^* \qquad\text{for}\ a \in \Sigma,
\end{align*}
for every input string $a_1 \dots a_n$ the $i$-th output letter is $f_{a_i}(a_1
\dots a_{i-1},\; a_{i+1} \dots a_n)$.

Having defined register updates and rational functions, we are ready to define the variant of streaming string transducers used in this paper.



\begin{definition}\label{def:usual-sst}
    The syntax of a streaming string transducer (\sst) is given by:
\begin{itemize}
    \item A finite \emph{input alphabet} $\Sigma$ and an \emph{output semigroup $A$}.
    \item A finite set $R$ of \emph{register names}. All register valuations and updates below use $R$ and $A$.
    \item A designated \emph{initial register valuation}, and a designated \emph{final register}.
    \item An update oracle, which is a rational function of type 
        $\Sigma^* \to (\text{copyless register updates})^*$.
\end{itemize}
\end{definition}
The semantics of the \sst{} is a function of type $\Sigma^* \to A$ defined as follows. When given an input string, the \sst{} begins in the designated initial register valuation. Next, it applies all updates produced by the update oracle, in left-to-right order. Finally, the output of the \sst{} is obtained by returning the semigroup element stored in the designated final register.

In a rational function, the label of the $i$-th output position is allowed to depend on letters of the input string that are on both sides of the $i$-th input position; this corresponds to regular lookaround in a streaming string transducer. Therefore, the model described above is easily seen to be equivalent to copyless \sst{}s with regular lookaround, which are one of the equivalent models defining the regular string-to-string functions, see~\cite[Section~IV.C]{AlurFT12}.

\subsection{From a regular function to a transducer semigroup}
\label{sec:easy}

Having defined the transducer model, we prove the easy implication in
\Cref{thm:regular-functions}. First, we treat the special case of rational
functions:
\begin{lemma}\label{lem:rational-to-functor}
  Finiteness-preserving transducer semigroups recognize all rational functions.
\end{lemma}
\begin{proof}
  By definition, a rational function comes from a finite family of recognizable
  functions. Recognizability means that they factor through homomorphisms to
  some finite semigroups -- without loss of generality, we may consider that
  they all use the same semigroup $B$. To recognize the rational function, we
  then consider a functor that maps a semigroup $A$ to a semigroup whose
  underlying set is $B \times (B \to A) \times B$ (a construction similar in
  spirit to the classical \emph{two-sided semidirect
    product}~\cite[\S6]{rhodes1989kernel}). Details can be found in the
  appendix.
\end{proof}

For the general case, it is apparent from \Cref{def:usual-sst} that every
regular function can be decomposed as a rational function followed by a function
computed by a streaming string transducer whose $i$-th register update depends
only on the $i$-th input letter -- let us call that a \emph{local} \sst. Since
rational functions are covered by the above lemma, and composition by
\Cref{prop:composition}, we only need to handle the case of local streaming
string transducers.

So, suppose now that a string-to-semigroup function $f\colon \Sigma^* \to A$ is
computed by some local \sst. In the proof below, when referring to register valuations and register updates, we refer to those that use the registers and output semigroup of the fixed transducer. We say that a register update is in \emph{normal form} if, in every right-hand side, one cannot find two consecutive letters from the semigroup $A$.
Any register update can be \emph{normalized}, i.e.~converted into one that is in normal form, by using the semigroup operation to merge consecutive elements of the output semigroup in the right-hand sides.
Here is an example, which uses three registers $X,Y,Z$ and the semigroup $A = (\set{0,1}, \cdot)$:
\begin{align*}
  \myunderbrace{
  \begin{array}{rcl}
    X &:=& 01Y1111X111\\
    Y &:=& 01011
  \end{array}
  }{not in normal form}
  \qquad \xrightarrow{\;\text{normalization}\;} \qquad
  \myunderbrace{
  \begin{array}{rcl}
    X &:=& 0Y1X1\\
    Y &:=& 0
  \end{array}
  }{in normal form}
\end{align*}
The register updates before and after normalization act in the same way on
register valuations. If an update is copyless and in normal form, then the
combined length of all right-hand sides is at most three times the number of
registers. Therefore, if a semigroup $A$ is finite, then the set of copyless
register updates in normal form, call it $\functors A$, is also finite.
(However, there are infinitely many copyful register updates even when $A$ is
finite.) This set $\functors A$ can be equipped with a composition operation
\begin{align*}
    u_1,u_2 \in \functors A  \quad \mapsto \quad u_1u_2 \in \functors A,
\end{align*}
which is defined in the same way as applying a register update to a register
valuation, except that we normalize at the end. This composition operation is
associative, and  compatible with applying register updates to register
valuations, in the sense that $(vu_1)u_2 = v(u_1u_2)$ holds for every valuation
$v$ and all updates $u_1$ and $u_2$. Therefore, $A \mapsto \functors A$ is a finiteness-preserving semigroup functor (with the natural extension to morphisms, where the homomorphism is applied to every semigroup element in a right-hand side). 

The functor $\functors$ described above is almost but not quite the functor that
will be used in the transducer semigroup that we will define to prove the easy
implication in Theorem~\ref{thm:regular-functions}. That functor $\functor$ will
also take into account the initial register valuation: if the input semigroup is~$B$, then the underlying set of the output semigroup is
\[ \functor A =  \functors A \times \myunderbrace{(R \to A)}{\qquad\qquad\qquad\qquad\qquad endowed with the trivial left zero semigroup structure} \]
The semigroup operation on $\functor A$, and the action of $\functor$ on
morphisms, are defined componentwise.

Given $(u,v) \in \functor A$, the output mechanism in the transducer semigroup
applies the register update $u$ to the register valuation $v$, and then from the
resulting valuation $vu$ extracts the value of the designated final register.
Using this, we can recognize $f$ via the homomorphism that sends each input letter to:
\begin{itemize}
\item the register update that this letter determines (our \sst being local) in
  the first component;
\item the designated initial register valuation in the second component.
\end{itemize}

